%!TEX root = ../../super_main.tex

% \translated
% Will indicate that a given string is translated from danish
% arg1    the original string
% arg2    the string that has been translated
\newcommand{\translatedwithsource}[2] {
  \emph{``#1'' (translated: ``#2'')}\xspace
}

% \translated
% Will indicate that a given string is translated from danish
% arg2    the string that has been translated
\newcommand{\translated}[1] {
  \emph{``#1'' (translated)}\xspace
}

% \centerfig
% Will insert a figure with 75% textwidth in the horizontal center of the page.
% arg1 		graphics file to include.
% arg2 		the caption of the figure.
% arg3		the label of the figure.
\newcommand{\centerfig}[3] {
  \begin{figure}[htbp]
    \centering
    \includegraphics[width=0.75\textwidth]{#1}
    \caption{#2}
    \label{#3}
  \end{figure}
  \noindent
}

% \centerfigwithwidth
% Will insert a figure with a custom width in the horizontal center of the page.
% arg1 		graphics file to include.
% arg2		the caption of the figure.
% arg3		the label of the figure.
% arg4 		the width of the figure. For instance 0.5 for 50% of the page.
\newcommand{\centerfigwithwidth}[4] {
  \begin{figure}[!htbp]
    \centering
    \includegraphics[width=#4]{#1}
    \caption{#2}
    \label{#3}
  \end{figure}
  \noindent
}

% \wrapfig
% Will insert a figure wrapping with the text of the page.
% arg1		the graphics file to include.
% arg2 		the caption of the figure.
% arg3 		the label of the figure.
% arg4		what side the figure should wrap to. l for left, r for right.
\newcommand{\wrapfig}[4] {
  \begin{wrapfigure}{#4}{0.5\textwidth}
    \begin{center}
      \includegraphics[width=0.5\textwidth]{#1}
    \end{center}
    \caption{#2}
    \label{#3}
  \end{wrapfigure}
  \noindent
}

% \wrapfigwithwidth
% Will insert a figure wrapping with the text of the page.
% arg1		the graphics file to include.
% arg2 		the caption of the figure.
% arg3 		the label of the figure.
% arg4		what side the figure should wrap to. l for left, r for right.
% arg5 		the width of the figure. For instance 0.5 for 50% of the page. 
\newcommand{\wrapfigwithwidth}[5] {
  \begin{wrapfigure}{#4}{0.5\textwidth}
    \begin{center}
      \includegraphics[width=#5]{#1}
    \end{center}
    \caption{#2}
    \label{#3}
  \end{wrapfigure}
  \noindent
}

% Use the newfloat package (For labeling and adding  grammars)
\usepackage{newfloat}

% declare the floating environment {Grammar}
% this will also define \listofGrammars:
\DeclareFloatingEnvironment[
  % the file extension for the file used to create the list:
  fileext   = logr,% don't use log here!
  % the heading for the list:
  listname  = {List of Grammars},
  % the name used in captions:
  name      = Grammar,
  % the default floating parameters if the environment is used
  % without optional argument:
  placement = htbp
]{Grammar}

% Set the grammar indent to 80 pt. The indent indicates the space between LHS and RHS.
\grammarindent = 80pt

% Define how non-terminals should look
\renewcommand{\syntleft}{$<$\ttfamily}
\renewcommand{\syntright}{$>$}

% \cfg
% used for declaring cfgs
% arg1    the graphics file to include.
% arg2    the caption of the figure.
% arg3    the label of the figure.
\newcommand{\cfg}[3] {
  \begin{Grammar}
    \ttfamily
    \begin{grammar}
      #1
    \end{grammar}

    \caption{#2}
    \label{#3}
  \end{Grammar}
}



% declare the floating environment {Equation}
% this will also define \listofEquations:
\DeclareFloatingEnvironment[
  % the file extension for the file used to create the list:
  fileext   = logr,% don't use log here!
  % the heading for the list:
  listname  = {List of Equations},
  % the name used in captions:
  name      = Equation,
  % the default floating parameters if the environment is used
  % without optional argument:
  placement = htbp
]{Equation}