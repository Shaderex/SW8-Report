%!TEX root = ../super_main.tex

% ========================================================= %
%         _____           _                         		% 
%        |  __ \         | |                                %
%        | |__) |_ _  ___| | ____ _  __ _  ___  ___ 		%
%        |  ___/ _` |/ __| |/ / _` |/ _` |/ _ \/ __|        %
%        | |  | (_| | (__|   < (_| | (_| |  __/\__ \        %
%        |_|   \__,_|\___|_|\_\__,_|\__, |\___||___/        %
%                                    __/ |                  %
%                                   |___/                   %
% ========================================================= %

% Margin
\usepackage[a4paper, inner = 4cm, outer = 2.4cm, top = 2.4cm, bottom = 2.4cm, pdftex]{geometry}

% Line numbering
\usepackage{lineno}

% Colors packages
% Default colors can be found here http://en.wikibooks.org/wiki/LaTeX/Colors#The_68_standard_colors_known_to_dvips
\usepackage[usenames, dvipsnames]{xcolor} 
\usepackage{colortbl}
\definecolor{eclipse_blue}{RGB}{46,55,247}
\definecolor{eclipse_red}{RGB}{147,86,135}

% Language support
\usepackage[T1]{fontenc}
\usepackage[utf8]{inputenc}

% Used for making CFG 
% Rounded is to use EBNF
% Nounderscore is to use underscores in label names
\usepackage[rounded, nounderscore]{syntax}

% Used to make a directory Tree (CD content etc.)
\usepackage{dirtree}

% Fonts
\usepackage{lmodern}
\usepackage{courier}

% Use any font size
\usepackage{anyfontsize}

% Greek letters
\usepackage{textgreek}

% Used for making pretty math
\usepackage{amssymb}
\usepackage{amsmath}
\usepackage{stmaryrd}
\usepackage{upquote} % to use ^

% Adds pretty commas in math expressions
\usepackage{icomma}

% FiXme comments (fxnote)
\usepackage{fixme}

% Setup for the FiXme comments 
\fxsetup{
    status = draft,
    author = Comment,
    layout = footnote, % also try footnote or pdfnote
    theme = color
}

%todo setup
\usepackage[textwidth=2.0cm, textsize=small]{todonotes}
\setlength{\marginparwidth}{2cm}

% Graphics
\usepackage{graphicx}

% Captions (for figures etc.)
\usepackage[hang, footnotesize, bf]{caption}

% Used for making multiple figures within one figure
\usepackage{wrapfig}
\usepackage{subcaption}

% Custom styling for titles (chapter, section etc.)
\usepackage{titlesec}

% Special notation for appendix chapters
\usepackage[titletoc]{appendix}

% Used for floatbarriers
\usepackage{placeins}

% Used for double hlines
\usepackage{hhline}

% Used for \pdfbookmarks. Hidelinks will hide red boxes in some pdf-viewers.
\usepackage[bookmarks = true, hidelinks]{hyperref}

% Improved linebreaking
\usepackage{microtype}

% Used for tabels
\usepackage{booktabs}
\usepackage{rotating}

% Used for lists
\usepackage{enumerate}

% Header and footer on page
\usepackage{lastpage}
\usepackage{afterpage}
\usepackage{fancyhdr}

% Used for making multiple columbs
\usepackage{multicol}

% Used for making multiple rows
\usepackage{multirow}

% Used for generating lipsum text
\usepackage{lipsum}

% Used for making space after defined commands.
% This way we can write \command instead of \command{}.
\usepackage{xspace}

% Will show the name of the label in the margin (Uncomment to show labels)
% \usepackage[inline]{showlabels}

% Package used to make spacing between items in itemize and enumarations smaller.
\usepackage{enumitem}

% Makes it possible to embed PDF documents
\usepackage{pdfpages}


% ========================================================= %
% / / / / / / / / / / / / / / / / / / / / / / / / / / / / / %
% ========================================================= %



% ========================================================= %
%                _____       _ _   _       _ 				%
%               |_   _|     (_) | (_)     | |				%
%                 | |  _ __  _| |_ _  __ _| |				%
%                 | | | '_ \| | __| |/ _` | |				%
%                _| |_| | | | | |_| | (_| | |				%
%               |_____|_| |_|_|\__|_|\__,_|_|				%
%															%
% ========================================================= %

% Fixes some orphans (horeunger)
\widowpenalty = 300
\clubpenalty = 300

% All chapters must start on a new uneven page
\let\origdoublepage\cleardoublepage

\newcommand{\clearemptydoublepage}{
  \clearpage
  {\pagestyle{empty}\origdoublepage}
}
\let\cleardoublepage\clearemptydoublepage
\let\originalchapter=\chapter
\def\chapter{\cleardoublepage\originalchapter}

% Justification for captions
\captionsetup{justification = justified}

% Define graphics pack
\graphicspath{{graphic/}}

% Used for page header and footer
\pagestyle{fancyplain}
\fancyhf{}
\fancyhead[RO]{\slshape \rightmark}
\fancyhead[LE]{\slshape \leftmark}

% Set head height
\setlength{\headheight}{14.2pt}

%Set depth of toc to include sections
\setcounter{tocdepth}{1}

% ========================================================= %
% / / / / / / / / / / / / / / / / / / / / / / / / / / / / / %
% ========================================================= %



% ========================================================= %
%                 _____      _                				%
%                / ____|    | |               				%
%               | |     ___ | | ___  _ __ ___ 				%
%               | |    / _ \| |/ _ \| '__/ __|				%
%               | |___| (_) | | (_) | |  \__ \				%
%                \_____\___/|_|\___/|_|  |___/				%
%															%
% ========================================================= %

% Color used for fxnotes
\colorlet{fxnote}{Red}

% Primary color (Used for chapter headline etc.)
\colorlet{primarycolor}{Blue}

% ========================================================= %
% / / / / / / / / / / / / / / / / / / / / / / / / / / / / / %
% ========================================================= %                               



% ========================================================= %
%                   _____ _         _      					%
%                  / ____| |       | |     					%
%                 | (___ | |_ _   _| | ___ 					%
%                  \___ \| __| | | | |/ _ \					%
%                  ____) | |_| |_| | |  __/					%
%                 |_____/ \__|\__, |_|\___|					%
%                              __/ |       					%
%                             |___/        					%
% ========================================================= %

% Style for definitions
%!TEX root = ../../super_main.tex

% Packages used
\usepackage{amsthm}
\usepackage{thmtools}

% ================================================ %

% Style for the "definitionstyle" definition
\declaretheoremstyle
[
  spaceabove = \parsep, 
  spacebelow = \parsep,
  notebraces = {}{},
  headpunct = {},
  postheadspace = \newline,
  headformat = {{\NAME} {\NUMBER}:{\NOTE}},
  mdframed = {
    backgroundcolor = gray!15, 
    linecolor = gray!60,
    linewidth = 5pt,
    skipabove = 1em,
    skipbelow = 1em,
    innertopmargin = 0.7em,
    innerbottommargin = 0.7em,
    roundcorner = 0em,
    skipbelow = \parsep,
    skipbelow = \parsep,
    topline = false,
    bottomline = false,
    rightline = false,
  },
]
{definitionstyle}

% Declare a new theorem used for definitions
\declaretheorem
[
  style = definitionstyle,
  name = Definition,
  numberwithin = chapter,
]
{definition}

% Style for lstlistings (code snippets)
%!TEX root = ../../super_main.tex

% Packages used
\usepackage{listings}

% Package used to display pseudo code (algorithms)

\usepackage{algorithm}% http://ctan.org/pkg/algorithms
%\usepackage{algorithmic}
\usepackage{algpseudocode}% http://ctan.org/pkg/algorithmicx

% ================================================ %

\captionsetup[lstlisting]{
    format = listing
}

% Default style for all lstlistings
\lstset 
{
    backgroundcolor = \color{white},
    keywordstyle = \color{blue},
    commentstyle = \color{ForestGreen!80}\textit,
    stringstyle = \color{green}\textbf,
    basicstyle = \scriptsize\ttfamily\bfseries,
    numberstyle = \tiny,
    numbers = left,
    breaklines = true,
    breakatwhitespace=true,
    showstringspaces = false,
    tabsize = 3,
    captionpos = b,
    extendedchars = true,
    escapeinside = {//*}{\^^M}, % Use latex inside lstlistings. For instance for refferences.
    frame = tblr,
    backgroundcolor = \color{gray!5},
    xleftmargin = 3.5pt,
}

% Write "Code snippet" instead of "listing".
\renewcommand{\lstlistingname}{Code Snippet}

% General style for the whole lstlisting
\DeclareCaptionFormat{listing}{#1#2#3}
\captionsetup[lstlisting]{format = listing}

\lstdefinelanguage{TAInC} 
{
    morekeywords=[1]{number, boolean}, 
    keywordstyle=[1]\color{NavyBlue},
    morekeywords=[2]{do, if, else, run, when, startup, return, NAME},
    keywordstyle=[2]\color{Magenta},
    morekeywords=[3]{Tank, Gun, Battle},
    keywordstyle=[3]\color{Green},
    morekeywords=[4]{BattleEnded,BulletHit, BulletHitBullet, BulletMissed, Death, HitByBullet, HitTank, HitWall, TankDeath, RoundEnded, ScannedTank, Win},
    keywordstyle=[4]\color{Bittersweet},
    morekeywords=[5]{action, calculation},
    keywordstyle=[5]\color{Fuchsia},
    sensitive = true,
    morecomment = [l]{//},
    morecomment = [n]{/*}{*/}
}

% Custom lststyle named tainc
\lstdefinestyle{tainc}
{
    breaklines = true,
    columns = fullflexible,
    language = TAInC
}

% Custom lstinline for TAInC named taincinline
\newcommand{\taincinline}[1]{\lstinline[style = tainc, basicstyle = \ttfamily\normalsize]{#1}}

% Import the Java language
\lstloadlanguages{Java}

% Custom lststyle named java
\lstdefinestyle{java}
{
    breaklines = true,
    language = Java,
    columns = fullflexible,
    stringstyle=\color{eclipse_blue}\textbf,
    morekeywords=[1]{class, return}, 
    keywordstyle=[1]\color{eclipse_red}
}

% Custom lstinline for Java named javainline
\newcommand{\javacodeinline}[1]{\lstinline[style = java, basicstyle = \ttfamily\normalsize]{#1}}

% Custom highlightning for Xtext
\lstdefinelanguage{Xtext} 
{
    morekeywords=[1]{returns, current, terminal}, 
    keywordstyle=[1]\color{DarkOrchid},
    morekeywords=[2]{startup, run, do, action},
    keywordstyle=[2]\color{eclipse_blue},
    morekeywords=[3]{name},
    keywordstyle=[3]\color{red},
    stringstyle=\color{Goldenrod},
    sensitive = true,
    morecomment = [l]{//},
    morecomment = [n]{/*}{*/}
}

% Custom lststyle named xtext
\lstdefinestyle{xtext}
{
  breaklines = true,
  language = Xtext,
  columns = fullflexible
}

% Custom lstinline for Xtext named xtextinline
\newcommand{\xtextinline}[1]{\lstinline[style = xtext, basicstyle = \ttfamily\normalsize]{#1}}

\lstdefinelanguage{Xtend} 
{
    morekeywords=[1]{class, extends, new, extension, import, package, def, if, else, val, var, instanceof, null, true, false, return, default, private, case, switch, dispatch}, 
    keywordstyle=[1]\color{DarkOrchid},
    stringstyle=\color{eclipse_blue},
    sensitive = true,
    morecomment = [l]{//},
    morecomment = [n]{/*}{*/}
}

\lstdefinestyle{xtend}
{
    breaklines = true,
    columns = fullflexible,
    language = Xtend,
    stringstyle=\color{eclipse_blue},
    showstringspaces=false
}

% Custom lstinline for Xtend named xtendinline
\newcommand{\xtendinline}[1]{\lstinline[style = xtend, basicstyle = \ttfamily\normalsize]{#1}}

% Import the C language
\lstloadlanguages{C}

% Custom lststyle named c
\lstdefinestyle{c}
{
    breaklines = true,
    language = C,
    columns = fullflexible,
    stringstyle=\color{eclipse_blue},
    morekeywords=[1]{return}, 
    keywordstyle=[1]\color{eclipse_red},
}

% Custom lstinline for C named cinline
\newcommand{\cinline}[1]{\lstinline[style = c, basicstyle = \ttfamily\normalsize]{#1}}

% Import the C# language
\lstloadlanguages{[Sharp]C}

% Custom lststyle named cs
\lstdefinestyle{cs}
{
    breaklines = true,
    language = [Sharp]C,
    columns = fullflexible,
    stringstyle=\color{eclipse_blue},
    morekeywords=[1]{return}, 
    keywordstyle=[1]\color{eclipse_red},
}

% Custom lstinline for C# named csinline
\newcommand{\csinline}[1]{\lstinline[style = cs, basicstyle = \ttfamily\normalsize]{#1}}

% Workaround on global float for lstlisting
% \lstset{float}
% \makeatletter
% \let\lst@floatdefault\lst@float
% \makeatother

\colorlet{punct}{red!60!black}
\definecolor{background}{HTML}{EEEEEE}
\definecolor{delim}{RGB}{20,105,176}
\colorlet{numb}{magenta!60!black}

% Import the Json language
\lstdefinelanguage{Json}{
    numbers=left,
    stepnumber=1,
    showstringspaces=false,
    breaklines=true,
    literate=
     *{0}{{{\color{numb}0}}}{1}
      {1}{{{\color{numb}1}}}{1}
      {2}{{{\color{numb}2}}}{1}
      {3}{{{\color{numb}3}}}{1}
      {4}{{{\color{numb}4}}}{1}
      {5}{{{\color{numb}5}}}{1}
      {6}{{{\color{numb}6}}}{1}
      {7}{{{\color{numb}7}}}{1}
      {8}{{{\color{numb}8}}}{1}
      {9}{{{\color{numb}9}}}{1}
      {:}{{{\color{punct}{:}}}}{1}
      {,}{{{\color{punct}{,}}}}{1}
      {\{}{{{\color{delim}{\{}}}}{1}
      {\}}{{{\color{delim}{\}}}}}{1}
      {[}{{{\color{delim}{[}}}}{1}
      {]}{{{\color{delim}{]}}}}{1},
}

% Custom lststyle named json
\lstdefinestyle{json}
{
    breaklines = true,
    language = Json,
    columns = fullflexible,
    stringstyle=\color{eclipse_blue},
    morekeywords=[1]{class, return}, 
    keywordstyle=[1]\color{eclipse_red}
}

\lstloadlanguages{PHP}

\lstdefinestyle{PHP}
{
    breaklines = true,
    columns = fullflexible,
    language = PHP,
    stringstyle=\color{eclipse_red},
    showstringspaces=false,
    morekeywords=[1]{function, return}, 
    keywordstyle=\color{eclipse_blue}
 }


% Style for formatting text (chapter style etc.)
%!TEX root = ../../super_main.tex

% Titleformat for chapter (x | Chapter)
\titleformat
{\chapter}                                          % Command
[hang]                                              % Shape
{\Huge\bfseries}                                    % Format
{													% Label
	\thechapter
	\hspace{10pt}
	\textcolor{primarycolor}{|}
	\hspace{10pt}
} 
{0pt}                                               % Seperator
{\Huge\bfseries}                                    % Before
[]                                                  % After

% Format for chapter without numbers
\titlespacing*{\chapter}{0pt}{0pt}{18pt}

% Alignment and width for tabulars
\usepackage{array}
\newcolumntype{L}[1]{>{\raggedright\let\newline\\\arraybackslash\hspace{0pt}}m{#1}}
\newcolumntype{C}[1]{>{\centering\let\newline\\\arraybackslash\hspace{0pt}}m{#1}}
\newcolumntype{R}[1]{>{\raggedleft\let\newline\\\arraybackslash\hspace{0pt}}m{#1}}

% No indentation
\setlength{\parindent}{0pt}

% Style for the bibliografi
%!TEX root = ../../super_main.tex

% Packages used
\usepackage{csquotes}
\usepackage{bookmark}
\usepackage[backend = bibtex, bibencoding=ascii, citestyle = authoryear, bibstyle = authoryear, maxcitenames = 2, maxbibnames = 99, url = true]{biblatex}

\usepackage{xpatch}
\xpatchbibmacro{cite}{\usebibmacro{cite:label}}{\printnames{labelname}}{}{}
\renewbibmacro*{cite}{%
  \iffieldundef{shorthand}
    {\ifthenelse{\ifnameundef{labelname}\OR\iffieldundef{labelyear}}
       {\ifnameundef{labelname}
          {\usebibmacro{cite:label}}
          {\printnames{labelname}}%
          \setunit{\addspace}}
       {\printnames{labelname}%
        \setunit{\nameyeardelim}}%
     \usebibmacro{cite:labelyear+extrayear}}
    {\usebibmacro{cite:shorthand}}}

% ================================================ %

%add comma after authors in the citations
\renewcommand*{\nameyeardelim}{\addcomma\space}

% Uses square brackets instead of parentheses
\AtEveryCite{%
  \let\parentext=\parentexttrack%
  \let\bibopenparen=\bibopenbracket%
  \let\bibcloseparen=\bibclosebracket}

% Add references location
\addbibresource{references/references.bib}
   
% Add space between entries with different author's entries.
\setlength\bibnamesep{1.5\itemsep}

% ========================================================= %
% / / / / / / / / / / / / / / / / / / / / / / / / / / / / / %
% ========================================================= %



% ========================================================= %
%    _____                                          _       %
%   / ____|                                        | |      %
%  | |     ___  _ __ ___  _ __ ___   __ _ _ __   __| |___   %
%  | |    / _ \| '_ ` _ \| '_ ` _ \ / _` | '_ \ / _` / __|  %
%  | |___| (_) | | | | | | | | | | | (_| | | | | (_| \__ \  %
%   \_____\___/|_| |_| |_|_| |_| |_|\__,_|_| |_|\__,_|___/  %
%															%
% ========================================================= %

% Marks commonly used. For instance checkmark
%!TEX root = ../../super_main.tex

% Packages used
\usepackage{amssymb}
\usepackage{pifont}

% ================================================ %

% Checkmark  
\renewcommand{\checkmark}{\ding{51}}
\renewcommand{\cmark}{\ding{51}} 	
\newcommand{\bcheckmark}{\ding{52}}		 

% Cross
\newcommand{\cross}{\ding{53}}			
\newcommand{\bcros}{\ding{54}}			

% Crossmark
\newcommand{\crossmark}{\ding{55}}		
\newcommand{\bcrosmark}{\ding{56}}		

% CD paths with icon.
\newcommand{\cdpath}[1]{%
  % Param1: path
  \hyperref[app:cd]{\raisebox{-0.28ex}{\includegraphics[height=0.85em]{cd}}\mono{/#1}}%
}

% Name variable commands.
%!TEX root = ../../super_main.tex

% GIRAF
\newcommand{\giraf}[0]{\emph{GIRAF}\xspace}

% Launcher
\newcommand{\launcher}[0]{\emph{Launcher}\xspace}

% Category tool
\newcommand{\ct}[0]{\emph{Category tool}\xspace}

% C emphed
\renewcommand{\c}[0]{\emph{C}\xspace}

% C# emphed
\newcommand{\csharp}[0]{\emph{C\#}\xspace}

% Pictosearch
\newcommand{\ps}[0]{\emph{Pictosearch}\xspace}

% GIRAF Components
\newcommand{\gc}[0]{\giraf \emph{Components}\xspace}

% How different things should beformatted. For instance bold lexems etc.
%!TEX root = ../../super_main.tex

% Format specification for all lexems 
\newcommand{\lexical}[1]{\texttt{\textbf{#1}}\xspace}

% Define how the justify-alignment should be
\newcommand*\justify{
	\fontdimen2\font = 0.4em	% interword space
	\fontdimen3\font = 0.2em	% interword stretch
	\fontdimen4\font = 0.1em	% interword shrink
	\fontdimen7\font = 0.1em	% extra space
	\hyphenchar\font = `\-		% allowing hyphenation
}

% Used for code
\newcommand{\mono}{\texttt}

% Inline TAInC Language style
\def\taincinline{\lstinline[style = taincinline]}

% Inline Java Language style
\def\javainline{\lstinline[style = javainline]}

% Hide entries in table of content (use \tocless)
\newcommand{\nocontentsline}[3]{}
\newcommand{\tocless}[2]{\bgroup\let\addcontentsline=\nocontentsline#1{#2}\egroup}

% Define the format of pages. For instance page numbering
%!TEX root = ../super_main.tex

% Makes normal page numbering
\newcommand{\normalpagenumbering}
{
  \label{lastRoman}
  \cleardoublepage
  \pagenumbering{arabic}
  \setcounter{page}{1}
  \afterpage{\fancyfoot[LE]{\thepage{}}}
  \afterpage{\fancyfoot[RO]{\thepage{}}}
}

% Make a new even side (So the next content appears on the right side).
\newcommand{\newevenside}{
  \ifthenelse{\isodd{\thepage}}{\newpage}{
    \newpage
    \phantom{placeholder} % Some phantom text. Required.
    \thispagestyle{empty} % Do not display header/footer text
    \newpage
  }
}


% Different commands to insert figures etc.
%!TEX root = ../../super_main.tex

% \translated
% Will indicate that a given string is translated from danish
% arg1    the original string
% arg2    the string that has been translated
\newcommand{\translatedwithsource}[2] {
  \emph{``#1'' (translated: ``#2'')}\xspace
}

% \translated
% Will indicate that a given string is translated from danish
% arg2    the string that has been translated
\newcommand{\translated}[1] {
  \emph{``#1'' (translated)}\xspace
}

% \centerfig
% Will insert a figure with 75% textwidth in the horizontal center of the page.
% arg1 		graphics file to include.
% arg2 		the caption of the figure.
% arg3		the label of the figure.
\newcommand{\centerfig}[3] {
  \begin{figure}[htbp]
    \centering
    \includegraphics[width=0.75\textwidth]{#1}
    \caption{#2}
    \label{#3}
  \end{figure}
  \noindent
}

% \centerfigwithwidth
% Will insert a figure with a custom width in the horizontal center of the page.
% arg1 		graphics file to include.
% arg2		the caption of the figure.
% arg3		the label of the figure.
% arg4 		the width of the figure. For instance 0.5 for 50% of the page.
\newcommand{\centerfigwithwidth}[4] {
  \begin{figure}[!htbp]
    \centering
    \includegraphics[width=#4]{#1}
    \caption{#2}
    \label{#3}
  \end{figure}
  \noindent
}

% \wrapfig
% Will insert a figure wrapping with the text of the page.
% arg1		the graphics file to include.
% arg2 		the caption of the figure.
% arg3 		the label of the figure.
% arg4		what side the figure should wrap to. l for left, r for right.
\newcommand{\wrapfig}[4] {
  \begin{wrapfigure}{#4}{0.5\textwidth}
    \begin{center}
      \includegraphics[width=0.5\textwidth]{#1}
    \end{center}
    \caption{#2}
    \label{#3}
  \end{wrapfigure}
  \noindent
}

% \wrapfigwithwidth
% Will insert a figure wrapping with the text of the page.
% arg1		the graphics file to include.
% arg2 		the caption of the figure.
% arg3 		the label of the figure.
% arg4		what side the figure should wrap to. l for left, r for right.
% arg5 		the width of the figure. For instance 0.5 for 50% of the page. 
\newcommand{\wrapfigwithwidth}[5] {
  \begin{wrapfigure}{#4}{0.5\textwidth}
    \begin{center}
      \includegraphics[width=#5]{#1}
    \end{center}
    \caption{#2}
    \label{#3}
  \end{wrapfigure}
  \noindent
}

% Use the newfloat package (For labeling and adding  grammars)
\usepackage{newfloat}

% declare the floating environment {Grammar}
% this will also define \listofGrammars:
\DeclareFloatingEnvironment[
  % the file extension for the file used to create the list:
  fileext   = logr,% don't use log here!
  % the heading for the list:
  listname  = {List of Grammars},
  % the name used in captions:
  name      = Grammar,
  % the default floating parameters if the environment is used
  % without optional argument:
  placement = htbp
]{Grammar}

% Set the grammar indent to 80 pt. The indent indicates the space between LHS and RHS.
\grammarindent = 80pt

% Define how non-terminals should look
\renewcommand{\syntleft}{$<$\ttfamily}
\renewcommand{\syntright}{$>$}

% \cfg
% used for declaring cfgs
% arg1    the graphics file to include.
% arg2    the caption of the figure.
% arg3    the label of the figure.
\newcommand{\cfg}[3] {
  \begin{Grammar}
    \ttfamily
    \begin{grammar}
      #1
    \end{grammar}

    \caption{#2}
    \label{#3}
  \end{Grammar}
}



% declare the floating environment {Equation}
% this will also define \listofEquations:
\DeclareFloatingEnvironment[
  % the file extension for the file used to create the list:
  fileext   = logr,% don't use log here!
  % the heading for the list:
  listname  = {List of Equations},
  % the name used in captions:
  name      = Equation,
  % the default floating parameters if the environment is used
  % without optional argument:
  placement = htbp
]{Equation}

% Commands to make references in the text. 
%!TEX root = ../../super_main.tex

% ===================== %
% == Page references == %
% ===================== %

% \onpageref
% References a page using the "Page"-prefix.
% arg1		the label to reference to.
\newcommand{\onpageref}[1]{Page \pageref{#1}\xspace}



% ======================= %
% == Figure references == %
% ======================= %

% \figref
% References a figure using the "Figure"-prefix.
% arg1 		the labelname of the figure to reference. 
\newcommand{\figref}[1]{Figure \ref{#1}\xspace}

% \figrefpage
% References a figure using the "Figure"-prefix. Will also display the page of the figure.
% arg1		the labelname of the figure to reference.
\newcommand{\figrefpage}[1]{\figref{#1} on \onpageref{#1}}



% ====================== %
% == Table references == %
% ====================== %

% \tabref
% References a table using the "Table"-prefix.
% arg1		the labelname of the tabel to reference.
\newcommand{\tabref}[1]{Table \ref{#1}\xspace}

% \tabrefpage
% References a table using the "Table"-prefix. Will also display the page of the table.
% arg1		the labelname of the tabel to reference.
\newcommand{\tabrefpage}[1]{\tabref{#1} on \onpageref{#1}}



% ======================== %
% == Chapter references == %
% ======================== %

% \charef
% References a chapter using the "Chapter"-prefix.
% arg1 		the labelname of the chapter to reference. 
\newcommand{\charef}[1]{Chapter \ref{#1}\xspace}

% \secrefpage
% References a chapter using the "Chapter"-prefix. Will also display the page of the chapter.
% arg1		the labelname of the chapter to reference.
\newcommand{\charefpage}[1]{\charef{#1} on \onpageref{#1}}



% ========================= %
% == Appendix references == %
% ========================= %

% \appref
% References a appendix using the "Appendix"-prefix.
% arg1 		the labelname of the appendix to reference. 
\newcommand{\appref}[1]{Appendix \ref{#1}\xspace}

% \secrefpage
% References a appendix using the "Appendix"-prefix. Will also display the page of the appendix.
% arg1		the labelname of the appendix to reference.
\newcommand{\apprefpage}[1]{\appref{#1} on \onpageref{#1}}



% ======================== %
% === Part references ==== %
% ======================== %

% \secref
% References a section using the "Section"-prefix.
% arg1 		the labelname of the section to reference. 
\newcommand{\prtref}[1]{Part \ref{#1}\xspace}

% \secrefpage
% References a section using the "Section"-prefix. Will also display the page of the section.
% arg1		the labelname of the section to reference.
\newcommand{\prtrefpage}[1]{\prtref{#1} on \onpageref{#1}}



% ======================== %
% == Section references == %
% ======================== %

% \secref
% References a section using the "Section"-prefix.
% arg1 		the labelname of the section to reference. 
\newcommand{\secref}[1]{Section \ref{#1}\xspace}

% \secrefpage
% References a section using the "Section"-prefix. Will also display the page of the section.
% arg1		the labelname of the section to reference.
\newcommand{\secrefpage}[1]{\secref{#1} on \onpageref{#1}}



% =========================== %
% == Definition references == %
% =========================== %

% \defref
% References a definition using the "Definition"-prefix.
% arg1 		the labelname of the definition to reference. 
\newcommand{\defref}[1]{Definition \ref{#1}\xspace}

% \defrefpage
% References a definition using the "Definition"-prefix. Will also display the page of the definition.
% arg1		the labelname of the definition to reference.
\newcommand{\defrefpage}[1]{\defref{#1} on \onpageref{#1}}



% =========================== %
% == Lstlisting references == %
% =========================== %

% \lstref
% References a lstlisting using the "Code snippet"-prefix.
% arg1 		the labelname of the lstlisting to reference. 
\newcommand{\lstref}[1]{Code Snippet \ref{#1}\xspace}

% \lstrefpage
% References a lstlisting using the "Code snippet"-prefix. Will also display the page of the lstlisting.
% arg1		the labelname of the lstlisting to reference.
\newcommand{\lstrefpage}[1]{\lstref{#1} on \onpageref{#1}}

% \lineref
% References a specific line. Using the "Line x".
% arg1		the labelname of the line in the listing to reference.
\renewcommand{\lineref}[1]{Line \ref{#1}\xspace}

% \lineref
% References a specific line. Using the "Line x".
% arg1		the labelname of the starting line in the listing to reference.
% arg2		the labelname of the starting line to reference.
\newcommand{\linesref}[2]{Lines \ref{#1}-\ref{#2}\xspace}


% \lstrefline
% References a specific line on a specific code snippet. Using the "Line x in Code snippet y".
% arg1		the labelname of the lstlisting to reference.
% arg2		the labelname of the line to reference.
\newcommand{\lstrefline}[2]{Line \ref{#2} in \lstref{#1}}

% lstreflines
% references specific lines on a specific code snippet.
% arg1		the labelname of the lstlisting to reference.
% arg2 		the labelname of the starting line to reference.
% arg3		the labelname of the ending line to reference.
\newcommand{\lstreflines}[3]{Lines \ref{#2}-\ref{#3} in \lstref{#1}}



% ======================== %
% == Grammar references == %
% ======================== %

% \graref
% References a grammar using the "Grammar"-prefix.
% arg1 		the labelname of the grammar to reference. 
\newcommand{\graref}[1]{Grammar \ref{#1}\xspace}

% \grarefpage
% References a grammar using the "Grammar"-prefix. Will also display the page of the grammar.
% arg1		the labelname of the lstlisting to reference.
\newcommand{\grarefpage}[1]{\graref{#1} on \onpageref{#1}}


% ========================= %
% == Equation references == %
% ========================= %

% \graref
% References a equation using the "Equation"-prefix.
% arg1 		the labelname of the equation to reference. 
\newcommand{\equref}[1]{Equation \ref{#1}\xspace}

% \grarefpage
% References a equation using the "Equation"-prefix. Will also display the page of the equation.
% arg1		the labelname of the lstlisting to reference.
\newcommand{\equrefpage}[1]{\equref{#1} on \onpageref{#1}}

% \graref
% References a equation using the "Equation"-prefix.
% arg1      the labelname of the equation to reference. 
\newcommand{\forref}[1]{Formula \ref{#1}\xspace}

% =========================== %
% == Algorithm references == %
% =========================== %

% \algref
% References a algorithm using the "Algorithm"-prefix.
% arg1      the labelname of the algorithm to reference. 
\renewcommand{\algref}[1]{Algorithm \ref{#1}\xspace}

% \algrefpage
% References a algorithm using the "Algorithm"-prefix. Will also display the page of the algorithm.
% arg1      the labelname of the algorithm to reference.
\newcommand{\algrefpage}[1]{\algref{#1} on \onpageref{#1}}

% Commands to make type rules
%!TEX root = ../../super_main.tex

% Will print a type rule
% arg1:		The name of the typerule
% arg2:		The label used
% arg3:		The premise of the type rule
% arg4:		The conclusion of the type rule
% arg5:		The side condition of the type rule
\newcommand{\typerule}[5] {
	\parbox{0.1\textwidth}{[#1]} \parbox{0.89\textwidth} {
		\begin{equation*}
		\frac{#3}{#4} ~ #5
		\label{#2}
		\end{equation*}
	}
}

\makeatletter
\newcommand{\mathleft}{\@fleqntrue\@mathmargin\parindent}
\newcommand{\mathcenter}{\@fleqnfalse}
\makeatother

% Will print a semantic rule
% arg1:		The name of the typerule
% arg2:		The content of the rule
\newcommand{\semanticrule}[2] {
	\parbox{0.20\textwidth}{[#1]} \parbox{0.79\textwidth} {
		\mathleft
		#2
	}
}

% Will make the semantic rules left-aligned
%\setlength{\mathindent}{0pt}

% Commands to format tables
%!TEX root = ../../super_main.tex

% \cm{}
% Allows centering while defining a width on a table column
% arg1 		the width of the cell
\newcolumntype{x}[1]{>{\centering\arraybackslash\hspace{0pt}}p{#1}}

\usepackage{adjustbox}
\usepackage{array}
\usepackage{booktabs}
\usepackage{multirow}

\newcolumntype{K}[2]{%
    >{\adjustbox{angle=#1,lap=\width-(#2)}\bgroup}%
    c%
    <{\egroup}%
}

\newcommand*\rot{\multicolumn{1}{K{90}{1em}|}}% no optional argument here, please!


\usepackage{diagbox}

\usepackage{longtable}

\usepackage{tabularx}

% Commands to format quotes
%!TEX root = ../../super_main.tex

\newcommand{\quotewithref}[2] {
    \begin{quote}
        ``#1''
    \end{quote}
    \vspace{-1em}
    \hspace{25pt} \emph{\textemdash \textcite{#2}}
    \\
}

% Pretty pie charts
\usepackage{tikz}
\usepackage{pgf-pie}

% This needs to be the last thing in the preamble. It is used to make pretty Matrixes
\usepackage{blkarray}
\usepackage{pgfplots}
\usetikzlibrary{backgrounds}
\pgfplotsset{compat=1.7}
\usepackage{setspace}


% Some magic that makes it possible to make symbols bold (using \boldmath or \boldsymbol)
\SetSymbolFont{stmry}{bold}{U}{stmry}{m}{n}

