%!TEX root = ../../super_main.tex
% Skal svare til Ivans: 2.6. Results from representation types and quadrants
% Her skal der tages en masse beslutninger baseret med vision scenarios etc. 
\section{Vision}
\label{sec:vision}
The vision for a solution for our problem is to develop a platform that allows for collection of data for reality mining. Different people who apply reality mining procedures often require different types of labeled training data. We would like to provide a platform that can gather snapshots, which will allow for human activity recognition. The campaigns should be configurable such that the gathered data can be useful for different purposes, for instance accumulating statistics or training a machine intelligence model. For this reason the platform should allow for configurable collection of snapshots. A customer should be able to to configure what type of information he wants in the snapshots, and how the labels of these should derived, as well as define the demographic group that the campaigns should be available to. The vision for the platform is, that people, who are in need of context aware training data, can use our platform, instead of developing and distributing their own specialized applications to gather these type of data. The way we intend to concretize this, is to establish two different interfaces. One interface for customers, where they can specify campaigns, and another interface for participants that allows for gathering of data for reality mining. An illustration of this can be seen in \figref{fig:system_vision}. Here, customers specify the data the want, through the \emph{uMiner} system. This campaign specification is made available to all mobile devices, such that participants can join the campaign, and their smart devices will know how to ``mine'' the reality around the participants. 

\begin{figure}[!htbp]
    \centering
    \includegraphics[width=0.8\textwidth]{problem_analysis/vision/system_vision}
    \caption{The system vision.}
    \label{fig:system_vision}
\end{figure}
\FloatBarrier

We imagine that our platform will handle all technical details related to the capturing of snapshots, and that customers should instead focus their effort on motivating participants to join their campaigns. 

\section{Delimitations}
\label{sec:delimitations}
From the vision scenarios, see \secref{sec:vision_scenarios}, we learned that the project could include profiles of participants, which could allow customers to target specific groups of people. We are certain the ability to target demographics would be valuable to customers, but we have however found this would significantly impact the implementability of our solution. We acknowledge the need for customers to have a more select group of people collect data. We have therefore; besides allowing participants to opt-in to campaigns, which we will refer to as public campaigns; decided to introduce a concept of private campaigns. A private campaign should work similarly to a public campaign, except that the campaign will not be listed publicly in the data gathering application. Access to a private campaign should be managed by customers, who would have to distribute an identifier for the campaign on their own. We feel confident that such a simple solution could provide some of the same value as including profiles of participants, while being easier to implement for first viable version of a solution.

% We could, besides sensors from mobile phones and wearables, also consider including external sensors such as cameras or smart home sensors like movement detectors or temperature sensors in the room. It is however demanding to support many different external sensors in an application, installation of compatible external sensors where the participants reside could entail high installation and maintenance cost and would require a different level of commitment from participants. There has been attempts to alleviate the burden of including many different external sensors in mobile applications by making it easier to interface with them and write drivers but the problem is still considered difficult and costly \parencite{open_data_kit}. We have chosen, at least initially, to not include external sensing as it is deemed too extensive to support collection of such sensors compared to the one of wearables and mobile devices.
