%!TEX root = ../../super_main.tex
\chapter{Introduction}
\label{cha:introduction}
In recent years mobile technologies have had an major growth in customers. The sales of Smartphones have increased in sales by a factor of 11 since 2007 \parencite{statsia_smartphones}. Other mobile technologies, like wearables, have also seen an increase in demand in the last couple of years. FitBit, a company selling smart wristbands, have seen a exponential increase in sales of their device in the past five years \parencite{statsia_fitbit} \todo{Describe how this increase affects the industry and the development of mobile platforms}.
\\\\
Mobile computing \todo{reference something about mobile computing?} differs a lot from the traditional understanding of computing. Mobile computing has the advantage that it can be more context aware, meaning that mobile devices are able to react and adapt accordingly to the context that they are used in. In traditional computing, using a desktop application, the context of which the application is used does not change as rapidly as mobile applications. This also means that mobile technologies in general are more sensitive to the environment in which they operate. Given that mobile technologies are highly dependent on factors such as limited computing power, battery life, and connectivity, it is important to take these into account when developing applications.
\\\\
The expansion in mobile technologies opens new opportunities for studying and understanding the context around the user more accurately. This means that we can obtain better knowledge regarding the users of a system, and understanding people in general. A common way of getting this type of information right now is by the use of surveys, questionnaires, and interviews. This type of information retrieval can also be used to understand users in their context. However, this type of information retrieval is rather explicit, as opposed to mobile technologies that can measure the environment indirectly using different built in sensors. The issue with this indirect understanding of the context is that users are more complex compared to the discrete values outputted from sensors\todo{Insert source}. One could imagine that a combination of surveys and sensor data could improve the understanding of people and their context. This project is centered around binding discrete sensor data to some context or classes. This report describes the development of a platform that can produce labeled training data where the classification is done through the help of users. \todo{Er det ok med de sidste 2 sætninger her i introduktionen?}

\todo[inline]{Should we talk more about morten and BJ Fogg behavior model?}
