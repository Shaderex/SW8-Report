%!TEX root = ../../super_main.tex
\chapter{Introduction}
\label{cha:introduction}

In recent years, computers have changed from being stationary work related tools to being personal devices we carry around with us at all times, in form of smartphones and wearables. These devices are becoming increasingly ubiquitous, and the number of consumers that have access to it increases with the decrease in hardware cost. With the increase of sensor variety and increase in connectivity and battery life of these devices, a lot of people carry a great amount untapped potential information around with them in these ubiquitous devices. \todo{Maybe ref some sales statistics?} An area of research that studies this type of untapped potential is reality mining \todo{insert ref}.
\\\\
\todo[inline, color=green]{Rikke: Må vi introducere reality mining her? eller bliver det for teknisk}
Reality mining is a computer science area of research that is concerned with gathering information from the real world using electronic sensors, and tries to derive some meaning of it using various techniques, such as pattern recognition of this type of data. This type of research is done to get a better understanding of human dynamics. The idea is that by monitoring sensor output from ubiquitous devices, one could get a better understanding of the context where the device is being used, and how this context correlates to the complexity of us as humans. However to enable studies in the reality mining area, one need a system that allows interested parties to gain these type of sensor data. Today it is common \todo{Vi skal nok finde en reference der siger dette?}, if such a party wants to get some data collected, to firstly create a system that can monitor subjects devices and upstream or store sensor data somewhere. Secondly they must also find participants for the surveys they are doing, before doing the actual study of the data. For instance if some medical researchers wants to figure out how different types of medication affects patients movement, mood and so on, they must construct their own application for the ubiquitous devices. Find suitable subjects that can participate in their survey, and then finally perform the analysis they were interested in the first place. Likewise, parties interested in machine intelligence might want to determine if it is possible to predict some human dynamics using the ubiquitous devices. They must also go through the tedious phase of constructing a system and finding participants in order to gain access to the data they want to analyze base a model on in the first place.
\\\\
This project focuses on creating a platform that allows for interested parties to gain access to information that can be used for reality mining. The focus will be to create a modular and scalable platform that can handle the variety of ubiquitous devices and their different types of sensors, while allowing interested parties to get the information they desire.

%Mobile computing \todo{reference something about mobile computing?} differs a lot from the traditional understanding of computing. Mobile computing has the advantage that it can be more context aware, meaning that mobile devices are able to react and adapt accordingly to the context that they are used in. In traditional computing, using a desktop application, the context of which the application is used does not change as rapidly as mobile applications. This also means that mobile technologies in general are more sensitive to the environment in which they operate. Given that mobile technologies are highly dependent on factors such as limited computing power, battery life, and connectivity, it is important to take these into account when developing applications.

%This project is centered around binding discrete sensor data to some context or classes. This report describes the development of a platform that can produce labeled training data where the classification is done through the help of participants. \todo{Er det ok med de sidste 2 sætninger her i introduktionen?}

%The goal of the project is to give developers, researchers, etc. a new type of platform where they are able to gather labeled training data from the mobile technologies that a lot of people use. One could imagine that this platform sets the stage for problems to be solved using machine intelligence and statistical analysis. A computer-scientific idea that can be guided by this type of platform is indoor positioning. One could imagine that developers are able to prompt participants for additional information in regards to a building that have not yet been mapped accurately enough. After the prompt they can use the wireless radio signals to create a model that can triangulate their position indoor. \todo{Reference Bent?}


%\todo[inline]{Should we talk more about morten and BJ Fogg behavior model?}
