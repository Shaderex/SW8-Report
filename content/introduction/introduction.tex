%!TEX root = ../../super_main.tex
\chapter{Introduction}
\label{cha:introduction}

In recent years, computers have changed from being stationary work related tools to being personal devices we carry around with us at all times, in form of smartphones and wearables. These devices are becoming increasingly ubiquitous \parencite{statsia_fitbit}\parencite{android_os_market_share}, and the number of consumers that have access to them increases with the decrease in hardware cost. With the increase of sensor variety and increase in connectivity and battery life of these devices, many people carry a great amount of untapped potential information around with them in these ubiquitous devices. An area of research that studies this type of untapped potential is reality mining \parencite{eagle2006_reality_mining_definition}.
\\\\
Reality mining is a branch of data mining, and data mining is about finding patterns and deriving knowledge from some given dataset, i.e. mining the data. A use case of data mining could be finding a pattern in the shopping habits of consumers in a supermarket. One could for instance use data mining techniques to find a pattern which reveals, that every time a consumer buys bread, there is high probability that he will also buy butter.
The super market can then use this information in different ways. They can either place bread and butter closely together, to please the consumer and possible increase their sale for the people who would not usually buy them together, or place them apart so that consumers might be tempted to buy some additional items when searching for the items they need. In a similar way, reality mining is about finding patterns in machine-sensed environmental data, in order to get a better understanding of human dynamics. The reality mining branch of data mining is concerned with data gathered from electronic sensors and finding patterns in the data from these sensors. The idea is, that by monitoring some context through sensor output and by possibly including human opinions about what is happening, possibly through surveys, one can get a better understanding of how human behavior and actions correlate to data that represents a subset of reality. However, to enable studies in the reality mining area, one needs a system that allows interested parties to gain this type of sensor data and opinions. 
\\\\
%One could create a mobile application which can collect sensor data and upstream it in order to store it somewhere centrally for later use. 
It is problematic that, if someone wants to get some data collected, they first have to find an existing solution that fits their exact needs, or create a system that can monitor the data they need. Another problematic aspect of this, is the difficulty of acquiring participants for their data collection. It is necessary to ensure a solid foundation of participants, which might be hard to acquire for people who create their own new systems, before any data will be available for analysis. For instance if some medical researchers want to figure out how different types of medication affect patients' movement or mood. One can imagine that they would have to construct their own solution for collecting data from the ubiquitous devices. This would be followed by finding suitable subjects that can participate in their survey, and then finally perform the analysis they were interested in the first place.
\\\\
This project focuses on creating a platform that allows interested parties to gain access to information, that can be used for reality mining, gathered through ubiquitous devices. The focus is to create a modular and scalable platform that can handle a variety of ubiquitous devices and their different types of sensors, while allowing interested parties to specify precisely what information they desire and then easily retrieve it.

%Mobile computing \todo{reference something about mobile computing?} differs a lot from the traditional understanding of computing. Mobile computing has the advantage that it can be more context aware, meaning that mobile devices are able to react and adapt accordingly to the context that they are used in. In traditional computing, using a desktop application, the context of which the application is used does not change as rapidly as mobile applications. This also means that mobile technologies in general are more sensitive to the environment in which they operate. Given that mobile technologies are highly dependent on factors such as limited computing power, battery life, and connectivity, it is important to take these into account when developing applications.

%This project is centered around binding discrete sensor data to some context or classes. This report describes the development of a platform that can produce labeled training data where the classification is done through the help of participants. \todo{Er det ok med de sidste 2 sætninger her i introduktionen?}

%The goal of the project is to give developers, researchers, etc. a new type of platform where they are able to gather labeled training data from the mobile technologies that a lot of people use. One could imagine that this platform sets the stage for problems to be solved using machine intelligence and statistical analysis. A computer-scientific idea that can be guided by this type of platform is indoor positioning. One could imagine that developers are able to prompt participants for additional information in regards to a building that have not yet been mapped accurately enough. After the prompt they can use the wireless radio signals to create a model that can triangulate their position indoor. \todo{Reference Bent?}


%\todo[inline]{Should we talk more about morten and BJ Fogg behavior model?}
