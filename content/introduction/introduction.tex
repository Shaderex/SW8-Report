%!TEX root = ../../super_main.tex
\chapter{Introduction}
\label{cha:introduction}
In recent years mobile technologies have had an major growth in customers. The sales of Smartphones have increased in sales by a factor of 11 since 2007 \parencite{statsia_smartphones}. Other mobile technologies, like wearables, have also seen an increase in demand in the last couple of years. FitBit, a company selling smart wristbands, have seen a exponential increase in sales of their device in the past five years \parencite{statsia_fitbit} \todo{Describe how this increase affects the industry and the development of mobile platforms}.
\\\\
Mobile computing \todo{reference something about mobile computing?} differs a lot from the traditional understanding of computing. Mobile computing has the advantage that it can be more context aware, meaning that mobile devices are able to react and adapt accordingly to the context that they are used in. In traditional computing, using a desktop application, the context of which the application is used does not change as rapidly as mobile applications. This also means that mobile technologies in general are more sensitive to the environment in which they operate. Given that mobile technologies are highly dependent on factors such as limited computing power, battery life, and connectivity, it is important to take these into account when developing applications.
\\\\
This project is centered around binding discrete sensor data to some context or classes. This report describes the development of a platform that can produce labeled training data where the classification is done through the help of users. \todo{Er det ok med de sidste 2 sætninger her i introduktionen?}

The goal of the project is to give developers, researchers, etc. a new type of platform where they are able to gather labeled training data from the mobile technologies that a lot of people use. One could imagine that this platform sets the stage for problems to be solved using machine intelligence and statistical analysis. A computer-scientific idea that can be guided by this type of platform is indoor positioning. One could imagine that developers are able to prompt users for additional information in regards to a building that have not yet been mapped accurately enough. After the prompt they can use the wireless radio signals to create a model that can triangulate their position indoor. \todo{Reference Bent?}
\\\\
Another field where this platform could be relevant is the Telehealth field of knowledge \parencite{telehealth_aau}. This area uses telecommunication to improve the general health of citizens. This platform could potentially enable a better understanding of the human psyche in a more implicit matter, and thereby assist in improving the Telehealth field of knowledge. A vision could be to have an application that could replace medicine used in psychiatric areas. Instead of prescribing antidepressants, a patient will get an application installed on their smart devices that will give them tools to handle their mental state in different contexts. The application could ask questions that counters negative thought patterns. Such an application could also be useful for doctors and psychiatrist, since they would have access to continuous monitoring of their patients. The patients would still need consultations, but the doctors do not have to rely solely on the patients memory. In the same way one could imagine applications for people struggling with drugs or alcohol, where a Disulfiram\footnote{For information about Disulfiram see \textcite{nlm_disulfiram}} treatment could be replaced or decreased by introducing an application that understands the mental state of alcoholics.

\todo[inline]{Should we talk more about morten and BJ Fogg behavior model?}
