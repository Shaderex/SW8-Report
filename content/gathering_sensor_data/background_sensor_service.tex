%!TEX root = ../../super_main.tex

\section{Background Sensor Service}
\label{sec:background_sensor_service}
We have implemented a service, which we call \mono{BackgroundSensorService}, in order to facility non-intrusive data collection in the Android system.
This section includes some technical details about how services work in Android and how we have implemented our \mono{BackgroundSensorService}. 

\subsection{What is an Android service?}
A services in Android is an application component which encapsulates long running background processing or a way to provide access to a shared ressource. Services can be shared and can be configured run independently, i.e. running in a different process, from the graphical interface shown to users, which is ideal for our background data collection. Services have their own lifecycles and can be configured in many different ways. They can be configured as an API where the service itself has a short lifespan while serving some data or while performing a short task or they can be configured as a long running background task with its own complex behavior.
\\\\
% Message parsing
The Android framework supports two-way message parsing as a way of communication with a service when it runs in a different process. Message parsing makes it easy for the Android \mono{Activity} elements in the graphical and interactive part of the application to communicate with our \mono{BackgroundSensorService}.

% Boot receiver
% On Application start
\subsection{Service Start}
Our \mono{BackgroundSensorService} should ideally always be running, which does not imply that it is constantly using processing ressources. We have implemented two measures to ensures that the service is running for as much time as possible. We have implemented an Android BroadcastReceiver, which upon receiving a system wide boot completed action (\mono{ACTION\_BOOT\_COMPLETED}), starts the service. The service is also attempted started everytime the application is started, but is not started twice if the service is already running. 

\subsection{Snapshot Generation and Synchronization}
\label{sub:background_sensor_service_snapshot_generation_and_synchronization}

The background \mono{BackgroundSensorService} uses an independed timer for snapshot generation. The timer for snapshot generation is encapsulated in a class called \mono{SnapshotTimer} which is started whenever the service has an active running campaign which should generate snapshots. Timed tasks are then scheduled at a constant rate, generating snapshots, until the campaign is completed or stopped by the user.
\\\\
Before we send all available snapshots to the server we transform them to a JSON format, because JSON is humanly readable, decent for machine learning purposes, and because it has to have some well defined format when we send it, so it can be made sense of on the server. We use this because it is easy, instead of inventing our own format, but at the sacrifice of spending extra bandwidth of the participant to send it, which is rather bad. A better solution would be to use some format with few-to-no keys, where we have optimized how much space we use. 
\\\\
When we have converted the data to JSON, the actual synchronization of snapshots is done seperately and is encapsulated in a class called \mono{SynchronizationManager}. This synchronization should ideally be done using as little network ressources as possible, and thereby also save battery capacity. 
\\\\
An example of an Android API, or rather a library with Google APIs for Android, which could help optimize network usage, and the, would be the \mono{GCMNetworkManager}, which is an Android service, that can schedule encapsulated network tasks. \mono{GCMNetworkManager} \parencite{gcmnetworkmanager} handles batching of network tasks; retries; backoffs, in case a remote server is not responding or is busy; waiting for Wi-Fi, for bandwidth heavy communication; waiting with commutation until the device is charging; and waiting until a radio becomes active, for instance by getting activated by another application. It does all of this based on parameters given to every encapsulated network tasks such as a desired time frame for the network communication and desired network or battery conditions.  
\\\\
Our \mono{SynchronizationManager} uses the \mono{GCMNetworkManager} to schedule a periodic task which attempts to synchronize gathered snapshots with the server at a fixed rate. Our periodic task is configured to override and removes the old pending tasks if it reschedulded but has not yet been executed. It is then up to the \mono{GCMNetworkManager} to determine when the task should be executed based parameters given with the task. Our task is configured to require that an unmetered connection, e.g. WiFi, is available before the task can be executed. 
\\\\
The \mono{GCMNetworkManager} incorporates radio based optimizations and will start all the pending tasks if the radio is active and a minimum time, as specified by the parameters, has elapsed. This minimizes unecessary wake-ups of the radio and thus minimizes battery consumption. Using \mono{GCMNetworkManager} makes sense for network related operations which can be delayed in order to minimize battery consumption. Other network tasks which should happen instantly, such a refreshing the GUI with lists of campaigns from the server, are therefore not applicable to be optimized with \mono{GCMNetworkManager}. 

\subsection{Background Sensor Listening}
The \mono{BackgroundSensorService} is configured to have one thread, in a pool of threads, available per sensor type, i.e. per \mono{SensorProvider}, see \secref{sub:providing_sensor_data_implementation}, instance used in the service. This lets the scheduling of the gathering from the different \mono{SensorProvider} instances up to the underlying Java implementation of threads in a hope to run the different types of sensors and other data sources as independently as possible. Data is only collected from the \mono{SensorProvider} instances if the corresponding sensors are included in the currently active campaign and if the system is currently gathering a snapshot. There is no data collection activity in the \mono{BackgroundSensorService} unless a timed task for a snapshot is running. CPU resources, and thereby battery, are therefore only consumed on demand.
\\\\
We do not provide any real time guarantees but the different threads from the different streams of data should always be able to run unless they get starved by other processes in the system. Temporary starvation of the data collecting threads might result in missed updates from different data sources. 

\subsection{Concurrency Overview}
We have attempted to give an overview of the current processes and threads running in the mobile application in \figref{fig:system_currency_and_lifecycle} illustrated with a near UML Activity Diagram Syntax. 

\todo[inline]{Rikke: Figuren skal beskrives i teksten og en legend vil være meget godt}
\begin{figure}[!htbp]
    \centering
    \includegraphics[width=\textwidth]{graphic/backgroundsensorservice/lifecyclestuff.pdf}
    \caption{An Activity Diagram-esque overview of the mobile Application Components.}
    \label{fig:system_currency_and_lifecycle}
\end{figure}
\FloatBarrier