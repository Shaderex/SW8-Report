%!TEX root = ../../super_main.tex
\section{Data Quantity Estimation}
\label{sec:data_quantity_estimation}
%This gave us a dataset with a size of 142908 bytes. This experiment would then yield a data set of approximately 205 MB if it was run for a day. Running the same experiment for 30 days with 100 different devices would then approximately yield a data set of 617 GB assuming similar mobile devices with similar sensors. This quick napkin math was only for one sensor with 100 people and this could quickly escalate if more sensor or more people are added. 
Some sensor measurements are continuous. We have experimented with different continuous sensors on a Nexus 5 smartphone and logged all values captured from the sensor for 1, 5, and 20 minutes. The results of these tests can be seen in \tabref{tab:sensor_experiment}. Note that the orientation sensor is a virtual sensor, which uses data collected from both the gyroscope and magnetometer, hence the correlation between the data sizes ($143 \text{KB} + 35 \text{KB} \approx 177 \text{KB}$ for 1 minute). These tests were only performed for four different sensors, but customers might need data from several different sensors, thus further increasing the amount of data collected. One should also consider that this data is only collected for a single phone / person. Collecting the data from several phones / people would further escalate the total collected data size. These quantities of data might present a problem even on modern mobile platforms due to paid limited data plans and battery consumption. There might be different data needs, some customers might require very detailed data from many sensors from a few devices and others might require more sparse data from a few sensors from a lot of different devices. However, we can conclude that gathering a continuous stream of data from all sensors all the time is not a viable solution.

\begin{table}[!htbp]
\centering
\begin{tabular}{l|c|c|c|c}
\textbf{Sensor}     & \textbf{Accellerometer} & \textbf{Gyroscope} & \textbf{Magnetometer} & \textbf{Orientation} \\ \hline
\textbf{1 minute}   & 142 KB                  & 143 KB             & 35 KB                 & 177 KB               \\ \hline
\textbf{5 minutes}  & 714 KB                  & 714 KB             & 178 KB                & 892 KB               \\ \hline
\textbf{20 minutes} & 2859 KB                 & 2859 KB            & 715 KB                & 3573 KB                
\end{tabular}
\caption{Data size of sensor data collection after a set amount of time.}
\label{tab:sensor_experiment}
\end{table}
\FloatBarrier