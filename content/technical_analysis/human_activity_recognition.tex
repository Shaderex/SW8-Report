%!TEX root = ../../super_main.tex

\section{Human Activity Recognition}
\label{sec:human_activity_recognition}

\todo{Overvej at få merges buzz worded reality mining ind i det her afsnit}

We would like to gather training data, i.e. human activity recognition (HAR) including and combined with contextual data from mobile devices. This data could be used for different purposes, this could for instance be to accumulate statistics or to gather a training data set for building a machine intelligence model. By training data set we mean a set of different samples with two or more features each, where a subset of the features can be used to learn something about the remaining features which constitute a class or classes.
\\\\
We could, besides sensors from mobile phones and wearables, also include external sensors such as cameras or smart home sensors like movement detectors or temperature sensors in the room. It is however demanding to support many different external sensors in an application, installation of compatible external senors where the participants reside could entail high installation and maintenance cost and would require a different level of commitment from participants. There has been attempts to alleviate the burden of including many different external sensors in mobile applications by making it easier to interface with them and write drivers but the problem is still considered difficult and costly \parencite{open_data_kit}. Cameras are also associated with other potential problems such as privacy issues and feature extraction from the produced video material. Users are not likely to allow continuous filming during the time of interest of the customers who would like have the training data. We have therefore chosen, at least initially, to not include external sensing as we think it would become too extensive to implement data collection compared to the implementation of data collection from wearables and mobile devices.
\\\\
There are at least two overall different approaches to this namely participatory and opportunistic sensing \parencite{opp_or_par} \parencite{har_wearables}. Participatory sensing requires that users actively participates in the data collection by own initiative, directly entering data, or by otherwise reacting to requests for input. Opportunistic sensing requires less to none user involvement but it does require more passive monitoring of the context of the user even when the system is not actively gathering data in order to sense if the users context matches triggering conditions for a campaign. A mobile data gathering application would have to actively monitor or somehow be notified when users are physically located near an area of interest. An application query i.e. a campaign could for instance require that data should be gathered when the user is drinking coffee i.e. they are near a coffee shop. This constant monitoring of the users context, in order to enable triggers, might though present its own privacy intrusion problems. This monitoring could however be performed strictly locally on the users mobile device and wearables and the users privacy could thus be somewhat preserved.
\\\\
The two approaches can be combined where opportunistic sensing could then provide context in the shape of features used to learn classes provided from participatory sensing where users are included to provide the target classes, i.e. labels, in a training data set. Opportunistic sensing could also provide triggers for when users should be involved in the process. Allowing customers to specify a campaign, or in the words of \parencite{opp_or_par} an application query, could then allow customer of the system to specify to which degree they want user involvement, i.e. participatory sensing, and when and how they want it, i.e. opportunistic sensing.
\\\\
The amount of different classes or labels one could infer directly from the sensor data is somewhat limited and would not allow many possible classes which captures human behavior or state. One could attempt to setup artificial rules and determine the labels in the data based on these rules, these rules would however likely be based on assumptions about the data, unless some other ground truth is known, and it would be difficult to infer many possibly interesting human behavior or state classes for a desired artificial intelligence (AI) model such as for instance stress levels or happiness. One might however with a given algorithm or another machine intelligence model be able to recognize patterns in the data to infer classes, this could for instance be if the person is dancing or not, e.g. by detecting rhythmic movement.    
\\\\
One way to involve users could be questionnaires with one or more questions. The answers could then determine the labels of the data, or it could be seen as features. The validity of this user generated data would depend on the truthfulness and perception of the users which could be a source of error. The validity of the data could however be evaluated and filtered by customers outside the data collection system. Data from external sensors or other certain combinations of values from the collected set of features for each sample, i.e. the users context, might be helpful in determining the truthfulness of answers provided by users.
\\\\
We will need to support some kind of user involvement in order to be able to gather labels for the users context. The success of the system thus depends on how willing users are to participate, first by actually installing the application and secondly how actively users are willing to participate and provide labels for the gathered contexts with respect to the campaigns specified by customers. It is thus up to the customers to design campaigns the users are willing to participate in. The success of the system thus depends on the ability of the customers, given that the system works as intended.
