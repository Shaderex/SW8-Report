%!TEX root = ../../super_main.tex

\chapter{User Stories and Acceptance Test}
\label{app:user_stories_and_acceptance_test}

\begin{center}
\begin{longtable}{| m{0.45\textwidth} | m{0.45\textwidth} |}
\hline
  \textbf{User Story}
& \textbf{Acceptance Test} \\ \hline
\endfirsthead

\multicolumn{2}{c}%
{{\bfseries \tablename\ \thetable{} -- continued from previous page}} \\
\hline
  \textbf{User Story}
& \textbf{Acceptance Test} \\ \hline
\endhead

\hline \multicolumn{2}{|r|}{{Continued on next page}} \\ \hline
\endfoot
\endlastfoot

	\multicolumn{2}{| c |}{\textbf{First iteration}} \\ \hline
	As a customer, I would like to be able to log sensor data from participants & 
	\begin{itemize}[noitemsep,topsep=0pt,parsep=0pt,partopsep=0pt]
	 	\item When you can measure and print accelerometer and gyroscope sensor data to the Android Studio log. Monitoring must be able to take place periodically and must be able to be adjusted by the programmer
	 	\item When you can measure and print the location sensor data to the Android Studio log. Monitoring must be able to take place periodically and must be able to be adjusted by the programmer.
	 \end{itemize} \\ \hline
	As a customer, I would like to get answers from questionnaires generated by me from participants & 
	\begin{itemize}[noitemsep,topsep=0pt,parsep=0pt,partopsep=0pt]
	 	\item When a customer can come up with a questionnaire, get it into the system without writing code, and have it answered by a participant.
	 	\item The questions must occur in sequence and it should only be possible to answer yes or no.
	 \end{itemize} \\ \hline
	As a developer, I would like to know how to work with sensors in Android & 
	\begin{itemize}[noitemsep,topsep=0pt,parsep=0pt,partopsep=0pt]
		\item When we know which hardware that are available and how we can communicate with them 
		\item When a section about this is written, and accepted by the group, in the report
	\end{itemize} \\ \hline
	As a developer, I would like to have a development environment & 
	\begin{itemize}[noitemsep,topsep=0pt,parsep=0pt,partopsep=0pt]
		\item When everybody in the group has installed Android Studio and it can run
		\item When continuous integration is up and running including 
			\begin{itemize}[noitemsep,topsep=0pt,parsep=0pt,partopsep=0pt]
				\item Builds
				\item Unit tests
				\item Static code analysis
				\item Code coverage
				\item Automatic builds
			\end{itemize}
		\item Code standard
		\item Version control
	\end{itemize} \\ \hline
	As a developer, I would like to know what is ethical/legally correct to log (continued to next iteration) \todo[inline]{Den fortsatte til næste iteration, hvordan dokumenterer vi det? bare sådan som det står nu, eller skal den også stå under ``second iteration''?} & 
	\begin{itemize}[noitemsep,topsep=0pt,parsep=0pt,partopsep=0pt]
		\item When rules about data legislation is known about for both Denmark and Europe
		\item When a section about this is written, and accepted by the group, in the report
	\end{itemize} \\ \hline
	
	\multicolumn{2}{| c |}{\textbf{Second iteration}} \\ \hline
	As a customer, I would like to be able to log data from many participants at a time & 
	\begin{itemize}[noitemsep,topsep=0pt,parsep=0pt,partopsep=0pt]
	 	\item If we can log and save the data persistant on a remote storage from four units
	 \end{itemize} \\ \hline
	As a participant, I would like to have my data stored persistently, anonymously and secure on my device (continued to next iteration) \todo[inline]{Marhlder: Virker lidt kunstigt at participant gerne vil have de her ting med undtagelse af secure} & 
	\begin{itemize}[noitemsep,topsep=0pt,parsep=0pt,partopsep=0pt]
	 	\item If we as developers (or other students) cannot obtain the data that has been saved in four hours, without the use of our own program, password, etc.
	 	\item The storage of the data must comply with the legislation written in the report
	 \end{itemize} \\ \hline
	As a customer, I would like to have snapshots modeled in the system, where the data is compressed (continued to next iteration) & 
	\begin{itemize}[noitemsep,topsep=0pt,parsep=0pt,partopsep=0pt]
	 	\item When there exist a snapshot model, which do not use more than 10 MB storage space using all sensors (per device) for one day
	 		\begin{itemize}[noitemsep,topsep=0pt,parsep=0pt,partopsep=0pt]
	 			\item Snapshot duration: one day
	 			\item Sample frequency: one minute
	 			\item Sample duration: 10 seconds
	 			\item Measurement frequency: 100 milliseconds
	 		\end{itemize}
	 \end{itemize} \\ \hline

	\multicolumn{2}{| c |}{\textbf{Third iteration}} \\ \hline
	As a customer, I would like to have snapshots modeled in the system, where the data is compressed & ~ \\ \hline
	As a participant, I would like to have my data stored persistently, anonymously and secure on my device & ~ \\ \hline
	As a customer, I would like to have access to the logged data specified in the created campaign & ~ \\ \hline
	

	\multicolumn{2}{| c |}{\textbf{Unknown for now}} \\ \hline
	As a customer, I would like to have my participants to be inconvenienced as little as possible by the application & 
	 \\ \hline
	As a customer, I would like the data I receive to be in a format that can be read in a machine intelligence tool & 
	 \\ \hline
	As a customer, I would like to be able to create questionnaires dynamically & 
	 \\ \hline
	As a customer, I would like to have both public and private questionnaires & 
	 \\ \hline
	As a customer, I would like to have access to the logged data & 
	\begin{itemize}[noitemsep,topsep=0pt,parsep=0pt,partopsep=0pt]
	 	\item When a customer can see the logged data using a web browser
	 \end{itemize} \\ \hline
	 As a customer, I would like to log the use of the mobile phone such as:
	\begin{itemize}[noitemsep,topsep=0pt,parsep=0pt,partopsep=0pt]
		\item Screen usage 
		\item Network usage 
		\item Calls/SMS
	\end{itemize} & 
	\begin{itemize}[noitemsep,topsep=0pt,parsep=0pt,partopsep=0pt]
	 	\item When screen-, network-, calls/ SMS-usage can be logged.
	 \end{itemize} \\ \hline
\caption{User stories and acceptance test.}
\label{tab:user_stories_and_acceptance_test}
\end{longtable}
\end{center}
