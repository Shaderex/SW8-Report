%!TEX root = ../../super_main.tex
\section{Security}

\todo[inline]{Write section about security of the communication (SSL)}

Since the system should be able to handle personal data, there are certain security constraints that we must comply with as described in \secref{sec:personal_data}. For one the legislation states that the personal data must be encrypted at all times, constraining that all ways these data are communicated must also be encrypted. For this reason we utilize an encrypted network protocol (HTTPS) using SSL encryption.

\todo[inline]{Describe briefly SSL, what is is, how it works (public private keys)}

Secure Socket Layer (SSL) is a broadly used security technology for creating a secure connection between a client and a server. SSL establishes this secure connection by utilizing both symmetric and asymmetric encryption. Symmetric encryption algorithms use the same encryption key for both the encryption and decryption of the communication, whereas asymmetric encryption encrypts the communication using one key and decrypts it using another. 
\\\\
Parameter negotiation first
Then Authentication --
Secret key exchange
%
SSL uses the asymmetric encryption to send a symmetric encryption key from the client to the server. 



\todo[inline]{Har svært ved at finde en god kilde til det her} 