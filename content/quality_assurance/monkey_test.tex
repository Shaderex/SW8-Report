%!TEX root = ../../super_main.tex

\section{Monkey Test}
\label{sec:monkey_test}
We executed UI/Application Exerciser Monkey tests on the android code. The exerciser monkey generates pseudo-random streams of user events, which can be used to test the robustness of the application. The monkey is able to stress-test the application because of frequent button clicks, etc., such that it most likely crashes if there are memory leaks or other bad implementations. It furthermore provides the possibility of simulating erratic user behavior, that might perform some trace of actions that human users would not typically follow, which might crash the application. Whenever the monkey crashes the application, the entire trace is available, but we mainly used it for the crash reports and exceptions that are available through the Android Debug Bridge. 
\\\\
We started by running one trace on 50000 inputs on a Galaxy Nexus phone, where the application crashed after 46000 actions, which allowed us to take a look at the exception and resolve it. Afterwards the monkey ran for 2 hours straight without crashing, and we therefore think the application is rather robust.
\\\\
Following this we also ran the monkey for a while on a Nexus 5 device which had Android 6.x, in contrast to the 5.x on the Galaxy Nexus. Here it also seemed to run without any difficulties, which gives some indication that our application is resilient in terms of different configurations and also compatible with different versions of Android without issues.
\\\\
We have also been using the monkey to discover cases where, if you executed a certain set of actions fast enough, it would would result in incorrect application behavior. This includes things such as subscribing to a campaign, immediately exiting the menu and then entering the same menu again, where the communication between application and server had not finished registering the participant yet. This allows us to find certain use patterns that we need to consider, or at least need to keep in mind, even if we do not have solutions for them at the time.
\\\\
It is also possible to configure CI servers to run automated monkey tests if so desired, but we did not think this was a good idea for our project. This was partially because it takes quite a bit of time to execute the monkey test, but also because it has access to the settings of the phone, which might mess up the unit tests. So given the amount of precautions, resets, etc., that might be necessary, we deemed it to take too much time to set up. 
