%!TEX root = ../../super_main.tex

\section{Monkey Test}
\label{sec:monkey_test}
In order to test the robustness of our Android application we have utilized what is known as Monkey Tests. A monkey test generates pseudo-random streams of user events, which might crash an application. The amount of successful steps before a crash occurs in such a sequence can be used as an indication of the robustness of an application.  
\\\\
Monkey tests are furthermore able to stress-test an application, because it simulates frequent button clicks, gestures, etc., such that it most likely crashes if there are memory leaks or similar pitfalls. It also provides the possibility of simulating erratic user behavior, that might perform some trace of actions that human users would not typically follow, which might crash the application. Whenever the monkey crashes the application, the entire trace is available, but we mainly used it for the crash reports and exceptions that are available through the Android Debug Bridge. 
\\\\
We started by running one trace on 50,000 inputs on a Galaxy Nexus phone, where the application crashed after 46,000 actions, which allowed us to take a look at the exception and resolve it. Afterwards the monkey ran for 2 hours straight without crashing, and we therefore think the application is rather robust.
\\\\
Following this we also ran the monkey for a while on a Nexus 5 device which had Android API level 23, in contrast to the API level 21 on the Galaxy Nexus. Here it also seemed to run without encountering any problems, which gives some indication that our application is resilient in terms of different configurations and also compatible with different versions of Android without issues.
\\\\
We have also been using the monkey to discover cases where, if you executed a certain set of actions fast enough, it would would result in incorrect application behavior. This allows us to find certain use patterns, or sequence of actions, that will crash the application. An example of such a pattern could be when subscribing to a campaign, immediately exiting the campaign and then entering the same screen again. Here the communication between application and server might not be finished registering the participant yet, which could cause potential issues.
\\\\
It is also possible to configure CI servers to run automated monkey tests if so desired, but we did not think this was a good idea for our project. This is partially because it has access to the settings of the phone, which might mess up the unit tests. So given the amount of precautions, resets, etc., that might be necessary, we deemed it to take too much time to set up. 
