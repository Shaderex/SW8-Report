%!TEX root = ../../super_main.tex

\chapter{Quality Assurance}
\label{cha:quality_assurance}

This section describes the different measures we have used in order ensure the quality of our developed product 
\\\\
\todo{Write more intro to QA}. 
When we executed tests that required that we run the code om some device, we always attempted to run it on multiple different devices in order to support several different hardware configurations and also that we are compatible with multiple Android versions. The phones we have available all run stock Android versions from the Android open source project. This might indicate that we also cover additional OS versions that are based on the stock versions, since if we find some bug when we test against the stock version, it is most likely also present in the modified version. 

%!TEX root = ../../super_main.tex

\section{Continuous Integration}
\label{sec:continuous_integration}
As mentioned in \secref{sec:extreme_programming}, we wanted a Continuous Integration (CI) server in order to help us keep our code base stable. We installed Jenkins on the same server that makes up the server part of our client-server architecture, because it was easily available. Jenkins is an open source automation server, which supports various different plugins that helps with automated builds, running tests and showing the results of those, etc. We configured the Jenkins server to be notified whenever any changes were made to the master branch of our Android or PHP Git repositories (hosted on GitHub). When notified, Jenkins will build the corresponding project and automatically run its test suite. Whenever a project would go from a previously successful build to a failing build or vice versa, the Jenkins system would send out mails to our group, notifying us that something went wrong or that it was back to normal. If a build failed, the mail would contain information regarding which tests failed, and their stack trace. This made it possible for us to give immediate attention to issues that we did not catch before pushing our content to the version control. \figref{fig:jenkins_front_page} shows the front page of the Jenkins website. In the left side, the build queue is seen, which shows if any builds are currently running. Jenkins displays helpful information about the state of the configured projects. The blue circle, depicted in the figure, changes to red if the most recent build was a failure, and besides that, it is how long time has passed since the most recent pass, most recent failure, and how long the last build took to finish. By clicking on one of the projects, different code and quality metrics will be shown, such as checkstyle warnings, code coverage, and test results. 

\begin{figure}[!htbp]
    \centering
    \includegraphics[width=\textwidth]{graphic/quality_assurance/jenkins_frontpage}
    \caption{Jenkins CI front page.}
    \label{fig:jenkins_front_page}
\end{figure}
\FloatBarrier

In our development we have utilized feature branching \parencite{feature_branching}, where each new feature would be developed on its own branch, and then, when finished, merged with the master branch. The master branch should for this reason never contain incomplete features. Previous experience with this work flow have shown that the process of merging with the master branch is error prone. Jenkins helped us unveil these errors, and ensured that we always knew if any of our tests failed, making us able to allocate people for fixing it. 


%!TEX root = ../../super_main.tex

\subsection{Coverage Metrics}
\label{sec:automated_unit_test}

Our development method states that all code we develop should be made in a test-first fashion (see \secref{sec:extreme_programming}). We therefore attempted to always make test cases for all the features we implemented. Our code coverage graphs can be seen in \figref{fig:android_project_code_coverage} and \figref{fig:php_project_code_coverage}, for our Android and PHP projects respectively. The two figures look different because we use different plugins on the CI server (due to the projects being written in different programming languages). We could have used a more strict coverage metric than line coverage, but due to prioritization of feature development we have chosen not to. Other interesting and useful coverage metrics includes branch coverage. In the Android project our line coverage percentage is $\sim 43\%$, which is relatively low. %But as explained previously, this is due to the complexity of the integration tests which were manual instead. 

% In \figref{fig:android_project_code_coverage} the line coverage is represented with green while missed lines are represented with red. In \figref{fig:php_project_code_coverage} the red line is method coverage, blue is line coverage and green is total.

\begin{figure}[!htbp]
    \centering
    \includegraphics[width=0.7\textwidth]{graphic/quality_assurance/jenkins_android_code_coverage}
    \caption{Android project code coverage}
    \label{fig:android_project_code_coverage}
\end{figure}
\FloatBarrier

In the PHP project, we have a $\sim 74\%$ line coverage, which is rather good. Most of the untested code is library- or auto generated code, and we have not tested this because we assume these parts work as they are supposed to. This is a risk assessment we have made, and deemed insignificant, in contrast to the speed we gain from using the libraries without testing them. % If we only considered coverage on the code we made ourselves, it would probably exceed 90\%. 

\begin{figure}[!htbp]
    \centering
    \includegraphics[width=0.7\textwidth]{graphic/quality_assurance/jenkins_php_code_coverage}
    \caption{PHP project code coverage}
    \label{fig:php_project_code_coverage}
\end{figure}
\FloatBarrier

\subsubsection{Manual Test}
In some cases it was infeasible to create automatic dynamic white box tests in decent time, and we therefore, in these cases, switched to a dynamic black box approach instead. Here we made test specifications, executed some part of the code manually and observed the result relative to the specifications. We mainly did this for the more complex parts of the code, such as the \mono{BackgroundSensorService} class, which schedules all the \mono{SensorProvider}s asynchronously and sends the results to the server. One of the manual tests for the \mono{BackgroundSensorService} is the \textbf{test-snapshot-from-device-to-server} found in \appref{app:manual_test}, which tests if our Android application correctly sends snapshots to the server, but also if the server correctly stores them. We also used this approach for other testing methods, such as regression and integration testing of UI features. For instance, we would, when a new UI-related feature was implemented, manually test if previous UI features still worked. We did not spend time on specifying written test-cases on the most basic of these tests, such as ``is it possible to subscribe to a campaign?'', since these would probably be found during ad-hoc testing while developing new features.
\newpage
These manual tests resulted in a generally lower line coverage, especially on the Android project, since the CI plugin we used was unable to compare our manual tests to the code base. This does not necessarily mean that the code is less tested, but it is harder to tell when tests are covering the code base well. 

\subsection{Static Code Analysis}
Besides using coverage metrics, we have used some metrics produced by static code analysis to improve the quality of our code base. We have used a linter, which will produce warnings on code that is known to be error prone, which helps us avoid common mistakes on the Android platform and Java in general. Furthermore, we have used check-style, which helps us find structural inconsistencies in our code base. Effectively, this type of metric will assist us in keeping our code base uniform and look consistent, which will make it easier for all developers to read the code, since all parts of the code base is structured in the same way.


\section{Monkey Test}
\label{sec:monkey_test}
We executed UI/Application Exerciser Monkey tests on the android code. The exerciser monkey generates pseudo-random streams of user events, which can be used to test the robustness of the application. The monkey is able to stress-test the application because of frequent button clicks, etc., such that it most likely crashes if there are memory leaks or other bad implementations. It furthermore provides the possibility of simulating erratic user behavior, that might perform some trace of actions that human users would not typically follow, which might crash the application. Whenever the monkey crashes the application, the entire trace is available, but we mainly used it for the crash reports and exceptions that are available through the Android Debug Bridge. 
\\\\
We started by running one trace on 50000 inputs on a Galaxy Nexus phone, where the application crashed after 46000 actions, which allowed us to take a look at the exception and resolve it. Afterwards the monkey ran for 2 hours straight without crashing, and we therefore think the application is rather robust.
\\\\
Following this we also ran the monkey for a while on a Nexus 5 device which had Android 6.x, in contrast to the 5.x on the Galaxy Nexus. Here it also seemed to run without any difficulties, which gives some indication that our application is resilient in terms of different configurations and also compatible with different versions of Android without issues.
\\\\
We have also been using the monkey to discover cases where, if you executed a certain set of actions fast enough, it would would result in incorrect application behavior. This includes things such as subscribing to a campaign, immediately exiting the menu and then entering the same menu again, where the communication between application and server had not finished registering the participant yet. This allows us to find certain use patterns that we need to consider, or at least need to keep in mind, even if we do not have solutions for them at the time.
\\\\
It is also possible to configure CI servers to run automated monkey tests if so desired, but we did not think this was a good idea for our project. This was partially because it takes quite a bit of time to execute the monkey test, but also because it has access to the settings of the phone, which might mess up the unit tests. So given the amount of precautions, resets, etc., that might be necessary, we deemed it to take too much time to set up. 

\section{Pair Group Review}
\label{sec:pair_group_review}

Two times during the project period we arranged meetings with another software student group at Aalborg University, where we presented the current states of our projects and provided critique for each other. One meeting was arranged halfway through the semester and another was arranged three weeks before project delivery. 

\subsection{First Meeting: Project Idea}
\label{sub:first_meeting_project_idea}
The goal of the first meeting was to evaluate the general project idea of the opponent group and also suggest improvements for the developed system. At this meeting the two groups were unfamiliar with the product of the other group, so an agenda was created that should increase the understanding for the opponent:

\begin{itemize}
    \item Introduction
    \item Demonstration
    \item Evaluation
\end{itemize}

We took turns in presenting, so the first group started by introducing their problem area, how they were going to solve the problem and which customers might be interesting in using the developed product. After the presentation of the problem area, the group presented the current state of the product, which was followed by two different evaluations from the opponent group. One evaluation was regarding the idea and the marketability of the solution, while the other was regarding how the product could be improved. 
\\\\
The result of this meeting were mainly on the idea level, and not so much about the product specifically. Our opponent group had concerns regarding how we were going to persuade participants to use the system, such that we could actually gather the data that we claimed would be available for our customers. We decided that this was not a big concern, since the customers would be interested in branding their campaigns in order to gather participants, or might offer rewards for the participants. In any case, we think that fact that we provide a platform for customers to use, so that they only have to provide some incentive for participants to join, is an improvement in itself. We furthermore hope that the participants who join for the rewards of the branded campaigns would be interested in selecting campaigns that only offered small or no rewards. We think it is realistic to assume, that if customers do not offer any rewards whatsoever on their own without our system, they would probably not get any participants anyway. 

\subsection{Second Meeting: Product Evaluation}
\label{sub:second_meeting_product_evaluation}

At the second meeting, both the groups' products were in an almost finished state, and were presented for the other group. The goal of this meeting was to determine if there were any obscure parts of the system, bugs, possible usability improvements or extensions to the system. Our opponent group could not find any bugs in the system, but they found some usability improvements and an extension that would be nice to have in the system. The suggestions are represented in \tabref{tab:suggestions_from_opponent_group}. 

\begin{table}[!htbp]
    \centering
    \begin{tabular}{|l|p{0.8\textwidth}|}
    \hline
    \textbf{Category} & \textbf{Description} \\ \hline
    Extension & Run questionnaires at specific times of the day, instead of relative to time of joining campaign.  \\ \hline
    Usability & Put the definition of our time intervals on the website, or show it with an image. \\ \hline
    Usability & Move the active campaign to the top of the campaigns list instead of where it was located before. \\ \hline
    Usability & Have focus on the size differences on the website. Many items are rather small, so it is hard to determine what is important. The most important things should be large. \\ \hline
    \end{tabular}
    \caption{Suggestions from opponent group.}
    \label{tab:suggestions_from_opponent_group}
\end{table}

These were all valid suggestions, but due to the meeting being held a bit too close to the project deadline, we had to down prioritize the development. Effectively, this resulted in only the second entry being implemented: \emph{Put the definition of our time intervals on the website, or show it with an image}. There is now an image on the website which represents how to understand the different intervals you can specify on the ``create campaign''-site. The other three entries are things that would be nice improvements to the system, and definitely something that should be worked on, if the system was to be extended at some point. 



