%!TEX root = ../../super_main.tex

\section{Sensor Providers}
\label{sec:sensor_providers}

We have implemented a threaded abstraction over the different data sources that the system should utilize in order to gather context for the labels that will eventually be combined to form the output training data for the system. The abstraction is threaded because we want the Android application to be able to gather information from multiple sources concurrently. This is done in order to make sure that the gathered data is obtained temporally close to when the label for the data was obtained. 

\subsection{Napkin Mathematics}
\label{sub:napkin_mathematics}

%This gave us a dataset with a size of 142908 bytes. This experiment would then yield a data set of approximately 205 MB if it was run for a day. Running the same experiment for 30 days with 100 different devices would then approximately yield a data set of 617 GB assuming similar mobile devices with similar sensors. This quick napkin math was only for one sensor with 100 people and this could quickly escalate if more sensor or more people are added. 
As described in \secref{sec:availability_of_data_sources}, some data sources are continuous. We have experimented with different continuous sensors on a Nexus 5 smart phone and logged all values captured from the sensor for 1, 5, 20 minutes. The results of these tests can be seen in \tabref{tab:sensor_experiment}. Note that the orientation sensor is a virtual sensor, which uses data collected from both the gyroscope and magnetometer, hence the correlation between the data sizes ($143 \text{KB} + 35  \text{KB} \approx 177  \text{KB}$ for 1 minute). These tests were only performed for four different sensors, but several different sensors may be wanted, thus increasing the amount of data collected even further. One should also consider that this data is only collected for a single phone / person. Collecting the data from several phones / people would further escalate the data size. These quantities of data might present a problem even on modern mobile platforms due to paid limited data plans and battery consumption. There might be different data needs, some customers might require very detailed data from many sensors from a few devices and others might require more sparse data from a few sensors from a lot of different devices. 

\begin{table}[!htbp]
\centering
\begin{tabular}{l|c|c|c|c}
\textbf{Sensor}     & \textbf{Accellerometer} & \textbf{Gyroscope} & \textbf{Magnetometer} & \textbf{Orientation} \\ \hline
\textbf{1 minute}   & 142 KB                  & 143 KB             & 35 KB                 & 177 KB               \\ \hline
\textbf{5 minutes}  & 714 KB                  & 714 KB             & 178 KB                & 892 KB               \\ \hline
\textbf{20 minutes} &                         &                    & 715 KB                & 3573 KB                
\end{tabular}
\caption{Data size of sensor data collection after a set amount of time.}
\label{tab:sensor_experiment}
\end{table}

% Træls for bruge
% Træls for folk der kan bruge data

\subsection{Temporality}

% Det bliver hurtigt til meget data når man arbejder med continious data
% Vi bruger sample frequency til at lave mellemrum mellem samples i tid fordi det ikke er så brugbart at have 8gb data for et minuts målinger
% Vi vil gerne have flere measurements per sample så vi kan undgå upræcise målinger

We have come up with a solution to the concerns described in \secref{sub:napkin_mathematics}. We have decided to facilitate periodic data collection instead of continues readings of data in order to facilitate different data needs.

\begin{figure}[!htbp]
    \centering
    \includegraphics[width=\textwidth]{unsorted/sample_temporality}
    \caption{Overview of sample and measurement temporality.}
    \label{fig:sample_temporality}
\end{figure}

\subsection{Implementation}

The solution was to abstract data collection away in subclasses of a class called.. \todo{Write more}

% Future<T> Objects
% 

% \lstinputlisting[
%    style = java,
%    caption = {Property similarity on a component.},
%    label = {lst:attribute_difference},
%]{content/implementation/annotation/attribute_difference.java}