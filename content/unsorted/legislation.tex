%!TEX root = ../../super_main.tex

\section{Legislation}
\label{sec:legislation}

Personal data is defined as information that allows for the identification of a single individual, either directly or indirectly. Directly referring to using unique identifiers such as CPR number, while indirectly could be done by tracking a person's movements and using this to identify the individual. The Danish laws about protecting personal data classifies personal data into different categories: Normal data and sensitive data. Sensitive data is defined as revealing racial or ethnic origin, political opinions, religious or philosophical beliefs, trade-union membership, and data concerning health or sex life \parencite{datatilsynet_stud1}, while normal data refers to everything else, such as name and address.  
\\\\
When located in Denmark one has to notify, and receive permission from, the Danish Data Protection Agency (DDPA), before being allowed to process personal data. When processing data the reason has to be well-founded, and the processed data should be the minimum amount required in order to fulfil the purpose of the data collection. However, special rules apply for university students. The main difference for students is, that the DDPA does not have to be notified, and no permission is needed. The rest of the general laws regarding personal data still hold, such as encrypting data transfers, ensuring limited (password protected) access to the data, and having consent from the users. 
\\\\
There are various statutory demands to take into consideration if the project was to be used outside the context of our study, but we will disregard these as they are not required for student projects. 




%protecting personal data conforms to directive 95/46/EC of the European Parliament \parencite{eu_personal_data_law},