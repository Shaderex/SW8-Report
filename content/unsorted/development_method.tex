%!TEX root = ../../super_main.tex

\section{Development Method}
\label{sec:development_method}

This section will describe the development method that was used in this project. We choose to use Extreme Programming (XP) \parencite{xp}, with a few changes, which will be described in this section.
\\\\
% Vi ville gerne lave TDD, så derfor tænkte vi at XP var godt
% Vi ville også gerne prøve at bruge XP
The main reason why we considered XP, was because we wanted a more concrete and test-driven method compared to other agile methods such as Scrum, which we have had experience with on previous semesters.
\\\\
The reason for the choice of this particular method is that it is test driven. 
\\\\
In previous projects an issue have been that testing became a more and more ad-hoc item of the development rather than a planned task of the development. We hope to achieve a structured approach to enforce testing, with as little overhead, in regards to testing, as possible. Furthermore knowledge sharing is also encouraged in XP by pair programming, which is preferable for a student project since all members of the group should have insight in the product and the report written. By using pair programming we achieve a higher level of knowledge sharing and at the same time increase the quality of the software produced. XP encourages to have rather short iterations in comparison to other agile development methods, which assists us, as a team, to steer the project with higher accuracy. Lastly XP also have techniques to handle uncertainty in a structured way, by the use of spikes, which can be useful in a problem based learning project as this one. 
\\\\
A possible issue with XP is that an on-site customer is a core idea of this method. However, this project is a student project with no paying customer or user. This means that, we must role play the on-site customer during the development. This might be challenging since we will have a hard time entering the mindset of a customer. For this reason during the iteration review and planning meetings we assign each team member roles, which they must play. This means that one team member is assigned to role play the customer during the entire review/planning meeting. Other roles include tester, developer, user. This role playing makes it easier for us as a team to prioritize story cards because one team member only have to juggle one mindset during these meetings. Another issue with XP could potentially be that it does not have any built in techniques to produce a student report during development. In general XP does not encourage documentation. Documentation is only considered in XP if it gives the customer value. We will have to alter the traditional approach to include producing a student report. Another potential pitfall for using XP is the lack of quality assurance (QA) team assigned to the project. A QA team assists the development team to spot potential issues with acceptance test. Because of the fact we have not been working solely XP in previous projects we cannot evaluate the quality of acceptance test based on prior experience.