%!TEX root = ../../super_main.tex

\section{Development Methodologies}
\label{sec:development_methodologies}

In software development, there exist various methodologies. The most common practice in traditional software development is the document driven using a classic water fall approach. This approach emphasizes that the developers should put a lot of effort in planning and separating the development in parts. For instance a software project could be separated into five parts. A part for analyzing and understanding the problem domain at hand, hence producing a requirement specification. A part for designing the product, hence converting requirements to a viable design that describes what technologies should be utilized and the structure and architecture of the system accordingly. A part for testing if the system is stable and requirements are actually met. Finally a part that focuses on maintaining the developed system. This separation allows for specialization, meaning that the specialist in different areas of software development can perform their best. For instance in the testing phase of the development life cycle an experienced tester could perform great tests to ensure that the quality of the final system does what it is supposed to do. However since the delegation of work often shifts from one expert to the another, this development methodology spends a lot of resources on producing documents in order to share knowledge between experts. 

\todo{Write about agile software development and why this is preferred.}
\todo{Add that a new paradigm (essence) of software development methodologies is being touched upon by us.}