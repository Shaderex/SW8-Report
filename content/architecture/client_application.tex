%!TEX root = ../../super_main.tex

\section{Client Application}
\label{sec:client_application}

The device component is the part of the system that should run directly on the smartphone of the participant. This component is responsible for interacting with the participants, gather data from the sensors on the device, and communicating with the other devices in order to get data from wearables. The sensors represent the underlying hardware, and the Wearable Application is the interface from the client application to external wearable sensors, which in our case is the Microsoft Health application that communicates with Microsoft Band 2 we have implemented support for in the system.
\\\\
The smartphone application contains a service component that is the main controller of the client application. This service runs independently in the background at all times, and is the part of the system on a device that communicates with the different sensor providers, and also uploads the collected snapshots to the server. Furthermore it is also responsible for prompting the participants to answer the questionnaires. This service is completely detached from the graphical interface and it runs in its own seperate process, since it needs to be responsive to changes read by sensors and in other data sources. This seperation is furthermore beneficial, because it allows the Android OS to properly close and free the UI components when they are not used and deallocate all memory used by them. 
\\\\
The UI components consists of two interfaces, a settings interface and a questionnaire interface. The settings interface allows for the participant to browse details and join campaigns. Furthermore, it also has some communication with the web server where it fetches the specifications of campaigns, and reports back to the server which campaigns the participants has joined. The questionnaire interface is a series of user inputs where the participants can provide answers for the questions of a questionnaire. This interface is prompted to the user by using notifications, which are sent by the service for every snapshot, in order to obtain labels on the data if possible.
\\\\
The service component is also in need of some persistent storage on the device, to ensure that it can store gathered snapshots, so that it can upload the data using best practices in regards of power consumption and robustness as described in \secref{sec:general_strategies}. For this reason the application has a storage module where we utilize a library called \mono{realm}, which will be further describe in \secref{sub:local_storage}.
\\\\
Lastly the service component is heavily dependent on the sensor providers, which are providers that abstracts various types of sensors, for example hardware sensors on the device, software sensors, and external sensors from wearable and so on. From the service point of view, it only needs to manage the specification of campaigns and request the data from these providers accordingly before it stores the data that has been gathered from these providers on the device. The controller of the service assumes that the interface of these providers guarantees the amount of data requested and that they keep their deadlines in regards to timing of measurements.

%!TEX root = ../../../super_main.tex
\subsection{Client Requests}
\label{sub:client_requests}

The Android client needs to be able to access the routes mentioned in \secref{sub:server_interface}, in order to properly present campaigns on the device, and create/send snapshots correctly. There are a few different situations where such interaction between the client and server should take place. Some interactions are GUI related, and some are related to the retrieval of necessary information for the application to perform data gathering. Most of the aspects of communicating over HTTP, such as establishing a connection and encoding its messages, can be generalized. To simplify the process of generalizing HTTP requests, we chose to utilize a library, called Webb\footnote{https://github.com/hgoebl/DavidWebb}, for handling all the HTTP communication and encoding through a simple interface. 
\\\\
There are still some aspects that can be further improved upon, even though we use this library. This includes ensuring that the communication happens asynchronously, and thereby does not block the application's main thread; ensuring that the right headers and request types are sent; and lastly making some modifications to the TLS verification (see \secref{sec:transport_layer_security}). We generalized these aspects by inheriting from a class in the Android framework called \mono{AsyncTask}, which is an abstract class with an abstract method, named \mono{doInBackground}. This is the primary method of the \mono{AsyncTask}, and is the one method that is executed in a background thread. 
\\\\
Besides the abstract method, the class also contains different empty overridable methods that will be called during the task's life cycle, such as \mono{onPostExecute}, which as the name indicates will be called after the execution of the background task. This method will be run on the main thread of the application, which, for example, allows it to modify GUI elements. We found that, in generalizing the communication with the server, we needed to ensure that the HTTP response codes matched what we expected in the different situations. This included handling of what to do if the status was OK (status code 200), and also specifying what to do if the response code was not what we expected. We also figured that we often would need to specify what should happen if no connection to the server is available, such that no response code is returned. Therefore we overrode the \mono{onPostExecute} method as seen in \lstref{lst:on_post_execute}. 

\lstinputlisting[
   style = Java,
   caption = {The \mono{onPostExecute} method, which is called after the asynchronous task has completed.},
   label = {lst:on_post_execute},
   float=!htbp,
]{content/architecture/code_snippets/onPostExecute.java}
\FloatBarrier

In the snippet we firstly check if the response parameter is null, in which case something went wrong in establishing the connection to the server, and the abstract method \mono{onConnectionFailure} will be called. Otherwise we check if the response code matches what we expect, and if it is we call the abstract method \mono{onResponse\-Code\-Matching} and if it does not we call \mono{onResponseCodeNotMatching}. These methods must be overridden when inheriting from this class. The expected response code is set through the constructor of the class, along with the URL the request should be sent to, and which HTTP request method should be used. The response that the \mono{onPostExecute} method receives as an argument, is the result of the doInBackground process, which can be seen in \lstref{lst:do_in_background}. 

\lstinputlisting[
   style = Java,
   caption = {The tasks responsible for sending the request.},
   label = {lst:do_in_background},
   float=!htbp,
]{content/architecture/code_snippets/doInBackground.java}
\FloatBarrier

In the \mono{doInBackground} method we firstly create an \mono{SSLContext} in \linesref{lst:switch_method_start}{lst:switch_method_end}, which is related to the Transport Layer Security that we utilize and which is described in \secref{sec:transport_layer_security}. This is followed by setting some of the properties that we want to apply to every request, such as adding a header describing that the request should return JSON data and not a HTML response as normally used in browsers. 
\\\\
This is followed by checking which HTTP request type the request should be sent with in \linesref{lst:switch_method_start}{lst:switch_method_end}. We then utilize the Webb library to create a request of the given type, and pass it as an argument to the abstract method, \mono{sendRequest}. E.g. if the wanted request method is \mono{POST}, the \mono{sendRequest} method will be called with \mono{webb.post(url)} as seen in \lineref{lst:webb_post}. The \mono{sendRequest} method must, as with the previously mentioned abstract methods, be overridden.
