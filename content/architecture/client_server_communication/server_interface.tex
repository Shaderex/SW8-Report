%!TEX root = ../../../super_main.tex
\subsection{Server Interface}
\label{sub:server_interface}
The interface from the clients to the server also needs to be considered. We chose to follow the representational state transfer (REST) principles \parencite{http_manden} and implement a RESTful API. This contributes to the ability of our architecture to handle an increasing amount of clients. The REST principles involve having a layered system with a uniform interface from the perspective of all devices, regardless of how we choose to expand the capacity of the back end. This means that we can use modern technologies to handle load balancing and caching for web APIs and applications, allowing us to serve more users, without these changes being apparent on the client side. This will help scaling out to many servers, by balancing work to all available server instances, and scaling up by providing better performance on individual machines through caching. This effectively means that we can reduce the problem of scalability to implementing a proper RESTful interface. 
\\\\
The communication between the server and the client will be done over HTTP, and we will ensure that each resource, e.g. a campaign or a snapshot, will have a Unique Resource Identifier (URI). In our system we need to store three different kinds of resources on our server, namely campaigns, snapshots, and participant information. The main type of resource in our system is the campaign, and each other resource will only exist in the system if it is linked to a campaign, meaning that it makes sense to form the URI for these associated items as a part of a campaign. \tabref{tab:api_routes} shows the routes of the system that have something to do with retrieving these resources and the associated information (the first two routes), and joining and uploading snapshots (the last two routes). As seen in \tabref{tab:api_routes}, all of these routes are prefixed with ``\mono{api}'', which is used to indicate that these routes will return a response in JSON-format. We choose the JSON format for our responses because there are many heavily used libraries that allows easy handling of such data on the client side, which we have previous experience working with. It is also a nice format to fill out UI's components with, because every value has a corresponding key, which can easily be used to match with specific components in a view. 

\begin{table}[!htbp]
    \centering
    \begin{tabular}{|l|l|l|} 
        \hline
        \textbf{Name} & \textbf{Method} & \textbf{URI}                                  \\ \hline 
        \emph{show all} & \mono{GET }   & \mono{api/campaigns}                          \\ \hline 
        \emph{show one} & \mono{GET }   & \mono{api/campaigns/\{identifier\}}           \\ \hline 
        \emph{join}     & \mono{POST}   & \mono{api/campaigns/\{identifier\}/participants}\\ \hline 
        \emph{upload}   & \mono{POST}   & \mono{api/campaigns/\{identifier\}/snapshots} \\ \hline 
    \end{tabular}
    \caption{The routes that the client uses to send and get information to and from the server.}
    \label{tab:api_routes}
\end{table}
\FloatBarrier

The first two routes use the HTTP \mono{GET} request method to retrieve information regarding campaigns. The \emph{show all} route is requested from the device to return a list of identifiers, names and authors for all publicly available campaigns. An identifier from the list of all campaigns can be used in the \emph{show one} route, which returns all the specifications of a specific campaign. The route captures the \mono{identifier} in the URI as a variable, which can be used to look up a record in our database. Assuming that a campaign with an id of 5 exists in the database, the route \mono{api/campaigns/5} would return a JSON response containing all details about this campaign. 
\\\\
The final two routes shown in \tabref{tab:api_routes} both use the HTTP \mono{POST} request method, meaning that they are used for upload. The \emph{join} route is utilized when a participant, through the client application, joins a campaign, and the \emph{upload} route is used for uploading snapshots after participants have joined a campaign and are contributing to it. In the \emph{upload} and \emph{join} routes, the \mono{identifier} is also identifying a campaign, and the data, such as the device identifier for the \emph{join} route or the snapshot data for the \emph{upload} route, are stored in the body of the request. 
\\\\
Besides the routes for the Android client, there also exist routes related to the user interface of the web application. These routes are not prefixed with ``\mono{api}'', and are used for creating and managing the campaigns in the system and retrieve information about the campaigns and their submitted snapshots. 

\subsubsection{Server Names}
\label{sub:server_names}

A problem that we have dealt with during the setup of the server was that it was placed on the university. Problems arose when we wished to access the server from outside the university network. We had to request the network committee on AAU to gain access from the outside of the university. However, in doing this, we would have to utilize another IP address to access the server from outside the university network. This would not have been a problem if the IP address, that was utilized to access the server outside the university network, also worked when one was connected to the university network, however this was not the case. Furthermore, we also wanted two version of our web site and API: one for development and one for production to ensure that we had a stable version of the server side available. The problem is that the physical server somehow should be able to distinguish these two versions from each other. 
\\\\
To solve these problems, we introduced different subdomains to our site as can be seen in \tabref{tab:server_names}. The primary domain name ``\mono{element67.dk}'' is a domain owned by a group member, which already had access to editing DNS records for this domain. The local domains have DNS records where the server name is mapped to the internal IP address on the university, whereas the global ones map to the external IP address. 

\begin{table}[!htbp]
    \centering
    \begin{tabular}{|l|l|l|} \hline 
                                  & \textbf{Production}                    & \textbf{Development}            \\ \hline  
        \textbf{Internal} & \mono{prod.local.element67.dk}         & \mono{dev.local.element67.dk}   \\ \hline 
        \textbf{External} & \mono{prod.global.element67.dk}        & \mono{dev.global.element67.dk}  \\ \hline
    \end{tabular}
    \caption{The different server names that we need.}
    \label{tab:server_names}
\end{table}

