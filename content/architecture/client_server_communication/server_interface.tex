%!TEX root = ../../../super_main.tex
\subsection{Server Interface}
\label{sub:server_interface}
The interface from the clients to the server also needs to be considered. We chose to follow the REST principles \todo{ref til tidligere rapport eller web guys phd thesis} and implement a RESTful API. This contributes to our architectures ability to handle an increasing amount of clients. The REST principles involves having a uniform interface from the perspective of all devices regardless of how we choose to expand the capacity of the back end. This means that we can add sophisticated tools to handle load balancing and caching for web APIs and applications such as Varnish \footnote{https://www.varnish-cache.org/} and HAProxy \footnote{http://www.haproxy.org/} to allow us to serve more users, without changing anything on the client side. Such technologies both help scaling out to many servers, by balancing work to all available server instances, and scaling up by providing better performance on individual machines through caching. This effectively means that we can reduce the problem of scalability to implementing a proper Restful interface. 
\\\\
The communication between the server and the client will be done over HTTP, and we will ensure that each resource, e.g. a campaign or a snapshot, will have a Unique Resource Identifier (URI). In our system we have need for storing three different kinds of resources on our server, namely campaigns, snapshots, and participants. The main object in our system is the campaign, and each other object will only exist in the system if it is somehow linked to a campaign, meaning that it makes sense to form the URI for these associated items as a part of a campaign. 
\\\\
\tabref{tab:api_routes} shows the routes of the system that have something to do with retrieving campaign information (the first two routes), and joining and uploading snapshots (the last two routes). As seen in \tabref{tab:api_routes} all of these routes is prefixed with ``api'' which is used to indicate that these routes will return a response in JSON-format. As mentioned the first two routes are used for retrieving information about campaigns, hence they utilize the GET requet method. The \emph{show all} route is requested from the device to return a list of identifiers, names and authors for all publicly available campaigns. The identifier from the request can be used using the \emph{show one} route, which return all the specifications of a specific campaign provided an identifier. The route captures the \mono{identifier} between the curly braces, provides it as a variable, which can be used to look up a record in our database. Assuming that a campaign with an id of 5 exists in the database, the route \mono{api/campaigns/5} would return a JSON response containing all details about this campaign. The final two routes shown in \tabref{tab:api_routes} are both ``POST'' routes, meaning that they are used for upload. The \emph{join} route is utilized when a participant, through the client application, joins a campaign, and the \emph{upload} route is used for uploading snapshots after participants have joined a campaign and are contributing to it. 

\begin{table}[!htbp]
    \centering
    \begin{tabular}{|l|l|l|} 
        \hline
        \textbf{Name} & \textbf{Method} & \textbf{URI}                                  \\ \hline 
        \emph{show all} & \mono{GET }   & \mono{api/campaigns}                          \\ \hline 
        \emph{show one} & \mono{GET }   & \mono{api/campaigns/\{identifier\}}           \\ \hline 
        \emph{join}     & \mono{POST}   & \mono{api/campaigns/\{identifier\}/participants}\\ \hline 
        \emph{upload}   & \mono{POST}   & \mono{api/campaigns/\{identifier\}/snapshots} \\ \hline 
    \end{tabular}
    \caption{The routes that the client uses to send and get information to and from the server.}
    \label{tab:api_routes}
\end{table}

Besides the routes for the Android client there also exist routes related to the user interface of the web application. These routes are not prefixed with ``api'', and are used for creating and managing the campaigns in the system and retrieve information about the campaigns and their submitted snapshots. These routes are furhter described in \secref{sec:customer_interaction}. 