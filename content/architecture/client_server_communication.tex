%!TEX root = ../../super_main.tex
\todo[inline]{Describe google cloud messaging as something we looked at early, but never got arround to using other than the device id}
% In the beginning we though a lot about using GCM
% It wasn't that necessary
% However it yielded unique device identifier
% Furthermore it could later be used to send push notifications to the devices
\section{Client-Server Communication}
\label{sec:client_server_interface}

\subsection{Server Interface}
\label{sub:server_interface}
The interface from the clients to the server also needs to be considered. We chose to follow the REST principles \todo{ref til tidligere rapport eller web guys phd thesis} and implement a RESTful API. This contributes to our architectures ability to handle an increasing amount of clients. The REST principles involves having a uniform interface from the perspective of all devices regardless of how we choose to expand the capacity of the back end. This means that we can add sophisticated tools to handle load balancing and caching for web APIs and applications such as Varnish \footnote{https://www.varnish-cache.org/} and HAProxy \footnote{http://www.haproxy.org/} to allow us to serve more users, without changing anything on the client side. Such technologies both help scaling out to many servers, by balancing work to all available server instances, and scaling up by providing better performance on individual machines through caching. This effectively means that we can reduce the problem of scalability to implementing a proper Restful interface. 
\\\\
\todo[inline]{Vi skal måske passe på med at kalde vores API restful. Vi mangler nogle essentielle ting, så som links eller i det mindste informationer om hvordan man skal komme videre i systemet. http://roy.gbiv.com/untangled/2008/rest-apis-must-be-hypertext-driven .. THE MAN HIMSELF.. En hjemmeside er restful i det alle mulige transitioner for brugeren bliver præsenteret i form af links og brugerens state er repræsenteret på siden. }

The communication between the server and the client will be done over http, and we will ensure that each resource, e.g. a campaign or a snapshot, will have a Unique Resource Identifier (URI). In our system we have need for storing three different kinds of resources on our server, namely campaigns, snapshots, and participants. The main object in our system is the campaign, and each other object will only exist in the system if it is somehow linked to a campaign, meaning that it makes sense to form the URI for these associated items as a part of a campaign. As can be seen in \tabref{tab:api_routes} each of the routes meant for communication between the Android client and the server has been prepended with the ``api'' prefix to indicate that each of these routes will return a response containing JSON code. 

\todo[inline]{This could have been done by utilizing HTTP header stating that the requester wants JSON instead, but it would complicate some of the controller code.}

The table then reveals that we for the android application only utilizes two types of requests namely GET and POST requests, which is normally respectively used for getting and storing new instances of resources. The first route will receive GET requests and retrieve a list of campaigns in form of essential information to display, such as name, description, the author of it and its id\todo{Here there would not only be a id if it was truly restful, but also the uri for joining the campaign, the uri for retrieving the remainder of the information}, which can be used to fetch the entire specification through the use of the second route. In the case of this route we utilize the curly braces to indicate that it should be replaced by the identifier for the given resource, e.g. if we wanted to retrieve all the information related to the campaign with a id of 5, its URI would be ``\mono{api/campaigns/5}''. To save a new snapshot for a specific campaign the third route would be used, where the body of the request send would contain the data of the snapshot, and to join a campaign the last route would be used with a body containing an identifier of the campaign and a identifier to describe which phone has joined.

\begin{table}[!htbp]
    \centering
    \begin{tabular}{|l|l|} 
        \hline
        \textbf{Type} & \textbf{URI}                                  \\ \hline 
        \mono{GET }   & \mono{api/campaigns}                          \\ \hline 
        \mono{GET }   & \mono{api/campaigns/\{identifier\}}           \\ \hline 
        \mono{POST}   & \mono{api/campaigns/\{identifier\}/snapshots} \\ \hline 
        \mono{POST}   & \mono{api/campaigns/join}                     \\ \hline 
    \end{tabular}
    \caption{The routes that the client uses to send and get information to and from the server.}
    \label{tab:api_routes}
\end{table}
\todo[inline]{join should have been \mono{api/campaigns/\{identifier\}(join)}}

Besides the routes for the Android client we have also created some for the web application that the customers would use. 
% Follows defacto standards for how routing should look and it is uniform for our api
% There are also some routes for customers to use through a browser:
\begin{table}[!htbp]
    \centering
    \begin{tabular}{|l|l|} 
        \hline
        \textbf{Type} & \textbf{URL}                            \\ \hline 
        \mono{GET}    & \mono{campaigns}                        \\ \hline 
        \mono{POST}   & \mono{campaigns}                        \\ \hline 
        \mono{GET}    & \mono{campaigns/create}                 \\ \hline 
        \mono{GET}    & \mono{campaigns/\{campaign\}}           \\ \hline 
        \mono{GET}    & \mono{campaigns/\{campaign\}/snapshots} \\ \hline 
        \mono{GET}    & \mono{home}                             \\ \hline 
        \mono{POST}   & \mono{login}                            \\ \hline 
        \mono{GET}    & \mono{login}                            \\ \hline 
        \mono{GET}    & \mono{logout}                           \\ \hline 
        \mono{POST}   & \mono{password/email}                   \\ \hline 
        \mono{POST}   & \mono{password/reset}                   \\ \hline 
        \mono{GET}    & \mono{password/reset/\{token?\}}        \\ \hline 
        \mono{POST}   & \mono{register}                         \\ \hline 
        \mono{GET}    & \mono{register}                         \\ \hline 
    \end{tabular}
    \caption{The routes that the is used on the website.}
    \label{tab:browser_routes}
\end{table}


% Talk about middleware mayhaps?
\subsection{Client Requests}
\label{sub:client_requests}

% Describe how the client makes requests
\subsection{Request handling}
\label{sub:request_handling}


