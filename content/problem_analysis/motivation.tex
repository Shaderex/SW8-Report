%!TEX root = ../../super_main.tex

\section{Motivation}
\label{sec:motivation}
% mobile devices er blevet mere udbredt blandt folk --> giver mulighed for at indsamle data
% Svært at komme et sted hen (i DK) uden at der er mobile devices i nærheden
% Nævn fluff om at det er en område under udvikling
% ``ubiquitous''
The area of ubiquitous devices is rapidly changing. In recent years mobile technologies have had a major growth in consumers. The sales of smart phones have increased in sales by a factor of 11 since 2007 \parencite{statsia_smartphones}. Other mobile technologies, like wearables, have also seen an increase in demand in the last couple of years. FitBit, a company selling smart wristbands, have seen a exponential increase in sales of their device in the past five years \parencite{statsia_fitbit}. New technologies, which have the potential to change the way we think about these ubiquitous devices, emerge and are difficult to ignore. An example of this is the Google Contact Lens \parencite{google_contact_lens} which is able to measure the blood sugar level of the wearer and transmit it wirelessly. Another new technology that has emerged is the ``big.LITTLE'' ARM CPU architecture \parencite{big_little_architecture}, which is a multi-core processor architecture that allows for better adjustments to dynamic computing needs and use less power when possible. It combines the best of power hungry and powerful ARM CPU cores with the best of slow, power efficient ones to achieve overall better performance and power consumption. This architecture allows for battery draining high performance on demand and very power efficient background processing when a device is not used. The development of the Google Contact Lens suggests that new mobile sensor technologies are continuously being made which gives access to additional wearable sensors. Because this type of sensors cannot be implemented in smart phones as we know them today, some context can only be described using external sensors, appearing in new sorts of ubiquitous devices. Even though new types of sensors makes sense in smart phones, it is not a guarantee for them to be put into new devices. Some manufactures have value edition of smart phones where a low price is preferred over many sensors. For instance Motorola sells both Moto E (a value edition) and Nexus 6 (a flagship model) where the price and specification varies \parencite{compare_moto_e_nexus_6}.
\\\\

% Hvis man gerne vil vide om context skal man spørge folk explicit
% Man kunne bruge MI til at udlede contexten uden at spørge folk explicit men blot måle dem implicit
% Reality mining vil geren vide noget om folks opførsel, men ingen MI data
% Telehealth vil også gerne vide noget om folks opførsel og sundhed, men ingen MI data

The expansion in mobile technologies opens new opportunities for studying and understanding the context around the user more accurately. This means that we can obtain better knowledge regarding the users of a system, and understanding people in general. A common practice today for deriving the context around the user is to ask them explicitly by surveys or similar techniques. A field of research where a less biased measurement of context would be appreciated is reality mining. Reality mining is a study of human behavior and social activities. It is a special area of data mining, where Machine Intelligence principles are applied to datasets gathered from electronic sensors in everyday scenarios. Reality mining gives researchers and industry an alternative to basing research and models solely on data provided through people, i.e. from public polls, focus groups and questionnaires. In order to understand human dynamics with less bias, a study from 2006 \parencite{eagle2006_reality_mining_definition} showed that collecting data from a hundred people using their phone showed some promising results. However, they concluded that data from new types of sensors could have a lot of potential in regards to reality mining. 
\\\\
The understanding of human dynamics, that reality mining have the potential to provide is applicable in Health and Medicine \parencite{pentland2009_reality_mining_health_medicine}, for instance Telehealth \parencite{telehealth_aau}. This area uses telecommunication to improve the general health of citizens. This platform could potentially enable a better understanding of the human psyche in a more implicit matter, and thereby assist in improving the Telehealth field of knowledge. A vision could be to have an application that could replace medicine used in psychiatric areas. Instead of prescribing antidepressants, a patient will get an application installed on their smart devices that will give them tools to handle their mental state in different contexts. The application could ask questions that counters negative thought patterns. Such an application could also be useful for doctors and psychiatrist, since they would have access to continuous monitoring of their patients. The patients would still need consultations, but the doctors do not have to rely solely on the patients memory. In the same way one could imagine applications for people struggling with drugs or alcohol, where a Disulfiram \parencite{nlm_disulfiram} treatment could be replaced or decreased by introducing an application that understands the mental state of alcoholics.
\\\\
Reality mining is not solely a field applicable in health related studies but also fields such as automobile traffic congestion \parencite{pentland2009reality_mining_mobile_communication_gps}, may benefit from learning human dynamics more accurately. A last remark in regards to reality mining is that the data that is supposed to improve the understanding is rather personal and for that reason privacy is a concern with reality mining \parencite{madan2009_reality_mining_privacy}.
%Big data is generally becoming a huge competitive advantage for large companies - \parencite{lavalle2011big}. 
%\todo{Small Data}