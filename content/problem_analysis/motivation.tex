%!TEX root = ../../super_main.tex

\section{Motivation}
\label{sec:motivation}

% mobile devices er blevet mere udbredt blandt folk --> giver mulighed for at indsamle data
% Svært at komme et sted hen (i DK) uden at der er mobile devices i nærheden

\todo[inline, caption={}]{
Stikord til hvad der evt. kan stå:
\begin{itemize}
	\item Flyt noget fra introduktion hertil
	\item Beskriv at smartphones ikke har alle sensorer fx.
		\begin{itemize}
			\item Pulsmåler
			\item Temperaturmåler
		\end{itemize}
\end{itemize}
}

% Nævn fluff om at det er en område under udvikling
% ``ubiquitous''
The area of mobile devices and wearables is rapidly changing. In recent years mobile technologies have had a major growth in consumers. The sales of smartphones have increased in sales by a factor of 11 since 2007 \parencite{statsia_smartphones}. Other mobile technologies, like wearables, have also seen an increase in demand in the last couple of years. FitBit, a company selling smart wristbands, have seen a exponential increase in sales of their device in the past five years \parencite{statsia_fitbit}. New technologies, which have the potential to change the way we think about these ubiquitous devices, emerge and are difficult to ignore. An example of this is the Google Contact Lens \parencite{google_contact_lens} which is able to measure the blood sugar level of the wearer and transmit it wirelessly. Another new technology that has emerged is the ``big.LITTLE'' ARM CPU architecture \parencite{big_little_architecture}, which is a multi-core processor architecture that allows for better adjustments to dynamic computing needs and use less power when possible. It combines the best of power hungry and powerful ARM CPU cores with the best of slow, power efficient ones to achieve overall better performance and power consumption. This architecture allows for battery draining high performance on demand and very power efficient background processing when a device is not used. The development of the Google Contact Lens suggests that new mobile sensor technologies are continuously being made which gives access to additional wearable sensors.
\\\\
The expansion in mobile technologies opens new opportunities for studying and understanding the context around the user more accurately. This means that we can obtain better knowledge regarding the users of a system, and understanding people in general. A common way of getting this type of information right now is by the use of surveys, questionnaires, and interviews. This type of information retrieval can also be used to understand users in their context. However, this type of information retrieval is rather explicit, as opposed to mobile technologies that can measure the environment indirectly using different built in sensors. The issue with this indirect understanding of the context is that users are more complex compared to the discrete values outputted from sensors\todo{Insert source}. One could imagine that a combination of surveys and sensor data could improve the understanding of people and their context. 
\\\\
% Reality Mining has been suggested useful in areas such as
% Reality mining privacy concerns 
Such technologies could help support a field known as Reality Mining \parencite{eagle2006_reality_mining_definition} with new data sources. Reality mining is a special area of data mining, where Machine Intelligence principles are applied to datasets gathered from electronic sensors in everyday scenarios. Reality mining gives researchers and industry an alternative to basing research and models solely on data provided through people, i.e. from public polls, focus groups and questionnaires. Improvements based on reality mining has been suggested in areas such as Health and Medicine \parencite{pentland2009_reality_mining_health_medicine}, automobile traffic congestion \parencite{pentland2009reality_mining_mobile_communication_gps}, and more. Privacy is a concern with Reality Mining \parencite{madan2009_reality_mining_privacy}.      
% Help closing a gap between social scientists and computer scientists.
% Democratization 
% digital oligarchy
% data inequality
\\\\
A related area is the Telehealth field of knowledge \parencite{telehealth_aau}. This area uses telecommunication to improve the general health of citizens. This platform could potentially enable a better understanding of the human psyche in a more implicit matter, and thereby assist in improving the Telehealth field of knowledge. A vision could be to have an application that could replace medicine used in psychiatric areas. Instead of prescribing antidepressants, a patient will get an application installed on their smart devices that will give them tools to handle their mental state in different contexts. The application could ask questions that counters negative thought patterns. Such an application could also be useful for doctors and psychiatrist, since they would have access to continuous monitoring of their patients. The patients would still need consultations, but the doctors do not have to rely solely on the patients memory. In the same way one could imagine applications for people struggling with drugs or alcohol, where a Disulfiram\footnote{For information about Disulfiram see \textcite{nlm_disulfiram}} treatment could be replaced or decreased by introducing an application that understands the mental state of alcoholics.
\\\\
%Big data is generally becoming a huge competitive advantage for large companies - \parencite{lavalle2011big}. 
%\todo{Small Data}