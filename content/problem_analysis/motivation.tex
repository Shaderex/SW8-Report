%!TEX root = ../../super_main.tex

\section{Motivation}
\label{sec:motivation}

% mobile devices er blevet mere udbredt blandt folk --> giver mulighed for at indsamle data
% Svært at komme et sted hen (i DK) uden at der er mobile devices i nærheden

\todo[inline, caption={}]{
Stikord til hvad der evt. kan stå:
\begin{itemize}
	\item Flyt noget fra introduktion hertil
	\item Beskriv at smartphones ikke har alle sensorer fx.
		\begin{itemize}
			\item Pulsmåler
			\item Temperaturmåler
		\end{itemize}
\end{itemize}
}

% Nævn fluff om at det er en område under udvikling
% ``ubiquitous''
The area of mobile devices and wearables is an area in development. New technologies, which have the potential to change the we think about these ubiquitous devices, emerge and are difficult to ignore. An example of this is the Google Contact Lens \parencite{google_contact_lens} which is able to measure the blood sugar level of the wearer and transmit it wirelessly. Another technology that has emerged is the big.LITTLE ARM CPU architecture \parencite{big_little_architecture} which combines the best of large power hungry but powerful ARM CPU cores with the best of smaller power efficient ones to achieve overall better performance and power consumption. This architecture allows for battery draining high performance on demand and very power efficient background processing when a device is not used. 
\\\\
The expansion in mobile technologies opens new opportunities for studying and understanding the context around the user more accurately. This means that we can obtain better knowledge regarding the users of a system, and understanding people in general. A common way of getting this type of information right now is by the use of surveys, questionnaires, and interviews. This type of information retrieval can also be used to understand users in their context. However, this type of information retrieval is rather explicit, as opposed to mobile technologies that can measure the environment indirectly using different built in sensors. The issue with this indirect understanding of the context is that users are more complex compared to the discrete values outputted from sensors\todo{Insert source}. One could imagine that a combination of surveys and sensor data could improve the understanding of people and their context. 
\\\\
% Nævn om nyt hardware/ny teknologi der kommer hele tiden (eg. google contact lens)
% Nævn noget omkring ``big-little arm architecture''
Google Contact Lens suggests new technologies which gives access to new wearable sensors, at least for people with a medical condition where continuous monitoring is beneficial. big.LITTLE suggests more continuous data collection capabilities which to a lesser degree affects the user experience of users by massively draining the battery of a mobile device than previously. 

\todo{Skriv om data inequality}
% Democratization 
% digital oligarchy
% data inequality
Big data is general becoming a huge competitive advantage for large companies - \parencite{lavalle2011big}

\\\\
The goal of the project is to give developers, researchers, etc. a new type of platform where they are able to gather labeled training data from the mobile technologies that a lot of people use. One could imagine that this platform sets the stage for problems to be solved using machine intelligence and statistical analysis. A computer-scientific idea that can be guided by this type of platform is indoor positioning. One could imagine that developers are able to prompt users for additional information in regards to a building that have not yet been mapped accurately enough. After the prompt they can use the wireless radio signals to create a model that can triangulate their position indoor. \todo{Reference Bent?}
\\\\
Another field where this platform could be relevant is the Telehealth field of knowledge \parencite{telehealth_aau}. This area uses telecommunication to improve the general health of citizens. This platform could potentially enable a better understanding of the human psyche in a more implicit matter, and thereby assist in improving the Telehealth field of knowledge. A vision could be to have an application that could replace medicine used in psychiatric areas. Instead of prescribing antidepressants, a patient will get an application installed on their smart devices that will give them tools to handle their mental state in different contexts. The application could ask questions that counters negative thought patterns. Such an application could also be useful for doctors and psychiatrist, since they would have access to continuous monitoring of their patients. The patients would still need consultations, but the doctors do not have to rely solely on the patients memory. In the same way one could imagine applications for people struggling with drugs or alcohol, where a Disulfiram\footnote{For information about Disulfiram see \textcite{nlm_disulfiram}} treatment could be replaced or decreased by introducing an application that understands the mental state of alcoholics.
\\\\
