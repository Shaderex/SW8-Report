%!TEX root = ../../super_main.tex

\section{Motivation}
\label{sec:motivation}

% mobile devices er blevet mere udbredt blandt folk --> giver mulighed for at indsamle data
% Svært at komme et sted hen (i DK) uden at der er mobile devices i nærheden

\todo[inline, caption={}]{
Stikord til hvad der evt. kan stå:
\begin{itemize}
	\item Flyt noget fra introduktion hertil
	\item Beskriv at smartphones ikke har alle sensorer fx.
		\begin{itemize}
			\item Pulsmåler
			\item Temperaturmåler
		\end{itemize}
		\item Nævn fluff om at det er en område under udvikling
		\item Nævn ``ubiquitous''
		\item Nævn om nyt hardware/ny teknologi der kommer hele tiden (eg. google contact lens)
		\item Nævn noget omkring ``big-little arm architecture''
\end{itemize}
}

The goal of the project is to give developers, researchers, etc. a new type of platform where they are able to gather labeled training data from the mobile technologies that a lot of people use. One could imagine that this platform sets the stage for problems to be solved using machine intelligence and statistical analysis. A computer-scientific idea that can be guided by this type of platform is indoor positioning. One could imagine that developers are able to prompt users for additional information in regards to a building that have not yet been mapped accurately enough. After the prompt they can use the wireless radio signals to create a model that can triangulate their position indoor. \todo{Reference Bent?}
\\\\
Another field where this platform could be relevant is the Telehealth field of knowledge \parencite{telehealth_aau}. This area uses telecommunication to improve the general health of citizens. This platform could potentially enable a better understanding of the human psyche in a more implicit matter, and thereby assist in improving the Telehealth field of knowledge. A vision could be to have an application that could replace medicine used in psychiatric areas. Instead of prescribing antidepressants, a patient will get an application installed on their smart devices that will give them tools to handle their mental state in different contexts. The application could ask questions that counters negative thought patters. Such an application could also be useful for doctors and psychiatrist, since they would have access to continuous monitoring of their patients. The patients would still need consultations, but the doctors do not have to rely solely on the patients memory. In the same way one could imagine applications for people struggling with drugs or alcohol, where a Disulfiram\footnote{For information about Disulfiram see \textcite{nlm_disulfiram}} treatment could be replaced or decreased by introducing an application that understands the mental state of alcoholics.
