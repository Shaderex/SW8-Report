%!TEX root = ../../super_main.tex

\section{Legislation}
\label{sec:legislation}

Personal data is defined as information that allows for the identification of a single individual, either directly or indirectly. Directly referring to using unique identifiers such as CPR number, while indirectly could be done by tracking a person's movements and using this to identify the individual. The Danish laws about protecting personal data classifies personal data into different categories: Normal data and sensitive data. Sensitive data is defined as revealing racial or ethnic origin, political opinions, religious or philosophical beliefs, trade-union membership, and data concerning health or sex life \parencite{datatilsynet_stud1}, while normal data refers to everything else, such as name and address.  
\\\\
When located in Denmark one has to notify, and receive permission from, the Danish Data Protection Agency (DDPA), before being allowed to process personal data. When processing data the reason has to be well-founded, and the processed data should be the minimum amount required in order to fulfill the purpose of the data collection. However, special rules apply for university students. The main difference for students is, that the DDPA does not have to be notified, and no permission is needed. The rest of the general laws regarding personal data still hold, such as encrypting data transfers, ensuring limited (password protected) access to the data, and having consent from the users. 
\\\\
The Danish law of protection of personal data is an implementation of the 95/46/EC directive of the European Parliament \parencite{eu_personal_data_law}. There has however been announced new legislation, General Data Protections Regulation (GDPR), in this area which will likely come into effect some time in 2018 \parencite{eu_data_law_changing}. The regulation contains a number of amendments to the previous directive, such as heavy financial penalties of up to 20 million euro or up to 4\% of annual worldwide turnover for groups of companies, whichever is greater, if they are to break the regulation. The danish legislation in the area has not been updated to meet these demands, and therefore we cannot know the exact implementation that we should follow. We have therefore chosen to follow the current Danish legislation for student projects, which as previously mentioned is less restrictive than the general legislation in the area. 
