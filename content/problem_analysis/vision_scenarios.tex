%!TEX root = ../../super_main.tex
\section{Vision Scenarios}
\label{sec:vision_scenarios}
% Essence suggests finding two fundamentally opposing questions that properly represent different directions the project could take. This allows for a clearer insight in the benefits of progressing in either way. The overall goal of this exercise is to explore the design space and to learn more about what we want to create, and eventually end up with a vision for the project. The Essence book has an example project called Psyche where several different opposing questions were found, such as Intervention vs Observation and Citizen Oriented vs Caretaker Oriented. Essence furthermore suggests the four previously mentioned representations for each direction of the two questions, which is typically represented by using the four quadrants in a graph representation. One can start by finding several opposing questions, and then selecting the two most central ones. 
Essence, as mentioned in \secref{sub:essence_vision_scenarios}, a suggests to utilize vision scenarios. We have come up with a list of opposing directions of development, which can be used to explore the different quadrants of a graph with two of the opposites as its axes. The purpose of this is to detail different directions the project could take and thereby learn more about the problem and possibly learn more about how it can be sovled. 


\begin{itemize}[itemsep=0.1em]
    \item Disruptive/Non-disruptive - Should the system be based on passively collected sensor data or interactively prompt with questionnaires? % 1
    \item On demand/Continuous - Should the system be activated by participants or run continuously in the background? % 2
    \item Easy-of-use/Customizable - Should we require that customers must customize campaigns or should it be ``one data collection configuration fits all''? % 3
    \item Assigned/Opt-in - Should campaigns be assigned automatically to participants using a participant profile or should participants actively choose campaigns? % 4
    \item Personal/Anonymous - Should the system require personal information about participants or not? % 5
\end{itemize}

We have chosen to use the Ease-of-use/Customizable and Assigned/Opt-in orientation questions as axes. 
\\\\
% 1
We found it difficult to imagine the usefulness of a system that exclusively collects sensor data for classification problems. This means that at least some data not based on sensor should be collected. The degree of disruptiveness could then perhaps be regulated by the campaign configuration. 
% 2
The system could allow participants to turn the system on/off easily but we think that having the system running continuously once activated would increase the probability of completing campaigns. Actual data collection would also depend on whether there is one or more active campaigns on the participant's device.
% 5
We found that some demographic information would be necessary or interesting for both the scenario where participants are assigned to campaigns and where they opt-in. Customers would likely be interested in whether the collected data represents the wide population or the specific group they are targeting for their data collection project. The degree of personal information required for a given campaign could perhaps then be defined in a campaign configuration. 
\\\\
We have chosen to include Static/Customizable because we thought it would be interesting to explore different ways of allowing customers to configure what data they want and how they want it and because we think that both direction could add value. We have chosen Assigned/Opt-in because these directions would have a large impact on how a solution should be formed. We would have to store a lot of demographic data about participants, thereby possibly impacting privacy, if we want to assign and match campaigns. Both directions would impact how the interaction with participants should be designed and what the server side should do in different ways.

% 1: rå sensor data har begrænset anvendelsesmuligheder --> spørgeskema kan hjælpe med dette.
%       brug for en eller anden grad af labels fra brugere. Behøver ikke komme fra spørgeskemaer

% 2: Skal bruger forstyrres "unødvendigt" eller selv aktivt starte appen op fx. hver dag eller skal den køre i baggrunden
% Fordel ved selv styring er at de kan regulere strøm forbrug og vælge hvilke dag de vil deltage. Ulempe ved at de vælger er at det kan være svært at generalisere over tid. 
% Begge dele kan påvirker bruger entusiasmen/involveringen

% 3: Skal det være nemt/hurtigt at lave campaigns