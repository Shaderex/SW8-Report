%!TEX root = ../../super_main.tex

\subsection{Ethical Considerations}
\label{sub:ethical_considerations}

While the legislative requirements for storing personal data are important, something that is almost equally as important is to look into the ethical aspects of storing possibly sensitive data about people. Both because one should consider how the data is being used, which implications it has to gather and use this data, but also how it can possible deter people from using applications that gather data about them. Professor of IT-ethics at Aalborg University, Thomas Ploug, deemed that personal data is one of the largest ethical dilemmas in the recent and upcoming years.

    % Med et voksende antal private og offentlige registre, hvortil flere og flere får adgang, vokser risikoen for tab af følsomme personoplysninger og misbrug af data. Det er det største etiske dilemma, vi går i møde i 2015.
\quotewithauthorandref{
	With a growing amount of private and public registers, where more and more people get access, the risk of loss and misuse of sensitive personal data increases. That is the biggest ethical dilemma we will face in 2015.
}{Thomas Ploug}{etik_dk_thomas_ploug}{true}

Experts deem that abuse of personal data is becoming an increasing risk, and one must tread carefully when this type of data is stored, transfered, and in general used. As mentioned in \secref{sub:legislation} there exist legislation in Denmark, which should ensure that these issues are being treated with care. However, because the value of personal data increases for malicious parties as well as non malicious parties, the devastation of it being accessible to the wrong people also increases. It should be in everyone's best interest that the personal data is handled with care, because users have no direct control of how the data is accessed or by whom it is accessed. Bill Schmarzo, chief technology officer for EMC Global Services - an influential cloud computing and storage company, is particular concerned with unauthorized parties accessing personal data.

\quotewithauthorandref{
    But the bigger concerns are related to unsanctioned organizations using my data and inferences about my interests, passions, affiliations and associations for borderline uses about my political, religious, sexual, etc. preferences.
}{Bill Schmarzo}{forbes_bill_schmarzo}{false}

Taking both the legislative requirements and our ethical obligations into consideration, a safe and secure infrastructure needs to be established in systems operating with information that can potentially be used to harm the people it was gathered from. We will have to be considerate of the participants, by informing them about how the system uses and processes the data they provide, and obtain their consent before gathering personal data from their ubiquitous devices.
\\\\
%thought to this collection of personal data 
A final observation related to persuading participants to use the application could be, that participants may have varying trust levels towards campaigns depending on what the data will be used for. One could imagine that the participants would be more willing to contribute to surveys conducted for a scientific purpose. On the other hand, participants might be less willing to contribute to commercial surveys, where the purpose is a matter of increasing revenue, by either developing a new product or improving existing products. One could imagine that the participants would have to be further persuaded to contribute to commercial surveys. To respect the participants opinions, the system will have to be transparent in terms of who the participants are generating data for, and what the data is intended for.