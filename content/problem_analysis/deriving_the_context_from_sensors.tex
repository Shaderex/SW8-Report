%!TEX root = ../../super_main.tex

\section{Deriving the Context from Sensors}
\label{sec:deriving_the_context_from_sensors}
This section describes how different sensors may contribute to describing the context in which the participants exist, in as much detail as possible. Note that there might be differences in the quality of the sensors, since not all devises are using the exact same sensor.
% We only consider sensors that are commonly found in mobile devices and smart wearables compatible with Android. 

\subsection{Availability of Sensors}
To discover which sensors are commonly available and how different type of devices can contribute, the specifications of some popular smartphones and wearables have been analyzed. \tabref{tab:sensors_in_devices} gives an overview of which sensors are available from each of these devices. 

%!TEX root = ../../super_main.tex

\begin{sidewaystable}
\centering
\begin{tabular}{|l|c|c|c|c|c|c|c|c|}
\hline
 & Nexus 5 & \begin{tabular}[c]{@{}c@{}}OnePlus \\ One\end{tabular} & \begin{tabular}[c]{@{}c@{}}Samsung \\ Galaxy S3\end{tabular} & \begin{tabular}[c]{@{}c@{}}Samsung \\ Gear S\end{tabular} & \begin{tabular}[c]{@{}c@{}}Microsoft \\ Band 2\end{tabular} & Moto 360 & \begin{tabular}[c]{@{}c@{}}LG Watch \\ Urbane LTE\end{tabular} & \begin{tabular}[c]{@{}c@{}}Huawei\\ Watch\end{tabular} \\ \hline
Accelerometer & \checkmark & \checkmark & \checkmark & \checkmark & \checkmark & \checkmark & \checkmark & \checkmark \\ \hline
\begin{tabular}[c]{@{}l@{}}Ambient\\ Light Sensor\end{tabular} & \checkmark & \checkmark &  & \checkmark & \checkmark &  &  &  \\ \hline
Barometer & \checkmark &  & \checkmark & \checkmark & \checkmark &  & \checkmark & \checkmark \\ \hline
\begin{tabular}[c]{@{}l@{}}Cellular\\ Networking\end{tabular} & \checkmark & \checkmark & \checkmark &  &  &  & \checkmark &  \\ \hline
Compass (Magnetometer) & \checkmark & \checkmark & \checkmark & \checkmark &  & \checkmark & \checkmark &  \\ \hline
\begin{tabular}[c]{@{}l@{}}Galvanic Skin\\ Response Sensor\end{tabular} &  &  &  &  & \checkmark &  &  &  \\ \hline
GPS & \checkmark & \checkmark & \checkmark & \checkmark & \checkmark &  &  &  \\ \hline
Gyroscope & \checkmark & \checkmark & \checkmark & \checkmark & \checkmark & \checkmark &  & \checkmark \\ \hline
\begin{tabular}[c]{@{}l@{}}Optical Heart\\ Rate Sensor\end{tabular} &  &  &  & \checkmark & \checkmark & \checkmark & \checkmark & \checkmark \\ \hline
Proximity Sensor & \checkmark & \checkmark & \checkmark & \checkmark &  &  & \checkmark &  \\ \hline
\begin{tabular}[c]{@{}l@{}}Skin Temperature\\ Sensor\end{tabular} &  &  &  &  & \checkmark &  &  &  \\ \hline
UV Sensor &  &  &  & \checkmark & \checkmark &  &  &  \\ \hline
Wi-Fi & \checkmark & \checkmark & \checkmark & \checkmark &  &  & \checkmark & \checkmark \\ \hline
\end{tabular}
\caption{Available sensors in each mobile device that we have considered.}
\label{tab:sensors_in_devices}
\end{sidewaystable}
\FloatBarrier

\subsection{Continuous Sensors}
A single or few reading form this type of sensor does not give much value to describe a context. However, a continuous stream of readings describes change over time of the context, thus grating value. These sensors are often noisy i.e. the read vary a bit from the expected value even though the device is laying perfectly still.

\begin{description}
	\item[Accelerometer] measures the acceleration of the device in three dimensions, which is also often together with the gyroscope sensor. The accelerometer is able to provide measurements that can describe the motion of the wearer or user along the three physical axes, typically measured in $m/s^2$.
	\item[Gyroscope] measures the speed of rotation of the device in space. The gyroscope will provide an idea of the orientation of the device relative to Earth's gravitational pull. The rotation of the gyroscope is measured on all three axises as rotation velocity in $rad/s$.
	\item[Compass] (also known as magnetometer) is able determine the orientation of the device relative to the Earth's magnetic field. This sensor can, in combination with the accelerometer be used to provide readings like that of a compass. The magnetometer measures the magnetic influence on all threes axises in $\mu T$.
	\item[Barometer] approximates the air pressure, in $hPa$ or $mbar$, around the device. The readings from a barometer can be used to estimate the altitude of the device.
    \item[Optical Heart Rate Sensor] estimates the heart rate, using a light source and a light sensitive sensor. Heart rates are measured in Beats per Minute (BPM). It can be used to detected elevated heart rate levels for instance caused by exercising or heightened levels of stress. Heart rate monitoring can also be used estimate the quality of ones sleep \parencite{guardian_fitness_tracker_rem_sleep}. 
    \item[Skin Temperature Sensor] measures the temperature of skin that touches the device, given in some type of degrees, usually Celcius. This can possibly indicate environmental effects on the body temperature, exercising, fever, etc.
    \item[Galvanic Skin Response Sensor] (GSR) measures the conductivity of the surface against the device. This can indicate whether or not a person is sweating, or more generally the moisture level on a surface. The GSR sensor is commonly used to infer whether a wearable device is being worn at the moment, and its output is measured in ohm ($\Omega$).
\end{description}

\subsection{Reactive Sensors}
Sensors described in this section are reactive meaning that they output measurements when an event from the environment triggers them. The events are typically used to handle some change in the environment rather than measuring the context.

\begin{description}
    \item[Proximity Sensor] can estimate how close the device is to the nearest object facing the sensor by using an infrared beam. This estimation has higher accuracy in lower distances. The distance is measured in $cm$. Some devices, however, does not measure the distance directly but return binary values that represent ``near'' or ``far'' instead of an actual distance, the output is still outputted as either zero or five centimeter. This sensor is commonly used to turn of the display when the device is placed in the pocket or is held against the ear.
    \item[Ambient Light Sensor] makes it possible to measure the illumination surrounding the device in $lx$. This is commonly used to regulate the background light of the device's screen.
    \item[Ultraviolet Sensor] (UV) measures the UV radiation hitting the device. This is useful for estimating whether a person is inside or outside. It can also be used to estimate whether a person has spent too much time outside, and thereby in risk of sunburns or skin cancer, by accumulating readings of UV radiation of an entire day. The output of the UV sensor in the Microsoft Band 2 is given by an enum with the values: none, low, medium, high, and veryhigh, which corresponds to different levels of UV indexes. 
\end{description}

\subsection{On-Demand Sensors}
On-demand sensors are sensors that have do not provide a stream of information but have to be requested whenever a measurement is needed. Measurements from this type of sensors are typically more expensive in terms of battery consumption and computation. An on demand reading can only be given asynchronously as the device would first have to contact multiple information providers such as cell towers and GPS satellites and over time reach a sufficiently accurate reading. To improve efficiency the latest readings are, for example in Android, cached, which allows for quick access the most recent result.

\begin{description}
    \item[Wi-Fi] in mobile devices can be used for more than just pure data communication with the local network and the Internet. Different information about access points and the signal to these access points can for instance be used to approximate indoor location. Features of Wi-Fi networks could be reachable access point names, their communication standards, and signal strength. 
    \item[Global Positioning System] (GPS) approximates the latitude and longitude of the device based on satellite triangulation. Most modern mobile devices use Assisted GPS (A-GPS), meaning that the GPS is assisted by the cellular network. The rough positioning is based on the cellular network and the accurate position is triangulated using GPS satellites. Some systems can also provide estimated information about speed and bearing of the device.
    \item[Cellular Networking] can, similarly to Wi-Fi be used for more than pure communication as data from different mobile towers and the signal strength to these towers can be useful besides communication. Potentially interesting features could be country code, network code, cell identity (mast identity), cell tower technology type, and signal strength.
\end{description}

We have briefly looked at the availability of various sensors in different smart devices, and it generally seems like most devices have many of the same sensors. In relation to our project this means that we can choose to develop a solution for either iPhone, Android or Windows Phone and their related wearables without limiting the available sensors. A context can at the same time be measured by multiple sensory devices in order to increase the accuracy of the data, which suggests combining outputs from multiple of the examined sensory devices. 
\\\\
Some of the sensors furthermore describe overlapping parts of the context, and we have therefore initially chosen not to include cellular networking as a data source directly. There is a lot of data to be found from this sensor, but the features and the layout of the data differs depending on which cellular technology the cell tower supports, which makes it slightly difficult to implement. The location of the user is already indirectly available through A-GPS and cellular positioning, and we have therefore decided not include this data source initially. 
\\\\
There are also other available sensors that we have not yet taken a look at in order to describe contexts, such as bluetooth, screen wake time, use of apps, etc. We will not exclude the possibility of using these at some point during the development, but we think the sensors from \tabref{tab:sensors_in_devices} describe an initial sensor context that we can use in order to build a base product. 