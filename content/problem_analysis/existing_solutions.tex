%!TEX root = ../../super_main.tex

\section{Existing Solution}
\label{sec:existing_solution}

% Gamification?

% Purple Robot
\subsection{Purple Robot}
\label{sub:purple_robot}

Purple Robot is an Android application originally developed to allow for easier or better integration for PhoneGap and Apache Cordova applications with the native functionality on the Android platform. The application allows other non-native applications to trigger status-bar notifications, app widgets, and full native dialogs. 
The application also allows other applications to access all the sensors on a device with an uniform interface. 
The communication for triggering and reading of the sensors is done using a locally running HTTP server that the Purple Robot application runs, meaning that the Purple Robot application actually does not provide a lot of functionality for a user on its own, but allows other applications to have easy access to the integrated native functionality.

% Seems unstable

% This framework could be used to create what we are striving for?

% Sensor Data
\subsection{Sensor Data} 
\label{sub:sensor_data}

Sensor Data is an iPhone application for gathering unlabeled data. The application can use all sensors that are available in iPhone devices in order to gather data, which is then made available through a web API. When gathering data it is possible to store it in two different ways: Capture Mode and Streaming Mode. When using Capture Mode the data is stored directly into the flash memory of the iPhone, while using Streaming Mode will transfer the data to another computer. The data is recorded and stored as comma separated values for easy integration into data analysis software.


\subsection{Mental Health} % (fold)
\label{sub:mental_health}

\begin{itemize}
    \item Research in mental health using an app \footnote{http://www.imedicalapps.com/2014/09/smartphone-sensors-mental-health/}
\end{itemize}

% subsection mental_health (end)




