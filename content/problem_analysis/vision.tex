%!TEX root = ../../super_main.tex
\section{Vision}
\label{sec:vision}

We would like to develop a platform that allows for dynamic collection of data for reality mining. Different area of reality mining require different types of labeled training data. For this reason the platform should allow for configurable campaigns of data collection. A customer should be able to to configure what type of data he wants in the set of training data and how the entries should be labeled. The vision for the platform is that people that are in needed of context aware training data can use our platform instead of developing and distributing their own specialized applications to gather these type of data. We imagine that our platform will handle all technical details in regard of gathering the training data, and the only task that is left for the customers of this platform is to motivate participants, which they would have to do in any case.

\todo[inline]{Find a place for this sentence}

% Vi har ikke nogen udefrakommende kunne, vi vil håndtere dette med et essence vision
This project aims to be innovative and does not have an external customer whom can define and verify requirements. 
\\\\
A software development methodology that aims to support high value software solutions is Essence \parencite{essence_book}. Essence mentions having a vision as a great way to start a project. Essence mentions that having different representations of a vision can help elicit objects, events, and qualities to persue. Four different representation types are suggested, namely: Icon, Prototype, Metaphor, Proposition. We have chosen to attempt to represent the existing condition, i.e. the problem area as we understood it at the time, as Icons, Metaphors, and Propositions. We found the prototype representation too costly in terms of development time. 