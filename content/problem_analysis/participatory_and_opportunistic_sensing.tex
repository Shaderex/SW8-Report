%!TEX root = ../../super_main.tex

% This section is based on: http://ieeexplore.ieee.org/stamp/stamp.jsp?tp=&arnumber=6365160

\section{Participatory and Opportunistic Sensing}
\label{sec:participatory_and_opportunistic_sensing}

\todo{Write section about different approaches to getting labels - 	}

- Human-centric sensing

We would like to gather training data, i.e. human activity recognition (HAR) including and combined with data from mobile devices. We could also include external sensors such as cameras or smart home sensors like movement detectors or temperature sensors in the room. 

% Hvorfor bruger vi ikke external sensors
External sensors 

% Folk har ikke lyst til at blive overvåget
Privacy issues with cameras 

% Der er meget data i videoer - Feature extraction difficult
Camera feature extraction is a problem



There are at least two overall different approaches to this namely Participatory and Opportunistic sensing. Participatory sensing requires that a user actively participates in the data collection either by directly entering data or otherwise react to requests and opportunistic sensing does not require any user involvement.



- Important distinction between external and wearable sensors 
- Intelligent homes and cameras could provide (external) data 

The Opportunistic Sensing should provide context for the information   

% Skriv noget om at det er meget begrænset hvad man kan "lære" hvis der kun er sensor data 
% Opp

% Skriv noget om at questionaires er en god måde at skaffe data (labels) fra users  