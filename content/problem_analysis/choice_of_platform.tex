%!TEX root = ../../super_main.tex

\section{Choice of target Platform}
\label{sec:choice_of_platform}

The system should give its customers access to a as much training data as possible. Therefore we would like to reach as large a user base as possible for the gathering of the training data. This means that developing multi-platform mobile application would be ideal. 
% There exists many tools to do this *nævn dem*
% Hvad er problemet med de tools?
% Hvorfor har vi valgt android?

% Introduction
% Vi kender det fra tidligere og målet proof of concept
A part of our system will have to be developed to one mobile platform and as we are limited by time. We have chosen not to  The overall goal of this project is to develop upon an idea and show that this idea can be implemented, we ha. 
\\\\
% Android er mest udbredte platform og de fleste wearables er nød til at være kompatible til
% Flest wearables der er kompatible med android
% Android har et helt økosystem til wearables, mens de 2 andre store kun har få til deres eget produceret ting
Many wearables are depended on communication with a smartphone and a lot of wearables are compatible with Android. This is probably partially due to the fact that Android is the most used smartphone operating system \parencite{android_os_market_share} and partially because Android is also a an open wearable platform through Android Wear \footnote{https://www.android.com/wear/}.
\\\\
% Vi har telefon til at teste på
% gratis og open-source/open-development environment
We have Android devices available as we have some ourself and as we can borrow Android devices from the university. Android SDK and IDE, called Android Studio \footnote{http://developer.android.com/sdk/index.html}, is free and easy to setup and run on multiple platforms contrary to the Apple iOS development environment called Xcode \footnote{https://developer.apple.com/xcode/download/} which only runs on mac with OS X. We have not considered the Windows phone Operating system as a target platform because of its limited market share.
\\\\
We have therefore decided to develop the application, which should be run on the devices of participants, to run on the Android platform. On this platform there exist something called API levels which determines what version of the operating system that our application will be compatible with, we have chosen to support the API level 21, which is one of the most recent versions of the Android platform that is compatible with wearables. However, we had some issues in regards to this choice, both in terms of development method and some technical detailes regarding hardware and supported devices. A description of these issues and some reflection upon these can be found in \secref{sec:the_android_platform_compatibility_and_test_driven_development}.
