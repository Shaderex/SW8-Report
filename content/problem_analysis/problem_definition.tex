%!TEX root = ../../super_main.tex

\section{Problem Definition} 
\label{sec:problem_definition}
% Opsumer analyse
% Legal stuff, Encryption stuff, ethics stuff
There are, as described in \secref{sec:personal_data}, different types of data with different privacy requirements. Some of these requirements apply to this project and the data we intend to collect. We must therefore take measures to accommodate these requirements.
\\\\
% Health care stuff requires real time - Impacts battery and
% Does not allow dynamic use of sensors from other mobile devices than smartphones
Available sensor gathering applications and accompanying platforms generally have real time goals, health care systems often have intervention goals which requires data to be synchronized with a central node in near real time. The existing solutions we have looked at do not incorporate more devices than smartphones. Smartphones do not, in general, have all the health monitoring sensors of for instance a smartwatch or a smart band, which, opposed to a smartphone, are directly in contact with the body at most times. One could for instance imagine that data from these sensors to be interesting in various health related classification problems. 
\\\\
% Participatory sensing and training data
Gathered sensor data, does not necessarily provide much value without being coupled with a meaningful label, which could for instance be the mental state of a participant through participatory sensing. It would not be possible for us to predict anything about human dynamics that is not already measurable by sensors, without the labeling from the participants. If we only had data from e.g. an accelerometer, a gyroscope and no label, it would only be possible for us to say something about the correlation between the accelerometer readings and the gyroscope readings. However, if we included a label describing if the person was under the influence of alcohol at the time of the reading, we would be able to say something about the correlation between the readings of these sensors and the state of being influenced by alcohol. 

\subsection{Term definitions}
Before we define our problem, we establish some terms that will help us to define some concepts in this problem domain. 

\begin{description}
    \item [Customers and Participants] are two types of actors that we see in this problem area. A customer is one who are in need of labeled training data, while a participant is one who will provide the data a customer need. 
\end{description}

\begin{description}
    \item[Campaign] is a term we borrow from AndWellness \parencite{hicks2010andwellness}, which is a concept that describes the configuration of triggers, sensor data collection, and participant input necessary to label the collected data. An analogue for a campaign could be a survey, where the questions of the questionnaire, or interview, is a campaigns specification defining what sensors the customer wants to get measurements from. 
\end{description}

\begin{description}
    \item[Snapshot] is our concept of one collection of sensor readings from the sensors specified in a campaign, that discretely captures the context. Furthermore a snapshot contains a label which is derived explicitly from the participants. Note that a label is some description of the reality in which the participant exists, which could be derived by having participants answer a series of questions. In the analogue where a campaign is a survey, snapshots would be answers to this special kind of context aware survey. In other words, a snapshot will be an element in a set of labeled training data.
\end{description}

\subsection{Problem Statement}
\label{sub:problem_statement}
The following problem statement and supporting subquestions are based on the prior analysis.
\\\\
%!TEX root = ../../super_main.tex

% Hovedspørgsmål 

\textbf{What characterizes a system which allows customers to specify campaigns, and allows distributed gathering of snapshots from participants to contribute to these campaigns, and how could such a system be realized?}

% Underspørsmål
\begin{itemize}
	\setlength\itemsep{-0.2em}
    \item How can we facilitate distributed collection of snapshots from ubiquitous devices, with sensors that may or may not be available at a given time?

    \item How can we acquire labels, from participants, for the gathered snapshots?  
    
    \item How can we allow customers to define a campaign for their participants?
    
    \item How can we minimize the effect which the mobile application has on the part of the participants' daily life that involves the use of smart devices?

    \item How can we uphold the requirements specified in the privacy legislation and protect participants' sensitive personal data, while still providing customers with detailed snapshots from these participants? 
\end{itemize}



