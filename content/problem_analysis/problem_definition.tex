%!TEX root = ../../super_main.tex

\section{Problem Definition} 
\label{sec:problem_definition}
% Opsumer analyse
% Legal stuff, Encryption stuff, ethics stuff
\todo[inline]{Consider if these paragraphs are too short}
There are, as described in \secref{sub:legislation}, different types of data with different privacy requirements. Some of these requirements apply to this project and the data we intend to collect. We must therefore take measures to accommodate these requirements.    
\\\\
% Health care stuff requires real time - Impacts battery
Available sensor gathering applications and accompanying platforms generally have real time goals, health care systems often have intervention goals which requires data to be synchronized with a central node in near real time. 
\\\\
% Does not allow dynamic use of sensors from other mobile devices than smartphones
The existing solutions we have looked at does not incorporate more devices than smartphones which in general does not have all the health monitoring sensor of for instance a smart watch or a smart band, which opposed to a smartphones are worn close to the body. One could for instance imagine that data from these sensors to be interesting in various health related classification problems. 
\\\\
% Participatory sensing and training data
The context of monitored ubiquitous devices, i.e. gathered sensor data, does not necessarily provide much value without being coupled with a meaningful label, which could for instance be the mental state of a participant through participatory sensing. The sensor data values, by them self, can only be used to train a model which predicts the behavior related to a subset of the sensors without the extra labels from participants. \todo{Måske en mere vag formulering eller kilde her?}

\subsection{Term definitions}
Before we define our problem, we establish some terms that will help us to define some concepts in this problem domain. We have a concept of two types of actors, customers and participants. A customer is one who are in need of labeled training data, while a participant is one who will provide the data a customer need. We borrow the terminology from AndWellness \parencite{hicks2010andwellness} and define a campaign to be all the configuration of triggers, sensor data collection, and label questionnaires necessary for a session of data collection from participants. An analogue for a campaign could be a survey, where the questions of the questionnaire, or interview, is a campaigns specification defining what sensors the customer wants to get measurements from. The result of such a campaign will then be the answers to this special survey. We call these results snapshots, which are measurements from the sensors specified and some label derived explicitly from the participants. Note that a label is simple some description of the reality in which the participant exist, which could be a series of questions.

\subsection{Problem Statement}
\label{sub:problem_statement}
\todo[inline]{Rewrite problem statement and so on to use the new terms.}
The following problem statement and supporting subquestions are based on the prior analysis.
\\\\

%!TEX root = ../../super_main.tex

% Hovedspørgsmål 

\textbf{What characterizes a system which allows customers to specify campaigns, and allows distributed gathering of snapshots from participants to contribute to these campaigns, and how could such a system be realized?}

% Underspørsmål
\begin{itemize}
	\setlength\itemsep{-0.2em}
    \item How can we facilitate distributed collection of snapshots from ubiquitous devices, with sensors that may or may not be available at a given time?

    \item How can we acquire labels, from participants, for the gathered snapshots?  
    
    \item How can we allow customers to define a campaign for their participants?
    
    \item How can we minimize the effect which the mobile application has on the part of the participants' daily life that involves the use of smart devices?

    \item How can we uphold the requirements specified in the privacy legislation and protect participants' sensitive personal data, while still providing customers with detailed snapshots from these participants? 
\end{itemize}



