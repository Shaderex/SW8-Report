%!TEX root = ../../super_main.tex

\section{Reality Mining}
\label{sec:reality_mining}
The expansion of mobile technologies opens new opportunities for studying and understanding the users' context better. By understanding the context of users, we can hope to obtain better knowledge about users and generally better understandings of people through their behavior. A common practice for identifying the context around users is to ask them explicitly with surveys or similar techniques. A field of research where a less biased description of the context would be appreciated is \emph{reality mining}, which is the study of environmental data that is related to human social behavior \parencite{madan2009_reality_mining_privacy}. Reality mining is a branch of data mining, where machine intelligence principles are applied to machine-gathered labeled training data. Labeled training data is here understood as a collection of data, often called features, attached with some label, where the ``training'' part does \underline{not} refer to actual physical training, but instead to the fact that it can be used to train machine intelligence models. In a machine intelligence setting, the features could be the different types of sensor readings, and the label could be determined by some user input as seen in \tabref{tab:labeled_training_data}.
\\
\begin{table}[!htbp]
    \centering
    \resizebox{\textwidth}{!}{%
    \begin{tabular}{|c|c|c|c|}
        \hline
         \textbf{Accelerometer:} & \textbf{Optical heart rate:} & \textbf{GPS:} & \textbf{User input:} \\
        \textit{(x, y, z)}  & \textit{(heartrate)}  & \textit{(longitude, latitude)}   & \textit{\begin{tabular}[c]{@{}c@{}}Are you influenced \\ by alcohol?\end{tabular}} \\ \hline
        \mono{10.23, 00.33, 02.44} & \mono{74}    & \mono{59.22, 31.22}    & \mono{no}   \\ \hline
        \mono{22.23, 13.38, 04.24} & \mono{86}    & \mono{58.44, 32.44}    & \mono{yes}  \\ \hline
        \multicolumn{4}{|c|}{\vdots}    \\ \hline
        \mono{09.82, 00.11, 01.22} & \mono{76}    & \mono{59.21, 31.22}    & \mono{no}   \\ \hline
    \end{tabular}%
    }
    \caption{Example of labeled training data.}
    \label{tab:labeled_training_data}
\end{table}

Reality mining gives research and industry an alternative to basing research and models on data provided exclusively through people, i.e. from public polls, focus groups, and questionnaires. A study from 2006 \parencite{eagle2006_reality_mining_definition} showed, that context could be derived from people and their behavior, which can be used to understand human dynamics with less bias. The study was conducted on 100 subjects with mobile phones, and showed promising results. The study furthermore concluded, that data from new types of sensors could have a lot of potential in regards to different fields based upon reality mining. 
\\\\
Reality mining is, generally speaking, reliant on some context where the user exists and some definition, e.g. which sensor based features and labels, that can be used to describe the context. An observation, which can then be made, is that interested parties, who need this type of information, need to define how they want to model the users' context, and at which granularity. Simply collecting all available data at a maximum rate will most likely be difficult to extract meaning from, due to the sheer amount of data, and might thus not be very useful. Collecting all available data would quickly add up in terms of space required to the represent all the captured contexts, and it might therefore be difficult or impractical to collect enough distinct data sets to derive any patterns.
\\\\
The variety of potentially interesting data and the amount of people that might need some specific data set, means that every interested party often has to define their own reality mining procedure, and find their own mining subjects. It would therefore be interesting to look at how the process can be generalized and streamlined, such that it will be easier to obtain the desired training data, and by that, allow various types of interested parties, such as scientist, companies, or students, to acquire the training data they desire.
\\\\
The understanding of human dynamics, that reality mining has the potential to provide, is applicable in various fields of study, for example Health and Medicine \parencite{pentland2009_reality_mining_health_medicine} which includes Telehealth \parencite{telehealth_aau}. Telehealth uses telecommunication to improve the general health of citizens. Reality mining could potentially enable a better understanding of the human psyche in a less intrusive way, and thereby assist in improving the telehealth field of knowledge. One could imagine that a software solution, that applies knowledge from telehealth and reality mining, could reduce the need of medicine used in psychiatry. Instead of prescribing antidepressants, a patient could install an application on their smart devices, which will give them tools to handle their mental state. This could for instance be attempting to interrupt a situation before a panic attack occurs.
\\\\
In the same way, one could imagine mobile applications for people struggling with drugs or alcohol, where a Disulfiram treatment \parencite{nlm_disulfiram} could be replaced or shortened, by introducing an application that understands the mental state of alcoholics. Reality mining is not solely a field applicable in health related studies, it could also be applicable to fields such as automobile traffic congestion \parencite{pentland2009reality_mining_mobile_communication_gps}, which could benefit from learning human dynamics more accurately.
\\\\
We initially wanted to investigate if these types of sensor data could be used to derive some context around the users of such ubiquitous devices. We imagined that, given some labeled training data, we would be able to implement a machine intelligence model in an application, that would guard users from using their phone in a regretful way under the influence of alcohol. However we quickly realized that this type of data would be close to impossible to find readily available for us to use. Even if someone has already gathered data that is specific enough for the issue at hand, it is not guaranteed that they have published it for everybody to use.