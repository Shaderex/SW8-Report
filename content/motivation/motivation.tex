%!TEX root = ../../super_main.tex

\chapter{Motivation}
\label{cha:motivation}
% mobile devices er blevet mere udbredt blandt folk --> giver mulighed for at indsamle data
% Svært at komme et sted hen (i DK) uden at der er mobile devices i nærheden
% Nævn fluff om at det er en område under udvikling
% ``ubiquitous''

The area of ubiquitous computing is rapidly changing, and especially smart mobile technologies have in recent years had a major growth in the number of users and variation of devices. Our definition of ubiquitous devices includes, but is not limited to, smart phones, smartwatches, and smartbands. We use the word ubiquitous because there exist so many different kinds of smart devices and they are becoming increasingly more available and common over time. The sales of smartphones have increased by a factor of 11 since 2007 \parencite{statsia_smartphones}. Other mobile technologies, like wearables, have also seen an increase sales in the last couple of years. FitBit, a company selling smart wristbands, have seen a exponential increase in sales of their devices in the past five years \parencite{statsia_fitbit}.
\\\\
There are many emerging technologies related to ubiquitous computing that are difficult to ignore, due to their potential to change the way we think about mobile devices and mobile technology in general. An example of this is the Google Contact Lens \parencite{google_contact_lens}, which is able to measure the blood sugar level of the wearer and transmit it wirelessly. The lens is supposed to improve the quality of life for diabetics, as they no longer have to use needles. Another new technology that has emerged is the ``big.LITTLE'' ARM CPU architecture \parencite{big_little_architecture}, which is a multi-core processor architecture that allows for better adjustments to dynamic computing needs by using less power whenever possible. It combines the best of powerful and battery draining ARM CPU cores with the best of slow, power efficient ones to achieve overall better performance and less power consumption. This architecture allows for battery draining high performance on demand and very power efficient background processing when a device is sleeping and not used actively. This is supposed to be a huge battery-improvement for mobile devices, which implies the introduction of a colossal amount of other technologies that benefit from increased battery lifetime and increased background processing capability. 
\\\\
New mobile sensor technologies are being introduced frequently, which could indicate increased access to additional wearable sensors. Some of these new sensors cannot be implemented in the smartphones of today, because some information can only be measured using sensors located close to the body. Examples of this could be the Google Contact Lens' ability to measure blood sugar, or heart rate monitoring from smart bands or smart earplugs. The sensors available from the same type of smart device vary and even a given sensor which is considered common in, for instance, smartwatches is not guaranteed to be present in all new devices. Some manufactures have value editions of their devices, where a low price is preferred over many sensors. For instance Motorola sells both Moto E (a value edition) and Nexus 6 (a flagship model) where the price and specification varies \parencite{moto_e_compared_to_nexus_6}

%!TEX root = ../../super_main.tex

\section{Reality Mining}
\label{sec:reality_mining}
The expansion of mobile technologies opens new opportunities for studying and understanding the context around the users more accurately. This means that we can obtain better knowledge about the users of a system and better understanding people through their behavior in general. A common practice for identifying the context around users is to ask them explicitly with surveys or similar techniques. A field of research where a less biased description of the context would be appreciated is \emph{reality mining}, which is the study of environmental data that is related to human social behavior \parencite{madan2009_reality_mining_privacy}. Reality mining is a branch of data mining, where machine intelligence principles are applied to machine-gathered labeled training data. Labeled training data is here understood as a collection of data (features or values) attached with some label, where the ``training'' part does \underline{not} refer to actual physical training, but instead to the fact that it can be used to train a machine intelligence model. In a reality mining context, features and values would be readings from sensors and the label could be determined by some user input as seen in \figref{tab:labeled_training_data}.

\begin{table}
	\centering
	\begin{tabular}{|m{0.25\textwidth}|m{0.25\textwidth}|m{0.20\textwidth}|m{0.17\textwidth}|}
		\hline
		\textbf{Accelerometer:}       & \textbf{GPS:}           & \textbf{Other sensors:} & \textbf{User input:} \\ 
		\emph{(x, y, z)} & \begin{tabular}[c]{@{}l@{}} \emph{(longitude, latitude)} \end{tabular} & \centering \ldots & \emph{Are you influenced by alcohol?} \\ \hline
		\begin{tabular}[c]{@{}l@{}} \mono{10.23},\\ \mono{00.33},\\ \mono{02.44} \end{tabular} & \mono{59.22, 31.22}  & \centering \ldots & \begin{center} \mono{no}  \end{center}	\\ \hline
		\begin{tabular}[c]{@{}l@{}} \mono{22.23},\\ \mono{13.38},\\ \mono{04.24} \end{tabular} & \mono{58.44, 32.44}  & \centering \ldots & \begin{center} \mono{yes} \end{center}	\\ \hline
		\multicolumn{4}{|c|}{\vdots} 																																					\\ \hline
		\begin{tabular}[c]{@{}l@{}} \mono{09.82},\\ \mono{00.11},\\ \mono{01.22} \end{tabular} & \mono{59.21, 31.22}  & \centering \ldots & \begin{center} \mono{no}  \end{center}	\\ \hline
	\end{tabular}
\caption{Example of labeled training data}
\label{tab:labeled_training_data}
\end{table}
\FloatBarrier

Reality mining gives research and industry an alternative to basing research and models solely on data provided exclusively through people, i.e. from public polls, focus groups, and questionnaires. In order to understand human dynamics with less bias, a study from 2006 \parencite{eagle2006_reality_mining_definition} showed that context could be derived from people and their behavior. The study was conducted on 100 subjects with mobile phones, and showed promising results. The study furthermore concluded that data from new types of sensors could have a lot of potential in regards to different fields based upon reality mining. 
\\\\
The understanding of human dynamics, that reality mining has the potential to provide, is applicable in various fields of study, for example Health and Medicine \parencite{pentland2009_reality_mining_health_medicine} which includes Telehealth \parencite{telehealth_aau}. Telehealth uses telecommunication to improve the general health of citizens. Reality mining could potentially enable a better understanding of the human psyche in a more implicit and less intrusive way, and thereby assist in improving the telehealth field of knowledge. One could imagine that an application that uses this field of research could reduce the need of medicine used in psychiatry. Instead of prescribing antidepressants, a patient could install an application on their smart devices which will give them tools to handle their mental state, and for instance attempt to interrupt a situation before, for instance, a panic attack occurs, in different contexts. 
\\\\
In the same way one could imagine applications for people struggling with drugs or alcohol, where, for instance, a Disulfiram \parencite{nlm_disulfiram} treatment could be replaced or shortened by introducing an application that understands the mental state of alcoholics. Reality mining is not solely a field applicable in health related studies, but could also be applicable to fields such as automobile traffic congestion \parencite{pentland2009reality_mining_mobile_communication_gps} which benefit from learning human dynamics more accurately.
\\\\
Reality mining is, generally speaking, reliant on some context where the user exists and some definition of what this context describes. Because of the huge variety of different contexts and definitions that can be ``mined'', interested parties in need of this type of information need to define what kind of context they desire by specifying what and how they want the labeled training data to be gathered. This is an issue, because that means that every interested party has to define his own reality mining procedure, find their own mining subjects, etc. It would therefore be interesting to look at how the process can be generalized and streamlined, such that it will be easier to obtain the desired training data, and by that, allow various types of interested parties, such as scientist, companies or students, to acquire the training data they desire.
\\\\
Initially we wanted to investigate if these types of sensor data could be used to derive some context around the users of such ubiquitous devices. We imagined that, given some labeled training data, we would be able to implement a machine intelligence model in an application, that would guard users from using their phone in a regretful way under the influence of alcohol. However we quickly realized that this type of data would be close to impossible to find readily available for us to use. Even if someone has already gathered data that is specific enough for the issue at hand, it is not guaranteed that they have published it for everybody to use.
