%!TEX root = ../../super_main.tex

\section{Essence}
\label{sec:essence}
% A software development methodology that aims to support high value software solutions is Essence \parencite{essence_book}. Essence mentions having a vision as a great way to start a project. Essence mentions that having different representations of a vision can help elicit objects, events, and qualities to persue. Four different representation types are suggested, namely: Icon, Prototype, Metaphor, Proposition. We have chosen to attempt to represent the existing condition, i.e. the problem area as we understood it at the time, as Icons, Metaphors, and Propositions. We found the prototype representation too costly in terms of development time. 
In this project we will try to employ some of our development time on using the Essence methodology that is currently being created by lecturer at Aalborg University, Ivan Aaen. Essence is a methodology that helps agile software development teams create innovative and sustainable solutions. The methodology focuses on the process after getting a basic idea of what the desired product is, in order to decide which direction the development should proceed in. Generally speaking, it focuses on translating ideas into solutions, and coming to a clarifying understanding of which challenges the team can encounter and how they can he resolved.  

\subsection{Start Configuration}
\label{sub:essence_start_configuration}

Essence describes a start configuration as a basis of software tools, platforms, libraries and system upon which both a problem and a corresponding solution can mature. This is supposed to provide insight in the problem area, such that further development on the system can be based on educated decisions. 

\subsection{Vision Scenarios}
\label{sub:essence_vision_scenarios}

The overall goal of creating vision scenarios is to explore the different directions there could be taken in order to solve the problem at hand. In order to create vision scenarios Essence suggests finding opposing lines of development, such as if a system should be intrusive or non-intrusive. Two of such opposites (the most important) are then used as the axes of a two-dimensional graph. Each quadrant in the graph will then contain different visions with varying focus for a solution to the same problem, vision scenarios.

\subsection{Vision Representations}
\label{sub:vision_representations}

The four quadrants chosen in the process of finding vision scenarios can then be used to discover different ways of describing the system in different ways, such as using icons, metaphors, prototypes or propositions. If one or more of such descriptions or representations is created for each quadrant it can be used as a catalyst for the discussion of the project focus and qualities of a potential solution. 

\subsection{Variants of Software Innovation}
\label{sub:essence_innovation_variants}

Essence describes four different ways that a system can be innovative. It can be innovative in the way that it will introduce a new or radically changed software-intensive solution to a problem. This is called product innovation. Another of the innovation types is process innovation, where the software-intensive system should offer a new or improved way of producing other solutions. The third type of innovation is project innovation, where it is about utilizing existing solutions in another application domain than it was initially intended. The last innovation variant is called paradigm innovation. This form of innovation is when a software intensive solution is changing the mindset of what users, an organizations or a market is. These innovation variants are not mutually exclusive meaning that a solution can be both paradigm and process innovation. Considering how a problem can be solved in the context of these different innovation variants can help explore different ways of thinking of the problem, and help discover new potential solutions to the problem.

\todo{Overvej at nævne flere Essence ting her hvis vi bruger dem}
\todo[inline]{Write how we combine Essence with XP somewhere}