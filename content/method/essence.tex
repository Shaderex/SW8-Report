%!TEX root = ../../super_main.tex

\section{Essence}
\label{sec:essence}
% A software development methodology that aims to support high value software solutions is Essence \parencite{essence_book}. Essence mentions having a vision as a great way to start a project. Essence mentions that having different representations of a vision can help elicit objects, events, and qualities to persue. Four different representation types are suggested, namely: Icon, Prototype, Metaphor, Proposition. We have chosen to attempt to represent the existing condition, i.e. the problem area as we understood it at the time, as Icons, Metaphors, and Propositions. We found the prototype representation too costly in terms of development time. 

% Vi har ikke nogen udefrakommende kunne, vi vil håndtere dette med et essence vision
This project aims to be innovative and does not have an external customer whom can define and verify requirements. In this project we will dedicate some of our development time to employing elements from the Essence methodology that is created by and currently being revised by a lecturer at Aalborg University, Ivan Aaen. Essence is a methodology that supports software development teams in creating innovative and high value software solutions. Generally speaking, it focuses on translating great ideas into great solutions. In this section we will define some of the concepts that we utilize from the Essence methodology and give insight in how these can help us in our development.



% \subsection{Start Configuration}
% \label{sub:essence_start_configuration}

% Essence describes a start configuration as a basis of software tools, platforms, libraries and system upon which both a problem and a corresponding solution can mature. This is supposed to provide insight in the problem area, such that further development on the system can be based on educated decisions. 

\subsection{Vision Scenarios}
\label{sub:essence_vision_scenarios}

Essence introduces a concept of vision scenarios with the overall goal of exploring different directions which could be taken in order to solve the problem at hand. In order to create vision scenarios Essence suggests finding opposing lines of development, this could for instance be if a system should be intrusive or non-intrusive. Two of such opposites (the most important) can then be used as the axes of a two-dimensional graph. Each quadrant in the graph then corresponds to different approaches for a solution to the same problem. These different quadrants are called vision scenarios.

\subsection{Vision Representations}
\label{sub:vision_representations}

The four quadrants chosen in the process of finding vision scenarios can then be used to discover different ways of describing the system in different ways, such as using icons, metaphors, prototypes or propositions. If one or more of such descriptions or representations are created for each quadrant it can be used as a catalyst for the discussion of the project focus and qualities of a potential solution. 

\subsection{Variants of Software Innovation}
\label{sub:essence_innovation_variants}

Essence describes four different ways that a system can be innovative. It can be innovative in the way that it will introduce a new or radically changed software-intensive solution to a problem. This is called product innovation. Another of the innovation types is process innovation, where the software-intensive system should offer a new or improved way of producing other solutions. The third type of innovation is project innovation, where it is about utilizing existing solutions in another application domain than it was initially intended. The last innovation variant is called paradigm innovation. This form of innovation is when a software intensive solution is changing the mindset of what users, an organizations or a market is. These innovation variants are not mutually exclusive meaning that a solution can be both paradigm and process innovation. Considering how a problem can be solved in terms of these different innovation variants can help explore different ways of thinking about the problem, and help discover new potential solutions to the problem.

\todo[inline]{Beskriv flere Essence ting her hvis vi bruger dem}

\subsection{Combination with Extreme Programming}
\label{sub:combination_with_extreme_programming}
Essence is a useful methodology for keeping a development team on the same page at all times in during software development. We deem to review our vision and progress iteratively through out the project. We do not believe we are able to do a full revision of the project between each iteration, so a true revision will be used after each release through out the development, and we hope that our vision for the project will be more stable using essence in combination with XP.

