%!TEX root = ../../super_main.tex

\section{Essence}
\label{sec:essence}
% A software development methodology that aims to support high value software solutions is Essence \parencite{essence_book}. Essence mentions having a vision as a great way to start a project. Essence mentions that having different representations of a vision can help elicit objects, events, and qualities to persue. Four different representation types are suggested, namely: Icon, Prototype, Metaphor, Proposition. We have chosen to attempt to represent the existing condition, i.e. the problem area as we understood it at the time, as Icons, Metaphors, and Propositions. We found the prototype representation too costly in terms of development time. 

% Vi har ikke nogen udefrakommende kunne, vi vil håndtere dette med et essence vision
The aim of the project was to be innovative without having an external customer who could define and verify requirements. In this project we dedicated some of our development time to employing elements from the Essence methodology, which has been created by, and is currently being revised by, Ivan Aaen, a lector at Aalborg University. Essence is a methodology that supports software development teams in creating innovative and high value software solutions. It focuses more generally on translating great ideas into great solutions, while dealing with insecurities during the exploration of possible solutions. In this section we define some of the concepts that we utilized from the Essence methodology and give insight in how these helped us in our development.



% \subsection{Start Configuration}
% \label{sub:essence_start_configuration}

% Essence describes a start configuration as a basis of software tools, platforms, libraries and system upon which both a problem and a corresponding solution can mature. This is supposed to provide insight in the problem area, such that further development on the system can be based on educated decisions. 

\subsection{Vision Scenarios}
\label{sub:essence_vision_scenarios}

Essence introduces a concept of vision scenarios with the overall goal of exploring different directions that could be taken in order to solve the problem at hand. In order to create vision scenarios, Essence suggests finding opposing lines of development. This could for instance be whether a system should be intrusive or non-intrusive. The two most important pairs of opposites can then be used as the axes of a two-dimensional quadrant system. Each quadrant in the system then corresponds to different approaches for a solution to the same problem. These different quadrants are called vision scenarios.

\subsection{Vision Representations}
\label{sub:vision_representations}

The four vision scenarios, represented by the quadrants, can then be used to discover different ways of describing the system, such as using icons, metaphors, prototypes or propositions. If one or more of such descriptions or representations are created for each quadrant, they can be used as catalysts for the discussion of the project focus and qualities of a potential solution. 

%\subsection{Variants of Software Innovation}
%\label{sub:essence_innovation_variants}

%Essence describes four different ways that a system can be innovative. It can be innovative in the way that it will introduce a new or radically changed software-intensive solution to a problem. This is called product innovation. Another of the innovation types is process innovation, where the software-intensive system should offer a new or improved way of producing other solutions. The third type of innovation is project innovation, where it is about utilizing existing solutions in another application domain than it was initially intended. The last innovation variant is called paradigm innovation. This form of innovation is when a software intensive solution is changing the mindset of what users, an organizations or a market is. These innovation variants are not mutually exclusive meaning that a solution can be both paradigm and process innovation. Considering how a problem can be solved in terms of these different innovation variants can help explore different ways of thinking about the problem, and help discover new potential solutions to the problem.

\subsection{Configuration Table}
Essence provides a tool called \emph{Configuration Table}\parencite{essence_book}, which supports both team members and stakeholders with useful reflection through the development of innovative projects. A configuration table should always reflect the state of some project, especially when team members gain new knowledge, go in alternative directions, or the general setting for the project changes.
\\\\
A configuration table is split into four columns: \emph{paradigm}, \emph{product}, \emph{project}, and \emph{process}; and three rows: \emph{rationale}, \emph{strategy}, and \emph{tactics}. Combinations of these aspects give insight in different areas of the project. Throughout the project, different cells will be updated, causing cascades in the table, further allowing the team to properly consider how to be innovative. 

\subsection{Combination with Extreme Programming}
\label{sub:combination_with_extreme_programming}
Essence is a useful methodology for keeping a development team on the same page at all times during software development, by creating a vision for the project and maintaining it iteratively throughout the development. We aimed to review our vision and progress iteratively throughout the project, but we did not believe that we would be able to perform a full revision of the project between each iteration. Instead, we decided to perform a revision after each release, and we thereby hoped that our vision for the project would be more stable, using Essence in combination with XP.