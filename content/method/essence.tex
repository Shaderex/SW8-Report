%!TEX root = ../../super_main.tex

\section{Essence}
\label{sec:essence}

In this project we will try to employ some of our development time on using the Essence methodology that is currently being created by lecturer at Aalborg University Ivan Aaen. Essence is a methodology that helps agile software development teams create innovative and sustainable solutions. The methodology focuses on the process after getting a basic idea of what the desired product is, in order to decide which direction the development should proceed in. Generally speaking, it focuses on translating ideas into solutions, and coming to a clarifying understanding of which challenges the team can encounter and how they can he resolved.  

\subsection{Start Configuration}
\label{sub:essence_start_configuration}

Essence describes a start configuration as a basis of software tools, platforms, libraries and system upon which both a problem and a corresponding solution can mature. This is supposed to provide insight in the problem area, such that further development on the system can be based on educated decisions. 

\subsection{Vision Scenarios}
\label{sub:essence_vision_scenarios}
The overall goal of creating vision scenarios is to explore the different directions there could be taken in order to solve the problem at hand. In order to create vision scenarios Essence suggests finding opposing lines of development, such as if a system should be intrusive or non-intrusive. Two of such opposites (the most important) are then used as the axes of a two-dimensional graph. Each quadrant in the graph will then contain different solutions to the same problem, vision scenarios.

\subsection{Vision Representations}
\label{sub:vision_representations}

\subsection{Innovation Variants}
\label{sub:essence_innovation_variants}

 

\todo{Overvej at nævne flere Essence ting her hvis vi bruger dem}