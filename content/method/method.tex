%!TEX root = ../../super_main.tex

\chapter{Development Method}
\label{cha:development_method}

\todo[inline]{Rewrite the introduction}
This chapter describes the development method that was used to guide this project. We chose to use Extreme Programming (XP) \parencite{xp}, with a few changes, which is described in this chapter. The main reason why we considered XP, was because we wanted a more concrete approach, with engineering practices, and test-driven method compared to other agile methods such as Scrum, which we have had experience with on previous semesters. Besides using XP we have chosen to also utilize the Essence methodology as well, which is a methodology that aims to support innovative projects, and focuses on enabling developers and other stakeholders to see a problem from multiple perspectives. 

\section{Development Methodologies}
\label{sec:development_methodologies}

In software development, there exist various methodologies. The most common practice in traditional software development is the document driven using a classic water fall approach. This approach emphasizes that the developers should put a lot of effort in planning and separating the development in parts. For instance a software project could be separated into five parts. A part for analyzing and understanding the problem domain at hand, hence producing a requirement specification. A part for designing the product, hence converting requirements to a viable design that describes what technologies should be utilized and the structure and architecture of the system accordingly. A part for testing if the system is stable and requirements are actually met. Finally a part that focuses on maintaining the developed system. This separation allows for specialization, meaning that the specialist in different areas of software development can perform their best. For instance in the testing phase of the development life cycle an experienced tester could perform great tests to ensure that the quality of the final system does what it is supposed to do. However since the delegation of work often shifts from one expert to the another, this development methodology spends a lot of resources on producing documents in order to share knowledge between experts. 

\todo[inline]{Write about agile software development and why this is preferred.}
\todo[inline]{Note til dette: Vi skal skrive et sted at vi ikke har nogen ``kunde'' som vi kan bruge til at lave og verificere krav}
\todo[inline]{Add that a new paradigm (essence) of software development methodologies is being touched upon by us.}

%!TEX root = ../../super_main.tex

\section{Extreme Programming}
\label{sec:extreme_programming}

In this project we have chosen to use some of the aspects from the Extreme Programming (XP) methodology, which is a type of agile software development method, that has a high amount of focus on customer satisfaction, developer productivity/welfare and high product quality. The methodology focuses on delivering partial products to the customers as they need them, instead of just a complete product far in the future. In XP this is normally achieved by use of several core practices, such as: on-site customer, test first development, pair programming, planning poker, continuous integration and delivery and daily meetings. XP emphasizes the value of simplicity, communication, feedback, respect, courage. 

\begin{itemize}
	\item Simplicity is about taking the simplest path to delivering a minimum viable product without compromising quality. 
	\item Communication between every member of the team, including the on site customers, is essential to deliver a product that everybody is proud of. 
	\item Feedback should be taken seriously, and demonstrations should be done early in the process and frequently.
	\item Respect between the team members, the customers, and management is important.
	\item Courage allows the team to take on new challenges without the fear of failing, because nobody works alone. The developers should have the courage to tell the truth about progress and estimates. 
\end{itemize}

The following sections will describe some of the basic concepts from the XP methodology that we have chosen to use, how we have implemented them, and some of the conflicts there are between our project and the methodology.

\subsection{Iterations}
\label{sub:iterations}
Extreme Programming encourages rather short iterations in comparison to other agile development methods. The iterations are of constant length and should be between one to three weeks in length. Because of their constant length they can be used to track the progression and the velocity of a project by evaluating the amount of work done in the end of an iteration compared to the estimates given to the tasks in the given iteration. 
\\\\
In our project we have chosen to use iterations of two weeks (calendar time) in length. Each of these will end with a review of what we have managed to accomplish in the given iteration. One of the issues with using calendar time to schedule our iterations is the fact that we are students, meaning that we have to follow the courses on the semester as well. This makes the amount of time we have available vary between the different iterations, making it more difficult to estimate how much we can handle in each iteration. To try to alleviate this issue we calculate the amount of time we have in a given iteration and take that into consideration when planning it. 

\subsection{Releases}
XP encourages frequent and small releases to the customers. Each iteration ends with a deliverable in XP and these deliverables can be evaluated by the customer, that then can decide if it should be taken into production. The idea behind frequent releases is that the earlier the functionality is used by the customers, the sooner any potential bugs can be found and fixed by the developers. Furthermore, it allows the customers and the team to evaluate if the project is on the right track.
\\\\
We have chosen to make a release every second XP iteration, meaning that we will make a release every four weeks. For each of the releases we will make a major product release, but we will also build a minor product release for iterations that does not end in an XP release if the product is accepted at the given time. All release are pushed to our \href{https://github.com/Shaderex/SW8-Android/releases}{GitHub Page}\footnote{\url{https://github.com/Shaderex/SW8-Android/releases}} and can also be found on the CD in \todo{Add reference to CD appendix}. The deliverables for each iteration we have completed are described in \todo{Ref til appendix med Deliverables}. 

\subsection{Taskboard and User Stories}
\label{sub:taskboard_and_user_stories}
In XP a release plan is created for each release, during a release planning meeting. 
\todo[inline]{Add photo of taskboard}
The user story cards are made in collaboration during release planning while trying to role-play on-site customer. 

The customer role-playing is backed by our vision as seen in \todo{Ref til Essence Vision afsnit}. 

All user stories are accompanied with an acceptance test which is designed together with the user story. 

A potential issue with Extreme Programming is that an on-site customer is required. However, this project is a student project with no paying customer or user. 

This means that we must role-play the on-site customer during the development. This might be challenging because it can be difficult to enter the mindset of a customer. For this reason during the iteration review and planning meetings we assign each team member roles, which they must role-play. This means that one team member is assigned to role-play the customer during the entire review/planning meeting. Other roles include tester, developer, and user. 

This role-playing makes it easier for us, as a team, to prioritize user story cards because one team member only have to consider one mindset during these meetings.

Lastly XP also have techniques to handle uncertainty in a semi structured way, by the use of spikes, which is used to explore areas with little to no knowledge. 

\subsection{Daily Standup Meeting}
\label{sub:daily_standup_meeting}
Knowledge sharing is encouraged in XP by using different practices such as daily stand up meetings and pair programming. For us, knowledge sharing is important, since all members of the group should have insight in the product and the written report. 

\subsection{Test Driven Development}
\label{sub:test_driven_development}
Extreme Programming seemed attractive to us due to the fact that the method is test-driven. We have been lacking structured test and evaluation during the development of the previous semester projects, where tests and evaluation were done in an ad-hoc fashion. In an agile development method, we feel that a test-driven, or a test-first approach would help us solve this problem and achieve a more structured approach to enforce testing.

\subsection{Documentation}
Another potential issue with XP could be that it does not have any built in practices or techniques to produce a technical student report during development. XP does not encourage documentation. Documentation is only considered in XP if it gives the customer value. We had to adapt our development method to include the production of a technical student report. Another potential pitfall was for us, when using XP, the lack of quality assurance (QA) team assigned to the project. A QA team assists the development team to spot potential issues with acceptance test. Because of the fact we have not been working solely with XP in previous semester projects we cannot evaluate the quality of acceptance test based on prior experience.








%!TEX root = ../../super_main.tex

\section{Essence}
\label{sec:essence}
% A software development methodology that aims to support high value software solutions is Essence \parencite{essence_book}. Essence mentions having a vision as a great way to start a project. Essence mentions that having different representations of a vision can help elicit objects, events, and qualities to persue. Four different representation types are suggested, namely: Icon, Prototype, Metaphor, Proposition. We have chosen to attempt to represent the existing condition, i.e. the problem area as we understood it at the time, as Icons, Metaphors, and Propositions. We found the prototype representation too costly in terms of development time. 

% Vi har ikke nogen udefrakommende kunne, vi vil håndtere dette med et essence vision
The aim of the project was to be innovative without having an external customer who could define and verify requirements. In this project we dedicated some of our development time to employing elements from the Essence methodology, which has been created by, and is currently being revised by, Ivan Aaen, a lector at Aalborg University. Essence is a methodology that supports software development teams in creating innovative and high value software solutions. It focuses more generally on translating great ideas into great solutions, while dealing with insecurities during the exploration of possible solutions. In this section we define some of the concepts that we utilized from the Essence methodology and give insight in how these helped us in our development.



% \subsection{Start Configuration}
% \label{sub:essence_start_configuration}

% Essence describes a start configuration as a basis of software tools, platforms, libraries and system upon which both a problem and a corresponding solution can mature. This is supposed to provide insight in the problem area, such that further development on the system can be based on educated decisions. 

\subsection{Vision Scenarios}
\label{sub:essence_vision_scenarios}

Essence introduces a concept of vision scenarios with the overall goal of exploring different directions that could be taken in order to solve the problem at hand. In order to create vision scenarios, Essence suggests finding opposing lines of development. This could for instance be whether a system should be intrusive or non-intrusive. The two most important pairs of opposites can then be used as the axes of a two-dimensional quadrant system. Each quadrant in the system then corresponds to different approaches for a solution to the same problem. These different quadrants are called vision scenarios.

\subsection{Vision Representations}
\label{sub:vision_representations}

The four vision scenarios, represented by the quadrants, can then be used to discover different ways of describing the system, such as using icons, metaphors, prototypes or propositions. If one or more of such descriptions or representations are created for each quadrant, they can be used as catalysts for the discussion of the project focus and qualities of a potential solution. 

%\subsection{Variants of Software Innovation}
%\label{sub:essence_innovation_variants}

%Essence describes four different ways that a system can be innovative. It can be innovative in the way that it will introduce a new or radically changed software-intensive solution to a problem. This is called product innovation. Another of the innovation types is process innovation, where the software-intensive system should offer a new or improved way of producing other solutions. The third type of innovation is project innovation, where it is about utilizing existing solutions in another application domain than it was initially intended. The last innovation variant is called paradigm innovation. This form of innovation is when a software intensive solution is changing the mindset of what users, an organizations or a market is. These innovation variants are not mutually exclusive meaning that a solution can be both paradigm and process innovation. Considering how a problem can be solved in terms of these different innovation variants can help explore different ways of thinking about the problem, and help discover new potential solutions to the problem.

\todo[inline]{Beskriv flere Essence ting her hvis vi bruger dem (Afhænger af Rikkes svar på Configuration Table)}

\subsection{Combination with Extreme Programming}
\label{sub:combination_with_extreme_programming}
Essence is a useful methodology for keeping a development team on the same page at all times during software development, by creating a vision for the project and maintaining it iteratively throughout the development. We aimed to review our vision and progress iteratively throughout the project, but we did not believe that we would be able to perform a full revision of the project between each iteration. Instead, we decided to perform a revision after each release, and we thereby hoped that our vision for the project would be more stable, using Essence in combination with XP.


