%!TEX root = ../../super_main.tex

\section{Extreme Programming}
\label{sec:extreme_programming}
% Basic idea
% Core practices 
% -TDD
% -Pair programming
% -Planning
% -CI
% -Standup meeting
% Time boxing
% Spikes, user stories, tasks, task board (images)
% Describe the things that we use or everything 
% How we differ

Extreme Programming seemed attractive to us due to the fact that the method is test-driven. We have been lacking structured test and evaluation during the development of the previous semester projects, where tests and evaluation were done in an ad-hoc fashion. In an agile development method, we feel that a test-driven, or a test-first approach would help us solve this problem and achieve a more structured approach to enforce testing. Furthermore knowledge sharing is also encouraged in XP by using different practices such as daily stand up meetings and pair programming. For us, knowledge sharing is important, since all members of the group should have insight in the product and the written report. Using these practices, we achieved a higher level of knowledge sharing and at the same time increase the quality of the software produced\todo{Vi ved sådan set ikke noget om det her endnu, så omformuler det hvis det ikke er sandt i fremtiden}. Extreme Programming encourages rather short iterations in comparison to other agile development methods. These short iterations have assisted us to steer the project with higher accuracy between iterations. Lastly XP also have techniques to handle uncertainty in a semi structured way, by the use of spikes, which was used to explore areas with little to no knowledge. 
\\\\
A potential issue with Extreme Programming is that an on-site customer is required. However, this project is a student project with no paying customer or user. This means that we must role-play the on-site customer during the development. This might be challenging because it can be difficult to enter the mindset of a customer. For this reason during the iteration review and planning meetings we assign each team member roles, which they must role-play. This means that one team member is assigned to role-play the customer during the entire review/planning meeting. Other roles include tester, developer, and user. This role-playing makes it easier for us, as a team, to prioritize user story cards because one team member only have to consider one mindset during these meetings.

\subsection{User Stories and Acceptance Tests}
The user story cards are made in collaboration during release planning while trying to role-play on-site customer. The customer role-playing is backed by our vision as seen in \todo{Ref til Essence Vision afsnit}. All user stories are accompanied with an acceptance test which is designed together with the user story. Our experience shows that a user story and an acceptance test coexists in a mutual relationship. A user story draft is usually formulated first and a draft for an acceptance is then designed. The user story and acceptance test drafts are then revised back and forth until everyone claims to have understood both and have agreed on their semantics. User stories and their corresponding acceptance tests can be seen in \appref{app:user_stories_and_acceptance_test}.

\todo{Overvej at nævne flere Essence ting her hvis vi bruger dem}

\subsection{Documentation}
Another potential issue with XP could be that it does not have any built in practices or techniques to produce a technical student report during development. XP does not encourage documentation. Documentation is only considered in XP if it gives the customer value. We had to 
adapt our development method to include the production of a technical student report. Another potential pitfall was for us, when using XP, the lack of quality assurance (QA) team assigned to the project. A QA team assists the development team to spot potential issues with acceptance test. Because of the fact we have not been working solely with XP in previous semester projects we cannot evaluate the quality of acceptance test based on prior experience.

\subsection{Deliverables}
% Vi laver delafleveringer 
Each iteration must end with deliverables in XP and these deliverables must be evaluated by the customer. We have two XP iterations per XP release. We build a minor product release for XP iterations that does not end in an XP release and we build a major version for iterations that ends in an XP release. All release are pushed to the project \href{https://github.com/Shaderex/SW8-Android/releases}{Project GitHub Page}\footnote{\url{https://github.com/Shaderex/SW8-Android/releases}} and can also be found in the CD appendix \todo{Add reference to CD appendix}. 
\\\\
The purpose of these deliverables is for the project customer to evaluate if the project is on the right track and to evaluate if the resulting deliverable is valuable. The deliverables for each iteration we have completed are described in \todo{Ref til appendix med Deliverables}. 

\todo[inline]{Det her er nogle af de idéer og holdninger vi har haft tidligt i projektet. Det skal reflekteres over hvad der er godt og skidt når vi kommere senere hen i projektet}

\subsection{Planning Poker Reflections}
We have discovered that planning poker, which we have used to estimate tasks in iteration planning, is more valuable to us than simply being a technique for tasks estimation. Many ambiguities in task cards were discovered automatically by estimating and having short discussions about a given task. We learned that time estimation discussions quickly transformed into discussions about the semantics of the tasks when there were significant disagreements. We embraced these discussions and found that they were great means to resolve ambiguities. Time estimation discussions, when performing planning poker, are in general timeboxed, but we allowed longer discussions when we felt that task cards should be updated in order to resolve ambiguities or if we felt that tasks should otherwise be split or clarified.


