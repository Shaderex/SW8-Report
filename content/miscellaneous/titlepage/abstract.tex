%!TEX root = ../../../super_main.tex

Computers are transforming into small ubiquitous mobile devices. Development of new technologies leaves a great untapped potential, regarding the data from the sensors in these devices. 
Reality mining is an area of research that tries to utilize this untapped potential to understand human dynamics. However, the task of gathering sensor data for this type of research can require a lot of effort in regards to developing systems for gathering the data and also have to persuade participants to contribute.
We introduce a platform called \textbf{uMiner}, which facilitates a more effortless way of gathering this kind of data for interested parties. The platform aims to solve all the tedious technical work of such a system, and only leaves the persuasion of participants to the users.
We design and implement a client-server architecture that by utilizing different design patterns and technologies provides support for securely gathering, labeling, compressing, uploading, and storing the continuous stream of sensor data from ubiquitous devices. 
By utilizing the data from our system, we create a model for detecting if a subject is performing certain actions. This shows that the data we collect is useful for creating models of human dynamics.  
By having this platform available we might enable new types of research that uses sensors from ubiquitous to improve knowledge regarding human dynamics or products that are aware of contexts.