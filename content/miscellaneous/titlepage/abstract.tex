%!TEX root = ../../../super_main.tex

Computers are transforming into small ubiquitous mobile devices. Development of new technologies leaves a great untapped potential, regarding the data from the sensors in these devices. 
Reality mining is an area of research which can be applied to utilize this untapped potential in order to understand human dynamics. However, the task of gathering sensor data for this type of research can require a lot of effort in regards to developing systems for gathering the data and also having to persuade participants to contribute.
We introduce a platform called \emph{uMiner}, which facilitates a more effortless way to gather this kind of data for interested parties. The platform aims to solve all the tedious technical work of such a system, and only leaves the persuasion of participants to its users.
We design and implement a client-server architecture that, by utilizing different design patterns and technologies, provides support for secure gathering, labeling, compressing, uploading, and storing the continuous stream of sensor data from ubiquitous devices. 
By utilizing the data from our system, we create a simple model for detecting whether a device is pocketed or not. This shows that the data which can be collected with \emph{uMiner} is useful for creating a predictive model.  
By having this platform available, we might enable the development of new products that are context aware or new types of research that uses sensors from ubiquitous devices to improve knowledge regarding human dynamics.
