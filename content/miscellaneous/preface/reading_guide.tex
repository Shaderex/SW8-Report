%!TEX root = ../../../super_main.tex

\section*{Reading Guide}

This report is written in chronological order and should be read as such. Sources are referenced with the Chicago style method of citations, for instance \parencite{android_adb}. A lexicographically sorted list of references is placed in the Bibliography on Page \pageref{bibliografi}. Personal pronouns throughout this report refer to the authors of the report. Throughout the report we refer to customers and users, where customers in the context of this report means a person in need of labeled training data and therefore conducts the data gathering, while a user refers to a person who is using the application and answering the questionnaire. 
\\\\
In the context of our work methodology we will also refer to customers, which is a core aspect of the Extreme Programming method. This customer is not necessarily the same customer as the customer who will use the finished application, but also does not have to be different. Assuming that one of the people in need of training data approached us and we agreed to build his application, the two customers would be the same person. Assuming that it was some third party who wanted to build an application to facilitate the process for the customers mentioned previously, that could also hold. Speaking generally, customers should be considered a person who desires the app in order to facilitate datacollection, and for convenience we assume it to be best to consider the methodology customer and the application customer to be the same. 