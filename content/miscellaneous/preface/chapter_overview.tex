%!TEX root = ../../../super_main.tex

\section*{Report Overview}
\label{sec:report_overview}

The report is divided into three parts: \emph{\nameref{prt:project_presentation}}, \emph{\nameref{prt:feature_description}}, and \emph{\nameref{prt:final_thoughts}}.
\\\\
\prtref{prt:project_presentation} starts with a motivation of the project in \charef{cha:motivation}. This is followed by a description of our development method in \charef{cha:development_method}. Lastly, this part analyzes the problem area in \charef{cha:problem_analysis} and determine the scope of the project in \charef{cha:scope_of_project}. 
\\\\
\prtref{prt:feature_description} covers the architecture of our system in \charef{cha:architecture}, which is followed by a description of how we solved some of the security concerns that we have dealt with in \charef{cha:security}. Then the part continues by covering the backbone of our client application and the different user interfaces in \charef{cha:gathering_sensor_data} and \charef{cha:user_interfaces} respectively. Lastly, the part will cover how we ensured the quality of the developed product in \charef{cha:quality_assurance}.
\\\\
The report concludes in \prtref{prt:final_thoughts}, which firstly reflects upon our development method and the developed product in general in \charef{cha:reflection}. Then the conclusion of the project will be presented in \charef{cha:conclusion}, and lastly some of the thoughts we have had regarding further work in \ref{cha:future_work}.