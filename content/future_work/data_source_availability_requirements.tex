%!TEX root = ../../super_main.tex

\section{Data Source Availability Requirements}
\todo[inline]{Læs om de to paragraffer udkommenteret er skrevet godt nok sammen til denne}

In the current system, when customers have added sensors, they are not guaranteed to be in the provided snapshots from the participants. This can be due to either the sensors are turned off, or are not available. An improvement to the system would be to have different levels of requirements for sensors when the customers creates the campaigns. These levels could be that sensors are either required or desired, meaning that in order for a participant to be able contribute to a specific campaign, he has to have the required sensors available, but not the desired sensors. This could be useful for customers if they need readings from specific sensors, while other readings are only nice to have. This could for instance be that a customer has specified a necessary sensor as the accelerometer from the smart phone, and a desired sensor as an accelerometer from an smart band. He could then use the readings from the smart phone corresponds to the ones from the smart band, if it is available, to check for correctness of the readings. A potential problem with this is, that the base of participants could decrease, as some participants do not have all sensors. Another potential problem with this is, that we should consider how to handle wearable devices and their sensors. The is due to wearable sensors not always being within reach of the device running the application, which would cause some issues on the client side and mess up the dataset for the customer.

% We have thought about allowing customers specify different levels of requirements for sensor and data source availability. It might be acceptable for some data sets to have missing values for some of the sensors while requiring that readings from other sensors must be present. This could for instance mean that some campaigns might require the presence of a certain sensor, say a heart rate sensor, while having for instance the Wi-Fi data source optional and only included if available, i.e. turned on by the participant. The different levels of requirements for sensor availability could both be in regards to the actual capability of devices involved by the participant, i.e. that the involved devices must support the desired readings, and in regards to the availability of the reading, i.e. in cases where the devices include the data source or sensor but it is for some reason temporarily unavailable.
% \\\\
% The sensors that customers add, when they specify a campaign, are not guaranteed to be in the snapshots the participants provide. This means, that if a customer specifies that he wants data from the accelerometer and barometer, he has no guarantee to get measurements from these sensors. The only guarantee we provide, is that the customer will get the snapshots if these sensors are available on the device of the participant. An improvement to the system would be to let the customer specify sensors to be either required or desired, meaning that he can be guaranteed that all the snapshots will contain for instance barometer data. However, if this was to be implemented, the potential base of participants would decrease, because some participants do not have all sensors. If our system allowed this type of restrictive specification, we would furthermore have to consider how to handle wearable devices and their sensors. This is due to wearable sensors not always being within reach of the device running the application, which would cause some issues on the client side and mess up the dataset for the customer.
