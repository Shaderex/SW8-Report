%!TEX root = ../../super_main.tex

\section{Pagination of Campaigns}
\label{sec:pagination_of_campaigns}

In the \mono{PublicCampaignFragment} of the app we have developed, we currently download all the campaigns that are present in the database. While we ensure that we only download the minimum amount of information about each campaign necessary to show the participant at this point, we still download the entire list. This could potentially consume a high amount of the participant's memory, which is not desirable. It will not crash the view, because we use the \mono{Adapter} class from Android with a \mono{ViewHolder} pattern, which ensures that we only generate views for the elements the participant can see, and we also reuse views if possible.
\\\\
The main issue is that, even if the participant only wants to see the first 20 campaigns, he might have to download 5000. This could be solved by having some type of pagination (see \secref{sub:viewing_campaign_details}), where we only download some amount of campaigns and then upon a request from the participant, present him with some new amount. This would preserve the internet bandwidth of the participant, as well as prevent unnecessary consumption of memory.