%!TEX root = ../../super_main.tex
\section{User Evaluation}

A natural next step for this project would be to find and introduce collaborators, who are actually interested in collecting labeled training data, to help evaluate the system and possibly help them. This could perhaps be done as beta testing. We need to test how the system performs out of the lab. How the system performs when uploading snapshots on a slow Wi-Fi connection, and if participants completes the campaigns, or just uninstall the application after a few minutes.
\\\\
Introducing actual customers could help us figure, if we are including support for enough sensors and data sources in order satisfy their data needs. Many different customers might require a special sensor or data source, that is not supported, which would be essential for their data needs. We could then begin to evaluate which sensors and data sources should be implemented next. Our website might also reveal usability problems or other user patterns for data collection, which we have not thought of yet.
\\\\
Introducing participants could also help evaluate the Android application. We have to so far only tested the Android application on a limited set of Android Devices, and we do not know if our Android application has problems running on some common configuration, i.e. a certain device and Android version. Configuration testing could however also be done without involving users as stated in \secref{sec:extension_of_continuous_integration_server}.  