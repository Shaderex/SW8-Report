%!TEX root = ../../super_main.tex

\chapter{Future Work}

\todo[inline]{Skriv noget med at hvis customers har en campaign at de skal kunne specificere om nogle af sensorerne er enten required eller optional}

\section{Network Bound Battery Consumption}
\label{sec:future_work_network_bound_battery_consumption}

We have in \secref{sub:background_sensor_service_snapshot_generation_and_synchronization} described how using a \mono{GcmNetworkManager} service could help us improve battery consumption. Future development of the Android application should attempt to implement and use such a service and evaluate whether or not using such an API actually saves battery.

\subsection{Server workload distribution}

Another dimension which would have to be considered in case batching of network tasks is implemented and in case we implement constraints on when campaign data should be uploaded to a server, for instance require that the device should be charging, is the distribution of workload on the server. Requiring that the phone is charging would for instance likely distribute workload towards the nightly hours where people tend to charge their mobile devices while sleeping.  

\subsection{Data Source Availability Requirements}

We have thought about allowing customers specify different levels of requirements for sensor and data source availability. It might be acceptable for some data sets to have missing values for some of the sensors while requiring that readings from other sensors must be present. This could for instance mean that some campaigns might require the presence of a certain sensor, say a heart rate sensor, while having for instance the Wifi data source optional and only included if available, i.e. turned on by the participant. The different levels requirements for sensor availability could both be in regards to the actual capability of devices involved by the participant, i.e. that the involved devices must support the desired readings, and in regards to the availability of the reading, i.e. in cases where the devices include the data source or sensor but it is for some reason temporarily unavailable. We have however not implemented this and data is simply not included in the snapshots if the sources where unavailable at the time.

\section{Utilizing Wearables for Interaction}
\label{sec:utilizing_wearables_for_interaction}

In our solution we only notify participants about pending questionnaires on their mobile phones, which is a suboptimal solution when users have other smart wearables available. Many new smart wearables offer new methods of interaction, such as voice interaction, gesture recognition, etc, which, in contrast to the hassle of having to extract a phone from the pocket, could provide new interesting ways of answering questionnaires. It is for example easy to imagine a scenario where a participant is at a situation where it would be frowned upon to pick up your phone and start interacting with it, but where it would be completely fine to look at your watch and tap it a couple of times to answer a questionnaire quickly. 
\\\\
In order to facilitate a scenario such as this, our application would have to be extended in a way that allows it to push notifications to the smart wearable and opening some activity where the user can answer the questionnaire. Furthermore the wearable would have have to have support for displaying questionnaires, or in some other way telling the participant what he/she has to answer, e.g. reading the questionnaire out loud. Additionally the wearable must make it possible to interact with it using some type of touch or gesture. If the wearable has support for these things, there are some considerations regarding our app having to be notified about the questionnaire answers, by whichever app handles the communication with the wearable. 