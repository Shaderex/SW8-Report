%!TEX root = ../../super_main.tex

\section{Security}
\label{sec:security}

In general the security in the system is at a place where it is close to meet all statutory demands. The only problem is that the key for the encryption of the remote data is located on the same system as the data. This leaves the data open if our server ever were to be compromised. Another issue that is solved on the device at least is the problem with using a self-signed certificate. In the browser the users will be warned that our certificate is not to be trusted, because it is not trusted by any certificate authority. This could potentially lead to problems where customers think they are accessing our servers but instead are communicating with a server with malicious intent. However, this is not that big of an issue, because the personal data is not uploaded in this way, but only through the Android client. There are still the issue of customers might be intimidated by the warning shown to them when visiting the site through a browser that does not trust our certificate, which might make them choose not to sign up or use our tool. This could be solved by buying a certificate from a trusted certificate authority. 
