%!TEX root = ../../super_main.tex

\chapter{Reflection}
\label{cha:reflection}

% ==== METHOD REFLECTION ====
\subsection{Planning Poker Reflections}
We have discovered that planning poker, which we have used to estimate tasks in iteration planning, is more valuable to us than simply being a technique for tasks estimation. Many ambiguities in task cards were discovered automatically by estimating and having short discussions about a given task. We learned that time estimation discussions quickly transformed into discussions about the semantics of the tasks when there were significant disagreements. We embraced these discussions and found that they were great means to resolve ambiguities. Time estimation discussions, when performing planning poker, are in general timeboxed, but we allowed longer discussions when we felt that task cards should be updated in order to resolve ambiguities or if we felt that tasks should otherwise be split or clarified.
\\\\
The short iterations have assisted us to steer the project with higher accuracy between iterations. 
\\\\
Using standup meetings and pair programming, we achieved a higher level of knowledge sharing and at the same time increase the quality of the software produced\todo{Vi ved sådan set ikke noget om det her endnu, så omformuler det hvis det ikke er sandt i fremtiden}. 
\\\\
Our experience shows that a user story and an acceptance test coexist in a mutual relationship. A user story draft is usually formulated first and a draft for an acceptance is then designed. The user story and acceptance test drafts are then revised back and forth until everyone claims to have understood both and have agreed on their semantics. User stories and their corresponding acceptance tests can be seen in \appref{app:user_stories_and_acceptance_test}.

%!TEX root = ../../super_main.tex

\section{The Android Platform, Compatibility and Test Driven Development}
\label{sec:the_android_platform_compatibility_and_test_driven_development}

\todo[inline]{Skriv om at vi har brugt meget lang tid på at sætte udviklings miljø op.. At tiden måske kunne være anderledes prioriteret}

% Uniform environment for easier knowledge sharing
%  - Android Studio
%   - code/checkstyle
%  - Phpstorm
%  - Laravel
%  - Bootstrap
%  - Jenkins
%  - Postgres

% Skriv om ovenstående her.
 

% Test
%  - Unit test
%   - Sensor data hard to mock
%  - Instrumental test 
%   - While developing -> own phones
%   - CI also requires phone
%   - Android fragmentation
%   - New OS
%    - Rooting
%    - Problems 
%    - Galaxy nexus 4.3 -> 5.0
%   - Migrating robolectric tests to instrumental

