%!TEX root = ../../super_main.tex

\section{RESTful Maturity}
\label{sec:restful_maturity}

The Application Programming Interface (API) that we have implemented is a good beginning for a full RESTful API. However, there are certain aspects that are missing in the resource representations that we send to the clients. If our API should be truly RESTful, we should be following the principle of Hypermedia As The Engine Of Application State (HATEOAS), which specifies, that any resource should in itself present hyperlinks to every modification or other interaction with the object that could be taken. This means that when sending the list of campaigns to the client we should not only provide the id, the author, and the title of the different campaigns. It should also provide the link used to fetch all the information about the campaign; the link used for joining the campaign; and the link for uploading snapshots to the specific campaign. There exists a model for evaluating ``how RESTful'' an API is, called Richardson's Maturity Model \parencite{richardsons_model}, which rates RESTful API's on a scale from 0 to 3. 

\begin{description}
	\item[Level 0] is not RESTful at all.
	\item[Level 1] is when one is starting to use the correct Unique Resource Identifiers.
	\item[Level 2] is when you utilize the correct HTTP request methods in the API.
	\item[Level 3] is when you utilize HATEOAS and thereby is truly RESTFul.
\end{description}

We believe that we would be placed on level 2, following all principles but the HATEOAS principle. However, the creator of the REST principles, Roy T. Fielding, writes in a blog post that he does not see the an API as RESTful unless it is hypertext driven (following HATEOAS)\parencite{http_manden_blog}, meaning that we, according to him, not are RESTful at all. 
