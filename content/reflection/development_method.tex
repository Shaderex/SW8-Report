%!TEX root = ../../super_main.tex

% ==== METHOD REFLECTION ====
\section{Development Method}
\todo[inline]{Skriv noget introducerende}
Our experience shows that a user story and an acceptance test coexist in a mutual relationship. A user story draft is usually formulated first and a draft for an acceptance is then designed. The user story and acceptance test drafts are then revised back and forth until everyone claims to have understood both and have agreed on their semantics. This help us get a mutual understanding of the task and also solve potential ambiguities. 
\\\\
Another aspect that helped us clarify the intention and solve ambiguities was by estimating the user stories. 
We discovered that planning poker, which we have used to estimate tasks in iteration planning, is more valuable to us than simply being a technique for tasks estimation. Many ambiguities in task cards were discovered automatically by estimating and having short discussions about a given task. We learned that time estimation discussions quickly transformed into discussions about the semantics of the tasks when there were significant disagreements. We embraced these discussions and found that they were great means to resolve ambiguities. Time estimation discussions, when performing planning poker, are in general time boxed, but we allowed longer discussions when we felt that task cards should be updated in order to resolve ambiguities or if we felt that tasks should otherwise be split or clarified.
\\\\
Even though the discussion during estimation made the tasks more clear, it made estimation a long task. Nearing the end of the project period (approximately the last two-three iterations) the time schedule was becoming increasingly tighter, and we therefore changed from estimating all created tasks to only estimating tasks until we had enough work hours to fulfill the iteration work schedule. By making this change, we wanted the developers to feel that they were not wasting time, and we therefore embraced this shift in development methodology. Even though this change was made we still were not at a pace that would yield a product that we could be satisfied with. Therefore, the last iteration planning was made quite differently. Everyone contributed in the creation of the tasks and user stories, but when they were to be estimated, only two person was involved, whilst the other members of the group worked on the report and product. Even though it might seem controversial to estimate the task in such a way, the estimated time was close to the actual spent time. The benefit was not only that other developers could focus on other tasks during the estimation, but the time spent on the estimation was low compared to the usual planning meetings. Even though we lost the group wide discussion there were no problems with ambiguities or tasks otherwise being unclear. 
\\\\
Another aspect that worked well for us during the planning meetings was the roles that we used. 
\todo[inline]{Skriv om roller var godt}
The short iterations we used during the development have assisted with the steering of the project as was expected. It allowed us to have higher accuracy between iterations, which effectively meant that it was easier for us to focus on individual areas and know which areas needed attention, meaning that it also increased the knowledge sharing in the group. The  ceremonies that were supposed to increase the knowledge sharing in the group, namely stand up meetings and pair programming, also worked as expected. We have in earlier semesters experienced that the daily stand up meetings would over time stop being daily and later even skipped for very long periods of time. However, during this semester this has not been the case, which we believe can be contributed to the fact that we did not use a task board but rather a physical one, which was used as a center point for the meetings. 
\\\\
\todo[inline]{Skriv om hvor godt essence har virket for os}

\subsection{Continuous Integration}
A place where we used a lot of time early on in the project was on our continuous integration server. We do not believe that we have gained as much as we have spent on issues with the server. Even though it did discover errors during the development period, we used a lot of time getting the instrumental tests to work on the server, and fix errors that occurred only because of hardware failures on the server or the equipment attached to it. One could consider that having a continuous integration server for a project only spanning over four months might be too time consuming to integrate, however, we have gained knowledge about the setup for such a server and have also used the server for serving our web application. 






