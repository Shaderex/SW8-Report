%!TEX root = ../../super_main.tex

\section{Motivation}
\label{sec:motivation}
% mobile devices er blevet mere udbredt blandt folk --> giver mulighed for at indsamle data
% Svært at komme et sted hen (i DK) uden at der er mobile devices i nærheden
% Nævn fluff om at det er en område under udvikling
% ``ubiquitous''

The area of ubiquitous devices is rapidly changing, and especially mobile technologies have in recent years had a major growth in consumers. Our definition of ubiquitous devices includes, but is not limited to, mobile phones, smartwatches, smartbands, smart earplugs. We use the word ubiquitous because there exist so many different kinds of smart devices and they are becoming increasingly omnipresent over time. The sales of smartphones have increased in sales by a factor of 11 since 2007 \parencite{statsia_smartphones}. Other mobile technologies, like wearables, have also seen an increase in demand in the last couple of years. FitBit, a company selling smart wristbands, have seen a exponential increase in sales of their device in the past five years \parencite{statsia_fitbit}.  
\\\\
There are many emerging technologies related to ubiquitous computing that are difficult to ignore, due to their potential to change the way we think about mobile technology and ubiquitous mobile devices in general. An example of this is the Google Contact Lens \parencite{google_contact_lens} which is able to measure the blood sugar level of the wearer and transmit it wirelessly, which is supposed to be a quality of life improvement for diabetics. Another new technology that has emerged is the ``big.LITTLE'' ARM CPU architecture \parencite{big_little_architecture}, which is a multi-core processor architecture that allows for better adjustments to dynamic computing needs and use less power when possible. It combines the best of power hungry and powerful ARM CPU cores with the best of slow, power efficient ones to achieve overall better performance and power consumption. This architecture allows for battery draining high performance on demand and very power efficient background processing when a device is not used. This is supposed to be a huge battery-improvement for mobile devices, which mean the introduction of a plethora of other technologies that benefit from increased battery lifetime. 
\\\\
New mobile sensor technologies are continuously being made which gives access to additional wearable sensors. Because these types of sensors often cannot be implemented in the smartphones of today, some information can only be measured using sensors located on other parts of the body, such as the Google Contact Lens' ability to measure blood sugar, or heart rate monitoring from smart bands or smart earplugs. Even sensors that can be implemented in smartphones are not guaranteed to be implemented in all phone models, which is exemplified by the fact that some manufactures have value editions, where a low price is preferred over many sensors. For instance Motorola sells both Moto E (a value edition) and Nexus 6 (a flagship model) where the price and specification varies \parencite{moto_e_compared_to_nexus_6}

\subsection{Reality Mining}
\label{sub:reality_mining}
The expansion in mobile technologies opens new opportunities for studying and understanding the context around the users more accurately. This means that we can obtain better knowledge regarding the users of a system and understanding people in general. A common practice today for deriving the context around the users is to ask them explicitly by surveys or similar techniques. A field of research, where a less biased description of the context would be appreciated is \emph{reality mining}, which is a study of human behavior and social activities\todo{Rikke skriver til ``which is a study of human behavior and social activities'': ``Og noget mere? Baseret på mining data eller hvad?''} \parencite{madan2009_reality_mining_privacy}. Reality mining is a special area of data mining, where machine intelligence principles are applied to labeled training data, where the dataset are gathered from electronic sensors in everyday scenarios. Reality mining gives researchers and industry an alternative to basing research and models solely on data provided through people, i.e. from public polls, focus groups, and questionnaires. In order to understand human dynamics with less bias, a study from 2006 \parencite{eagle2006_reality_mining_definition} showed that context could be derived from people and their behavior. The study was conducted on 100 people, and their phone, and showed promising results\todo{Lyder lidt underligt}. The study furthermore concluded that data from new types of sensors could have a lot of potential in regards to different fields based upon reality mining. 
\\\\
The understanding of human dynamics that reality mining has the potential to provide is applicable in various fields of study, for example Health and Medicine \parencite{pentland2009_reality_mining_health_medicine} which includes Telehealth \parencite{telehealth_aau}. This area uses telecommunication to improve the general health of citizens. Reality mining could potentially enable a better understanding of the human psyche in a more implicit matter, and thereby assist in improving the telehealth field of knowledge. One could imagine that an application that uses this field of research could replace medicine used in psychiatric areas. Instead of prescribing antidepressants, a patient would get an application installed on their smart devices which will give them tools to handle their mental state in different contexts. In the same way one could imagine applications for people struggling with drugs or alcohol, where a Disulfiram \parencite{nlm_disulfiram} treatment could be replaced or decreased by introducing an application that understands the mental state of alcoholics. Reality mining is not solely a field applicable in health related studies, but also fields such as automobile traffic congestion \parencite{pentland2009reality_mining_mobile_communication_gps} may benefit from learning human dynamics more accurately.
\\\\
Generally speaking, reality mining is reliant on some context where the user exists and some definition of what this context describes. Because of the huge variety of different contexts and definitions that can be ``mined'', interested parties in need of this type of information need to define what kind of context they desire by specifying what and how they want the labeled training data to be gathered. This is an issue, because that means that every interested party has to define his own reality mining procedure, find their own mining subjects, etc. It would therefore be interesting to look at how the process can be generalized and streamlined, such it will be easy to get desired training data. One thing to consider is that the gathered data is personal (in relation to the one whose context is being mined), and for that reason, privacy is a concern with reality mining \parencite{madan2009_reality_mining_privacy}. 
\todo[inline]{Maybe write who the potential customers could be here?}

%Big data is generally becoming a huge competitive advantage for large companies - \parencite{lavalle2011big}. 
%\todo{Small Data}